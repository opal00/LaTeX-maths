\begin{longtblr}[
    caption={Notations}
    ]{
        colspec={|Q[m,c,2]||Q[m,c,4]|}, width = 0.8\linewidth,
        rowhead = 1, 
        hlines={0.4pt, black},
        row{odd} = {myorange!10}, row{1} = {myorange, fg=white, font=\bfseries}
    }
    Signifiant & Signifié \\
    Théorie des ensembles & \\
    $\card(E)$, $\abs{E}$ & Cardinal de $E$ \\
    $\mathcal{P}(E)$ & Parties de $E$ \\
    Algèbre générale & \\
    $\mathbb{K}$ & Le corps $\mathbb{R}$ ou $\mathbb{C}$ \\
    $\mathbb{K}(X)$ & Corps des fractions rationnelles \\
    $\mathbb{K}[X]$ & Anneau des polynômes \\
    $o$, $\mathcal{O}$, $\sim$ & Notations de Landau \\
    $\mathfrak{S}_E$, $\mathfrak{S}_n$ & Groupe des permutations, groupe symétrique \\
    $\GL(E)$, $\GL_n(\mathbb{K})$ & Groupe linéaire \\
    $\O(E)$, $\O_n(\mathbb{K})$ & Groupe orthogonal \\
    $\SO(E)$, $SO_np$ & Groupe spécial orthogonal \\
    Algèbre linéaire & \\
    $\mathcal{L}(E,F)$ & Espace vectoriel des fonctions linéaires de $E \to F$ \\
    $\mathcal{L}_c(E,F)$ & Espace vectoriel des fonctions linéaires continues de $E \to F$ \\
    $\spr{.}{.}$ & Produit scalaire \\
    $\norm{.}$ & Norme \\
    $N_{op}$ & Norme d’opérateur \\
    Calcul matriciel & \\
    $\mk{n,p}$ & Matrices de tailles $n \times p$ à coefficients dans $\mathbb{K}$ \\
    Calcul intégral & \\
    $\L^1(I)$ & Fonctions intégrables sur $I$ \\
    Théorie de la mesure & \\
    $\mathcal{B}(E)$ & Boréliens de $E$ \\
    Probabilités &  \\
    $\P$ & Probabilité \\
    $\E$ & Espérance \\
    $\V$ & Variance \\
    $\cov$ & Covariance \\
    $\mathcal{L}^p(\Omega, \mathbb{P})$ & Espace vectoriel des VA admettant un moment d’ordre $p$ sur l’ensemble $\Omega$ pour la probabilité $\mathbb{P}$. \\
\end{longtblr}

\newpage

\begin{longtblr}[
    caption={Acronymes}
    ]{
        colspec={|Q[m,c,2]||Q[m,c,4]|}, width = 0.8\linewidth,
        rowhead = 1, 
        hlines={0.4pt, black},
        row{odd} = {mypurple!10}, row{1} = {mypurple, fg=white, font=\bfseries}
    }
    Signifiant & Signifié \\
    DCC & Dans ce cas \\
    DF & Dimension finie \\
    DL & Développement limité \\
    DSE & Développement en série entière \\
    EP & En particulier \\
    (S)EV & (Sous) Espace vectoriel \\
    OA & On appelle \\
    OC & On considère \\
    ON & On note \\
    OS & On suppose que \\
    PS & Presque sûrement \\
    SSI & Si et seulement si \\
    VA(D)(I)(ID) & Variable.s aléatoire.s (discrète.s) (indépendantes) (identiquement distribuées) \\
\end{longtblr}

\newpage