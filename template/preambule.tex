%%-----PAGE-LAYOUT-----%%

\usepackage[left=2cm, right=2cm, top=2cm, bottom=2cm, headheight=1cm, headsep=.5cm, footskip=1cm]{geometry}

%-----USEFUL-PACKAGES-----%

\usepackage{comment}
\usepackage{xpatch}
\usepackage[final]{microtype}

%-----FONT------%

\usepackage{fontspec}
\usepackage{amssymb}
\usepackage{amsmath} 
%\usepackage{kpfonts-otf}
%\usepackage{fourier-otf}

%-----COULEURS-----%

\usepackage{color}
\usepackage[svgnames, dvipsnames]{xcolor} 

\definecolor{myred}{rgb}{0.82, 0.1, 0.26}
\definecolor{myblue}{rgb}{0.27, 0.42, 0.81}
\definecolor{mynavyblue}{rgb}{0.0, 0.0, 0.5}
\definecolor{mygreen}{rgb}{0.0, 0.26, 0.15}
\definecolor{myyellow}{rgb}{1.0, 0.65, 0.0}
\definecolor{mypink}{rgb}{0.65, 0.04, 0.37}
\definecolor{myorange}{rgb}{0.91, 0.45, 0.32}
\definecolor{myjade}{rgb}{0.0, 0.66, 0.42}
\definecolor{mytea}{rgb}{0.96, 0.76, 0.76}
\definecolor{mybrown}{rgb}{0.44, 0.26, 0.08}
\definecolor{mypurple}{rgb}{0.2, 0.07, 0.48}
\definecolor{myolive}{rgb}{0.33, 0.42, 0.18}

\definecolor{codegreen}{rgb}{0,0.6,0}
\definecolor{codepurple}{rgb}{0.58,0,0.82}
\definecolor{backcolour}{RGB}{0.98, 0.98, 0.98}

\definecolor{nfpgreen}{RGB}{52, 176, 109}
\definecolor{nfpred}{RGB}{229, 0, 42}
\definecolor{nfpyellow}{RGB}{255, 238, 0}
\definecolor{nfppurple}{RGB}{150, 19, 130}
\definecolor{nfppink}{RGB}{230, 0, 82}

\definecolor{titlecolor}{rgb}{0,0,0}

%-------------------%

%-----TIKZ-SETTINGS-----%

\usepackage{tikz}
\usepackage{tkz-tab}
\usetikzlibrary{patterns}
\usetikzlibrary{decorations.fractals}
\usetikzlibrary{decorations.text}  
\usetikzlibrary{angles,quotes}
\usetikzlibrary{shapes.multipart, arrows.meta, calc, tikzmark, positioning}
\usetikzlibrary{shapes.geometric}

%-COMMANDS-%

\newcommand*{\annotate}[4]{
    \begin{tikzpicture}[remember picture, overlay]
        \draw[->, #4] ($(#1.south)+(0,-#3em)$) node[#4, yshift=-1.5mm]{\tiny #2} -- (#1.south);
    \end{tikzpicture}
}
\newcommand*{\lilbox}[2]{
    \tikz[baseline=(char.base)]{
    \node[shape=rectangle, rounded corners, draw, #1, fill= #1!2, inner sep=4pt] (char) {#2}
    } 
}

%-----FANCYHDR-SETTINGS-----%

\usepackage{fancyhdr}
\pagestyle{fancy}
\usepackage{authoraftertitle}

\renewcommand{\headrule}{\color{titlecolor}{\hrule width\headwidth height\headrulewidth \vskip-\headrulewidth}}
\renewcommand{\chaptermark}[1]{\markboth{#1}{}}

\fancyhead[C]{\color{black!70}  \textsc{\bfseries \thechapter \, \leftmark \, /  \, \nouppercase{\rightmark}}}
\fancyhead[LE, RO]{\href{https://creativecommons.org/licenses/by-nc/4.0/}{\ccbync}}
\fancyhead[RE, LO]{
    \tikz[baseline=(char.base)]{
    \node[shape=rectangle, fill=titlecolor, text=white, inner sep=4pt] (char) {\thepage}
    } 
    \hspace{-5.5pt}
    }

\fancyfoot[C]{}
\fancyfoot[LE]{\color{black!70} \textsc{\bfseries PC}}
\fancyfoot[RO]{\color{black!70} \textsc{\bfseries Lycée Hoche, \MyDate}}
\fancyfoot[RE, LO]{\color{black!70} \textsc{\bfseries Olivier Allard}}

\fancypagestyle{plain}{
    \fancyhead[C]{}
}

%-------------------%

%-----TITLES-&-TOC-----%

\usepackage{titletoc}
\usepackage[newparttoc]{titlesec}

\setcounter{secnumdepth}{3}
\setcounter{tocdepth}{2}

\renewcommand{\thesection}{\Roman{section}}
\renewcommand{\thesubsection}{\Alph{subsection}}
\renewcommand{\thesubsubsection}{\arabic{subsubsection}} 

%--TOC--%

\titlecontents{part}[0pt]
    {\addvspace{15pt}\Large\bfseries}
    {\tikz[baseline=(char.base)]{
        \node[shape=rectangle, rounded corners, fill=black, text=white, inner sep=6pt] (char) {Partie \thecontentslabel}} \quad}
    {}
    {\hfill Page~\contentspage}
    [\addvspace{10pt}]

\titlecontents{chapter}[3pt]
    {\addvspace{10pt}\large\bfseries}
    {\tikz[baseline=(char.base)]{
        \node[shape=rectangle, rounded corners, fill=black, text=white, inner sep=4pt] (char) {Chapitre \thecontentslabel}} \quad}
    {}
    {\hfill Page~\contentspage}
    [\addvspace{5pt}]

%--TITLES--%

\titleformat{\part}[display]
    {\Huge\bfseries\filcenter}
    {\tikz[baseline=(char.base)]{
        \node[shape=rectangle, rounded corners, fill=black, text=white, inner sep=8pt] (char) {Partie \thepart}} 
        }
    {0pt}
    {}
    {}

\titleformat{\chapter}[display]
    {\huge\bfseries\filcenter}
    {\tikz[baseline=(char.base)]{
        \node[shape=rectangle, rounded corners, fill=black, text=white, inner sep=6pt] (char) {Chapitre \thechapter}}}
    {0pt} 
    {}
    {}
\titlespacing*{\chapter}{0mm}{6mm}{18mm}

\titleformat{\section}
  {\Large\bfseries\scshape\color{titlecolor}} 
  {\tikz[baseline=(char.base)]{
    \node[text=white, inner sep=.5em] (char) {\thesection};
    \begin{scope}
        \shade[rounded corners=.5em, left color=titlecolor!60, right color= titlecolor, overlay, remember picture] ($(char.east)+(-4em,-.8em)$) rectangle ++(4.1em,1.6em);
    \end{scope}
    \node[text=white, inner sep=.5em] (char) {\thesection};}}
  {.5em}
  {}
  {}
\titlespacing*{\section}{0mm}{8mm}{8mm}

\titleformat{\subsection}
  {\large\bfseries\scshape\color{titlecolor}} 
  {\tikz[baseline=(char.base)]{
    \node[text=white, inner sep=.5em] (char) {\thesubsection};
    \begin{scope}
        \shade[rounded corners=.5em, left color=titlecolor!60, right color= titlecolor, overlay, remember picture] ($(char.east)+(-3.5em,-.8em)$) rectangle ++(3.6em,1.6em);
    \end{scope}
    \node[text=white, inner sep=.5em] (char) {\thesubsection};}}
  {.5em}
  {}
  {}
\titlespacing*{\subsection}{0mm}{6mm}{6mm}

\titleformat{\subsubsection}
  {\normalsize\bfseries\scshape\color{titlecolor}} 
  {\tikz[baseline=(char.base)]{
    \node[text=white, inner sep=.5em] (char) {\thesubsubsection};
    \begin{scope}
        \shade[rounded corners=.5em, left color=titlecolor!60, right color= titlecolor, overlay, remember picture] ($(char.east)+(-3em,-.8em)$) rectangle ++(3.1em,1.6em);
    \end{scope}
    \node[text=white, inner sep=.5em] (char) {\thesubsubsection};}}
  {.5em}
  {}
  {}
\titlespacing*{\subsubsection}{0mm}{4mm}{4mm}

%------ENUMITEM-SETTINGS-----%

\usepackage{enumitem}
\setenumerate{topsep=1pt, itemsep=1pt, parsep=0pt, partopsep=0pt} 
\setitemize{topsep=1pt, itemsep=1pt, parsep=0pt, partopsep=0pt} 

\setlist[enumerate,1]{label=\textbf{(\roman*)}}
\setlist[enumerate,2]{label=\textit{(\alph*)}}
\setlist[enumerate,3]{label=(\arabic*)}

\setlist[itemize,1]{label=$\star$}
\setlist[itemize,2]{label=$\to$}
\setlist[itemize,3]{label=$\circ$}

\setlength{\columnseprule}{0pt}
\setlength{\columnsep}{0.5cm}
\setlength{\parindent}{0pt}
\setlength{\parskip}{5pt plus 0.5pt minus 0.5pt}

%---------------------%

%-----TCB-----%

\usepackage[most]{tcolorbox}

%--SET--%

\tcbset{
    enhanced, breakable,
    left=1.5mm, right=1.5mm, top=1.5mm, bottom=1.5mm, arc=1.5mm,
    leftrule=.2mm, rightrule=.2mm, toprule=.2mm, bottomrule=.2mm, titlerule=.2mm,
    fonttitle=\bfseries,
    theorem/.style={
        theorem style = break,
        drop small lifted shadow,
        oversize,
        theorem hanging indent=1cm,
        lower separated=false,
        description color=black, description font=\bfseries, description delimiters parenthesis,
        terminator sign none,
        separator sign={\quad}
    }
}

%--RD--%

\newcommand{\chaptertoc}{
    \begin{tcolorbox}[enhanced, oversize, top=3mm, title=Plan du chapitre,
        colframe=titlecolor,colback=titlecolor!2,colbacktitle=titlecolor!2,
        fonttitle=\bfseries, coltitle=titlecolor, attach boxed title to top center=
        {yshift=-0.25mm-\tcboxedtitleheight/2,yshifttext=2mm-\tcboxedtitleheight/2},
        boxed title style={boxrule=0.5mm,
        frame code={ \path[tcb fill frame] ([xshift=-4mm]frame.west)
        -- (frame.north west) -- (frame.north east) -- ([xshift=4mm]frame.east)
        -- (frame.south east) -- (frame.south west) -- cycle; },
        interior code={ \path[tcb fill interior] ([xshift=-2mm]interior.west)
        -- (interior.north west) -- (interior.north east)
        -- ([xshift=2mm]interior.east) -- (interior.south east) -- (interior.south west)
        -- cycle;} }, drop small lifted shadow]
        \startcontents[chapters]
        \printcontents[chapters]{}{1}{}{}
    \end{tcolorbox}
}

%--THEOREMS--%

\newcounter{theoremcounter}
\newcounter{propositioncounter}
\newcounter{corollarycounter}
\newcounter{lemmacounter}
\newcounter{exercicecounter}
\newcounter{definitioncounter}
\newcounter{definitiontheoremcounter}

\newtcbtheorem[use counter=theoremcounter]{theo}{Théorème}{%
    drop small lifted shadow,
    oversize,
    attach boxed title to top left = {xshift = 1cm, yshift=-\tcboxedtitleheight/2, yshifttext=-5mm},
    boxed title style={
        frame code={ 
            \draw[myred, line width = .2mm, tcb fill interior] (frame.west) -- ($(frame.south west)+(2mm,0)$) -- ($(frame.south east)+(-2mm,0)$) -- (frame.east) -- ($(frame.north east)+(-2mm,0)$) -- ($(frame.north west)+(2mm,0)$) -- cycle;
            \draw[myred, fill=myred!2, line width=.2mm] (frame.west) circle (1mm) ;
            \draw[myred, fill=myred!2, line width=.2mm] (frame.east) circle (1mm) ;
        },
        interior code={}},
    description color=black, description font=\bfseries, description delimiters parenthesis, terminator sign none,
    separator sign={\quad},
    colframe=myred, colback=myred!2, coltitle=myred, colbacktitle=myred!2,
    }{theo}
\newtcbtheorem[use counter=propositioncounter]{prop}{\hspace*{5mm}Proposition}{%
    theorem,
    colframe=myolive, colback=myolive!2, coltitle=myolive, colbacktitle = myolive!2, 
    overlay unbroken and first={\draw[myolive, fill=myolive!2, line width=.2mm] ($(frame.north west)+(.12mm,-4.5mm)$) circle (1mm);}
    }{prop}
\NewTcbTheorem[use counter=corollarycounter]{coro}{\hspace*{5mm}Corollaire}{%
    theorem,
    colframe=myorange, colback=myorange!2, coltitle=myorange, colbacktitle = myorange!2, 
    overlay unbroken and first={\draw[myorange, fill=myorange!2, line width=.2mm] ($(frame.north west)+(.12mm,-4.5mm)$) circle (1mm);}
    }{coro}
\NewTcbTheorem[use counter= lemmacounter]{lem}{\hspace*{5mm}Lemme}{%
    theorem,
    colframe=mybrown, colback=mybrown!2, coltitle=mybrown, colbacktitle = mybrown!2,
    overlay unbroken and first={\draw[mybrown, fill=mybrown!2, line width=.2mm] ($(frame.north west)+(.12mm,-4.5mm)$) circle (1mm);} 
    }{lem}
\newtcbtheorem[use counter=exercicecounter]{exo}{\hspace*{5mm}Exercice}{%
    theorem,
    colframe=nfpgreen, colback=nfpgreen!2, coltitle=nfpgreen, colbacktitle = nfpgreen!2,
    overlay unbroken and first={\draw[nfpgreen, fill=nfpgreen!2, line width=.2mm] ($(frame.north west)+(.12mm,-4.5mm)$) circle (1mm);} 
    }{exo}
\newtcbtheorem[use counter= definitioncounter]{defi}{\hspace*{5mm}Définition}{%
    theorem,
    colframe=myyellow, colback=myyellow!2, coltitle=myyellow,  colbacktitle=myyellow!2, 
    overlay unbroken and first={\draw[myyellow, fill=myyellow!2, line width=.2mm] ($(frame.north west)+(.12mm,-4.5mm)$) circle (1mm);}
    }{defi}
\newtcbtheorem[use counter=definitiontheoremcounter]{defitheo}{\hspace*{5mm}Définition-Théorème}{%
    theorem,
    colframe=mypurple, colback=mypurple!2, coltitle=mypurple, colbacktitle = mypurple!2,
    overlay unbroken and first={\draw[mypurple, fill=mypurple!2, line width=.2mm] ($(frame.north west)+(.12mm,-4.5mm)$) circle (1mm);} 
    }{defitheo}

\NewDocumentEnvironment{omed}{mm}{
    \begin{tcolorbox}[
        oversize,
        frame hidden,
        detach title, before upper={\tcbtitle\quad},
        title=#1,
        colframe=white, colback=white, coltitle=#2, colbacktitle = white,
    ]
}{
    \end{tcolorbox}
}

\NewDocumentEnvironment{demo}{mm}{
    \begin{tcolorbox}[
        left skip = .5cm, 
        frame hidden,
        detach title, before upper={\tcbtitle\quad},
        title=#1,
        colframe=white, colback=white, coltitle=#2, colbacktitle = white,
    ]
}{
    \null\hfill{\textcolor{#2}{\ding{113}}}
    \end{tcolorbox}
}

%-------------------%

%---TBLR SETTINGS---%

\usepackage{tabularray}

%%-----PACKAGES-----%%

\usepackage{polyglossia}
\setdefaultlanguage[frenchpart=false]{french}
\usepackage{setspace}   
\everymath{\displaystyle}
\allowdisplaybreaks
\usepackage{graphicx}
\usepackage{rotating}
\usepackage{wrapfig}
\usepackage{tabularx}
\usepackage{longtable}
\usepackage{multicol}
\usepackage{lipsum}  
\usepackage{fancybox}
\usepackage{fancyvrb}
\usepackage{framed}
\usepackage{array}
\setcounter{MaxMatrixCols}{20}
\usepackage{centernot}
\usepackage{pstricks}
\usepackage[european, straightvoltages, RPvoltages, cute inductor]{circuitikz}
\usepackage{stmaryrd}
\usepackage{siunitx}
\sisetup{
     detect-all,
     output-decimal-marker={,},
     group-minimum-digits = 3,
     group-separator={~},
     number-unit-separator={~},
     inter-unit-product={~},
     list-separator = {, },
     list-final-separator = { et },
     range-phrase = --,
     separate-uncertainty = true,
     multi-part-units = single,
     list-units = single,
     range-units = single
}
\usepackage{arydshln}
\setlength{\dashlinedash}{0.2pt}
\setlength{\dashlinegap}{1.5pt}
\usepackage{pdfpages}
\usepackage{verbatim}
\usepackage{float}
\usepackage[normalem]{ulem}
\usepackage{soul}
\usepackage{pgfplots}
\pgfplotsset{compat=1.18}
\usepackage{ifthen}
\usepackage{listings}
\lstdefinestyle{mystyle}{
    numberstyle=\tiny,
    basicstyle=\sffamily\footnotesize,
    breakatwhitespace=true,         
    breaklines=true,                 
    captionpos=t,                  
    keepspaces=true,                                 
    showspaces=false,                
    showstringspaces=false,
    showtabs=false,  
    numbers=left,                    
    numbersep=2pt         
}
\lstset{style=mystyle}
\usepackage{xparse}
\usepackage{pdflscape}
\usepackage{JeuxCartes}
\usepackage{makecell}
\usepackage{booktabs}
\usepackage{ccicons}
\usepackage{upgreek}

%-----HYPERREF-SETTINS-----%

\usepackage{hyperref} 
\hypersetup{%
    colorlinks=true,
    linkcolor=black,
    urlcolor=mynavyblue,
    filecolor=mypurple,
    pdfauthor=Olivier Allard
    }

%-----ENVIRONNEMENTS-----%

\newenvironment{soient}{%
    Soient
    \begin{itemize}[label=$\circ$]
    }
    {
    \end{itemize}
    }
\newenvironment{soit}{%
    Soit
    \begin{itemize}[label=$\circ$]
        \begin{multicols}{2}
    }
    {           
        \end{multicols}
    \end{itemize}
    }
\newenvironment{suppose}{%
    On suppose que
    \begin{enumerate}[label=\textit{(\alph*)}]
    }
    {
    \end{enumerate}
    }
\newenvironment{alors}{%
    Alors
    \begin{enumerate}
    }
    {
    \end{enumerate}
    }

%-------------------%

%%-----COMMANDS-----%%

\newcommand*\f[1]{f \in \mathcal{F}(\mathcal{D},\mathbb{#1})}
\newcommand*\g[1]{g \in \mathcal{F}(\mathcal{D'},\mathbb{#1})}
\newcommand*\un{(u_n)_n \in \mathbb{K}^{\mathbb{N}}}
\newcommand*\mk[1]{\mathcal{M}_{#1}(\mathbb{K})}
\newcommand*\mat[2]{\text{Mat}_{#1}\left(#2\right)}
\newcommand*\pass[2]{P_{#1 \rightarrow #2}}
\newcommand*\esp[1]{\quad \text{#1} \quad}
\newcommand*\eqlabel[1]{\quad (\text{#1})}
\newcommand*\modulo[1]{\phantom{o} \left[#1\right]}
\newcommand*\eqmodulo[3]{#1 \equiv #2 \phantom{o} \left[#3\right]}
\newcommand*\argu{\quad \downarrow \quad}

\newcommand*{\cita}[2]{\textit{« #1 »} (\textbf{\textsc{p}}. \oldstylenums{text})}

%--BRACES--%

\newcommand*\application[4]{%
    \left\{ \begin{array}{ccl} 
      #1 & \longrightarrow & #2 \\
      #3 & \longmapsto & #4 
    \end{array} \right. \kern-\nulldelimiterspace
    }
\newcommand*\fonction[5]{%
    #1 : \left\{ \begin{array}{ccl} 
      #2 & \longrightarrow & #3 \\
      #4 & \longmapsto & #5 
    \end{array} \right. \kern-\nulldelimiterspace
    }
\newcommand*\sisi[4]{%
    \left\{ \begin{array}{cl}
        #1 & \text{si } #2 \\
        #3 & \text{si } #4 
    \end{array} \right. \kern-\nulldelimiterspace
}   
\newcommand*\sisinon[3]{%
    \left\{ \begin{array}{cl}
        #1 & \text{si } #2 \\
        #3 & \text{sinon}  
    \end{array} \right. \kern-\nulldelimiterspace
}   
\newcommand*\et[2]{
    \left\{ \begin{array}{l}
        #1  \\
        #2  
    \end{array} \right. \kern-\nulldelimiterspace
}
\newcommand*\ou[2]{
    \left| \begin{array}{l}
        #1  \\
        #2  
    \end{array} \right. \kern-\nulldelimiterspace
}
\newcommand*\textou[2]{
    \left| \begin{array}{l}
        \text{#1}  \\
        \text{#2}  
    \end{array} \right. \kern-\nulldelimiterspace
}
%--DELIMITERS--%

\newcommand*\vbar{\, \big| \,}
\newcommand*\intervalleOO[2]{\left] #1 \, , #2 \right[}
\newcommand*\intervalleFF[2]{\left[ #1 \, , #2 \right]}
\newcommand*\intervalleOF[2]{\left] #1 \, , #2 \right]}
\newcommand*\intervalleFO[2]{\left[ #1 \, , #2 \right[}
\newcommand*\intervalleEntier[2]{\left\llbracket #1 \, , #2 \right\rrbracket}
\newcommand*\ent[1]{\left\lfloor #1 \right\rfloor}
\newcommand*\fract[1]{\left\lceil #1 \right\rceil}
\newcommand*\enstq[2]{\left\{ #1 \, , \, #2 \right\}}
\newcommand*\enspr[2]{\left\{#1 \, \middle| \, #2\right\}}
\newcommand*\independent{\perp\mkern-9.5mu\perp}
\newcommand*\notindependent{\centernot{\independent}}
\newcommand*\limi[2]{\underset{#1 \rightarrow #2}{\longrightarrow}}
\newcommand*\limit[3]{\underset{#2 \rightarrow #3}{#1}}
\newcommand*\comp[4]{#1_{#2 \rightarrow #3} \left(#4\right)}

%--MATHOPERATORS--%

\newcommand*\spr[2]{\left<\,#1\,\middle|\,#2\,\right>}
\newcommand*\abs[1]{\left\lvert \,#1 \, \right\rvert}
\newcommand*\norm[1]{\left\lVert \,#1 \, \right\rVert}
\newcommand*\nnorm[2]{\left\lVert\,#2\,\right\rVert_{#1}}
\newcommand*\vct[1]{\overrightarrow{#1}}
\newcommand*\ovl[1]{\overline{#1}}
\newcommand*\dpart[2]{\frac{\partial #1}{\partial #2}}

\newcommand{\E}{\mathbb{E}}
\newcommand{\V}{\mathbb{V}}
\let\P\relax
\newcommand{\P}{\mathbb{P}}

\DeclareMathOperator{\card}{card}
\DeclareMathOperator{\im}{im}
\DeclareMathOperator{\id}{id}
\DeclareMathOperator{\vect}{vect}
\DeclareMathOperator{\rg}{rg}
\DeclareMathOperator{\tr}{tr}
\DeclareMathOperator{\cov}{cov}
\let\d\relax
\DeclareMathOperator{\d}{d}
\DeclareMathOperator{\Div}{Div}
\DeclareMathOperator{\diag}{diag}
\let\sp\relax
\DeclareMathOperator{\sp}{sp}
\DeclareMathOperator{\fr}{fr}
\DeclareMathOperator{\VA}{VA}
\DeclareMathOperator{\CM}{CM}
\DeclareMathOperator{\GL}{GL}
\DeclareMathOperator{\SL}{SL}
\DeclareMathOperator{\SO}{SO}
\let\O\relax
\DeclareMathOperator{\O}{O}
\DeclareMathOperator{\orb}{orb}
\DeclareMathOperator{\stab}{stab}
\DeclareMathOperator{\grad}{grad}
\DeclareMathOperator{\cste}{cste}
\DeclareMathOperator{\rot}{rot}
\DeclareMathOperator{\Aut}{Aut}
\DeclareMathOperator{\Int}{Int}
\DeclareMathOperator{\pgcd}{pgcd}
\DeclareMathOperator{\ppcm}{ppcm}
\DeclareMathOperator{\com}{com}
\let\S\relax
\DeclareMathOperator{\S}{S}
\DeclareMathOperator{\argtanh}{argtanh}
\DeclareMathOperator{\argcosh}{argcosh}
\let\L\relax
\DeclareMathOperator{\L}{L}

%--RESTRICTIONS--%

\newcommand\restr[2]{{
  \left.\kern-\nulldelimiterspace
  #1
  \vphantom{\big|}
  \right|_{#2}
  }}
\newcommand\corestr[2]{{
    \left.\kern-\nulldelimiterspace
    #1
    \vphantom{\big|}
    \right|^{#2}
    }}
\newcommand\restrcorestr[3]{{
    \left.\kern-\nulldelimiterspace
    #1
    \vphantom{\big|}
    \right|_{#2}^{#3}
    }}

%--------------------%

\usepackage{libertinus-otf}


