\customchapter{Techniques de calcul}{Ce chapitre est un répertoire des techniques de calcul couvrant différents domaines des mathématiques. Certaines méthodes, fondamentales, faciliteront l’utilisation des outils mathématiques.}

\section{Trigonométrie}

\subsection{Cercle trigonométrique}

    \begin{theo}{Conséquences de la construction du cercle trigonométrique}{}
        Soient $\alpha,\beta \in \mathbb{R}$ et $z \in \mathbb{C}$.

        \begin{alors}
            \item $(\cos \alpha)^2 + (\sin \alpha)^2 = 1$
            \item $|z| = 1 \iff \exists \alpha \in \mathbb{R}, z = \cos (\alpha) + i \sin (\alpha)$
            \item $\cos \alpha + i \sin \alpha = \cos \beta + i \sin \beta \iff \alpha \equiv \beta \mod 2\pi$
        \end{alors}
    \end{theo}

    \begin{longtblr}[
        caption={Les formules qui se lisent sur le cercle}
        ]{
            colspec={| X[4,c] X[1,c] X[4,c] || X[4,c] X[1,c] X[4,c]|}, width = \linewidth, 
            hlines={0.4pt, black},
            row{odd} = {myolive!30},
        }
        $\cos(- \alpha)$ & = & $\cos(\alpha)$ & $\sin(- \alpha)$  & = & $-\sin(\alpha)$ \\
        $\cos(\pi + \alpha)$ & = & $-\cos(\alpha)$ & $\sin(\pi + \alpha)$  & = & $-\sin(\alpha)$ \\
        $\cos(\pi - \alpha)$ & = & $-\cos(\alpha)$ & $\sin(\pi - \alpha)$  & = &  $\sin(\alpha)$ \\
        $\cos\left( \frac{\pi}{2} + \alpha \right)$ & = & $- \sin (\alpha)$ &  $\sin \left( \frac{\pi}{2} + \alpha \right)$ & = & $\cos(\alpha)$ \\
        $\cos\left( \frac{\pi}{2} - \alpha \right)$ & = & $\sin(\alpha)$ & $\sin \left( \frac{\pi}{2} - \alpha \right)$ & = & $\cos(\alpha)$ 
    \end{longtblr}

\subsection{Formules de trigonométrie}

    \begin{prop}{Formules d’addition / soustraction}{}
        Soient $a,b \in \mathbb{R}$.
        \begin{align}
            \cos(a + b) = \cos(a)\cos(b) - \sin(a)\sin(b) \\
            \sin(a + b) = \sin(a)\cos(b) + \sin(b)\cos(a) \\
            \cos(a - b) = \cos(a)\cos(b) + \sin(a)\sin(b) \\
            \sin(a - b) = \sin(a)\cos(b) - \sin(b)\cos(a) 
        \end{align}
    \end{prop}

    \begin{demo}{Démonstration}{myolive}
        Identifier les parties réelle et imaginaire de 
        \[ e^{i(a \pm b)} = e^{ia}e^{\pm ib} \]
    \end{demo}

    \begin{coro}{Formules de duplication}{}
        Soit $a \in \mathbb{R}$.
        \begin{align}
            \cos(2a) = \cos^2(a) - \sin^2(a) \\
            \sin(2a) = 2\sin(a)\cos(a)
        \end{align}
    \end{coro}

    \begin{demo}{Raisonnement}{myorange}
        Poser $b = a$ dans $(1.1)$ et $(1.2)$.
    \end{demo}

    \begin{prop}{Formule de déphasage}{}
        Soient $a,b,x \in \mathbb{R}$.
        \[ a\cos(x) + b\sin(x) = |a + ib|\cos\left(x - \arg(a + ib) \right) \]        
    \end{prop}

    \begin{demo}{Preuve}{myolive}
        Considérer
        \[ \Re \left(\overline{(a + ib)}e^{ix} \right) = \Re \left(|a +ib|e^{i \left(x-\arg(a+ib) \right)} \right) \]
    \end{demo}

    \begin{theo}{Formules d’Euler}{}
        Soit $a \in \mathbb{R}$.
        \[ \cos(a) = \frac{e^{ia}+e^{-ia}}{2} \qquad \sin(a) = \frac{e^{ia}-e^{-ia}}{2i}\]
    \end{theo}

    \begin{coro}{Formules de multiplication}{}
        Soient $a,b \in \mathbb{R}$.
        \begin{align}
            \cos(a)\cos(b) = \frac{1}{2} \left[ \cos(a+b) + \cos(a-b) \right] \\
            \sin(a)\sin(b) = \frac{1}{2} \left[ \cos(a-b) - \cos(a+b) \right] \\
            \sin(a)\cos(b) = \frac{1}{2} \left[ \sin(a+b) + \sin(a-b) \right]
        \end{align}
    \end{coro}

    \begin{demo}{Idée}{myorange}
        \begin{itemize}[label=\textcolor{myorange}{$\star$}]
            \item Exprimer les produits avec la formule d’Euler et développer puis rassembler.
            \item Sinon, additionner les formules d’addition soustraction.
        \end{itemize}
    \end{demo}

    \begin{coro}{}{}
        Soit $a \in \mathbb{R}$.
        \begin{align}
            \cos^2(a) = \frac{1 + \cos(2a)}{2} \\
            \sin^2(a) = \frac{1 - \cos(2a)}{2} \\
            \cos(a)\sin(a) = \frac{\sin(2a)}{2}
        \end{align}
    \end{coro}
    
    \begin{demo}{Raisonnement}{myorange}
        \begin{itemize}[label=\textcolor{myorange}{$\star$}]
            \item Prendre $b = a$ dans (1.7), (1.8) et (1.9).
            \item Sinon, s’aider des formules de duplication, en utilisant $\cos^2(a) + \sin^2(a) = 1$.
        \end{itemize}
    \end{demo}

    \begin{prop}{}{}
        Soient $p,q \in \mathbb{R}$.
        \begin{align}
            \sin(p) + \sin(q) = 2\sin \left( \frac{p + q}{2} \right) \cos \left( \frac{p - q}{2} \right) \\
            \sin(p) - \sin(q) = 2\cos \left( \frac{p + q}{2} \right) \sin \left( \frac{p - q}{2} \right) \\
            \cos(p) + \cos(q) = 2\cos \left( \frac{p + q}{2} \right) \cos \left( \frac{p - q}{2} \right) \\
            \cos(p) - \cos(q) = -2\sin \left( \frac{p + q}{2} \right) \sin \left( \frac{p - q}{2} \right)
        \end{align}
    \end{prop}

    \begin{demo}{Démonstration}{myolive}
        \begin{itemize}[label=\textcolor{myolive}{$\star$}]
        \item Utiliser l’angle moitié.
            \begin{align*}
            \Im(e^{ip} + e^{iq}) & = \sin(p) + \sin(q) \\
            & = \Im \left(e^{i\frac{p+q}{2}} \left( e^{i (p - \frac{p+q}{2})} + e^{i(q - \frac{p+q}{2})} \right)\right) \\
            & = \Im \left(e^{i\frac{p+q}{2}} \left( e^{i (\frac{p-q}{2})} + e^{-i(\frac{p-q}{2})} \right)\right) \\
            & = 2 \sin\left(\frac{p+q}{2}\right) \cos\left(\frac{p-q}{2}\right) \\
            \Re(e^{ip} + e^{iq}) & = \cos(p) + \cos(q) \\
            & = \Re \left(e^{i\frac{p+q}{2}} \left( e^{i (p - \frac{p+q}{2})} + e^{i(q - \frac{p+q}{2})} \right)\right) \\
            & = \Re \left(e^{i\frac{p+q}{2}} \left( e^{i (\frac{p-q}{2})} + e^{-i(\frac{p-q}{2})} \right)\right) \\
            & = 2 \cos\left(\frac{p+q}{2}\right) \cos\left(\frac{p-q}{2}\right)
            \end{align*}
        \item Sinon, remplacer $p$ et $q$ par $\left( \frac{a + b}{2} \right)$ et $\left( \frac{a - b}{2} \right)$ dans $(1.7)$, $(1.8)$, $(1.9)$.
        \end{itemize}
    \end{demo}

    \begin{prop}{}{}
        Soit $\alpha \in \mathbb{R} \backslash \left(\frac{\pi}{2} + \pi \mathbb{Z}\right)$.
        \[ 1 + \tan^2(\alpha) = \frac{1}{\cos^2(\alpha)} \] 
    \end{prop}

    \begin{demo}{Preuve}{myolive}
        \[ 1 + tan^2(\alpha) = \frac{cos^2(\alpha) + \sin^2(\alpha)}{\cos^2(\alpha)} = \frac{1}{\cos^2(\alpha)} \]
    \end{demo}

    \begin{prop}{}{}
        Soient $a,b \in \mathbb{R} \backslash \left(\frac{\pi}{2} + \pi \mathbb{Z}\right)$.
        \begin{align}
            \text{si } \cos(a+b) \neq 0, \quad  \tan(a + b) = \frac{\tan(a) + \tan(b)}{1 - \tan(a)tan(b)} \\
            \text{si } \cos(a-b) \neq 0, \quad \tan(a - b) = \frac{\tan(a) - \tan(b)}{1 + \tan(a)\tan(b)}
        \end{align}
    \end{prop}

    \begin{demo}{Idée de preuve}{myolive}
        Utiliser les formules d’addition puis diviser par $\cos(a)\cos(b)$ en haut et en bas.
    \end{demo}

    \begin{prop}{Formules de l’arc moitié}{}
        Soient $\alpha \in \mathbb{R} \backslash \left( \pi \mathbb{Z} \right)$ et $ t = \tan \frac{\alpha}{2}$.
            \begin{align}
                \cos(\alpha)=\frac{1-t^2}{1 + t^2} \\
                \sin(\alpha) = \frac{2t}{1 + t^2} \\
                \text{si } \alpha \not\equiv \frac{\pi}{2} \, [\pi], \quad \tan(\alpha) = \frac{2t}{1 - t^2}
            \end{align}
    \end{prop}

    \begin{demo}{Preuve}{myolive}
        \begin{itemize}[label=\textcolor{myolive}{$\star$}]
            \item Pour sin et cos, utiliser les formules d’addition et remarquer que le dénominateur s’écrit $\cos^2(\alpha) + \sin^2(\alpha)$, d’où le résultat en divisant en haut et en bas par $\cos^2 \left(\frac{\alpha}{2} \right)$.
            \item Pour tan, utiliser les formules précédentes, ou prendre $b = a$ dans $(1.17)$.
        \end{itemize}
    \end{demo}

\subsection{Annexe de trigonométrie}

    \begin{longtblr}[
        caption={Annexe de Trigonométrie}
        ]{
            colspec={|X[3,c]||X[2,l] |}, width = \linewidth,
            rowhead = 1, 
            hlines={0.4pt, black},
            row{odd} = {myolive!30}, row{1} = {myolive, fg=white, font=\bfseries}
        }
        Formules & Idée de preuve \\
            $\begin{aligned}[t]
                & \sin^2(x) + \cos^2(x) = 1 \\
                & \abs{e^{ix}} = 1
            \end{aligned}$
            &
            Voir le cercle trigonométrique
            \\
            $\begin{aligned}[t]
                \cos(x) &= \frac{e^{ix} + e^{-ix}}{2} \\
                \sin(x) &= \frac{e^{ix} - e^{-ix}}{2}
            \end{aligned}$
            & 
            $\Re(z) = (z + \bar{z})/2$ appliqué à $e^{ix}$, de même pour $\Im$. 
            \\
            $\begin{aligned}[t]
                \cos(a+b) &= \cos(a)\cos(b) - \sin(a)\sin(b) \\
                \sin(a+b) &= \sin(a)\cos(b) + \cos(a)\sin(b) \\
                \cos(a-b) &= \cos(a)\cos(b) + \sin(a)\sin(b) \\
                \sin(a-b) &= \sin(a)\cos(b) - \cos(a)\sin(b)
            \end{aligned}$
            &
            Identifier les parties réelle et imaginaire de $e^{i(a+b)} = e^{ia}e^{ib}$ pour les deux premières, puis remplacer $b$ par $-b$ pour les deux suivantes. 
            \\
            $\begin{aligned}[t]
                \cos(2a) &= 2\cos^2(a) - 1 = 1 - 2\sin^2(a) \\
                \sin(2a) &= 2\cos(a)\sin(a)
            \end{aligned}$
            & 
            Poser $b = a$ dans les formules ci-dessus. 
            \\
            $\begin{aligned}[t]
                a\cos(x) + b\sin(x) = |a+ib|\cos(x-\arg(a+ib))
            \end{aligned}$
            & 
            Considérer l’égalité entre $\Re\left((\overline{a+ib})e^{ix}\right)$ et $\Re\left(|a+ib|e^{i(x-arg(a+ib))}\right)$. 
            \\
            $\begin{aligned}[t]
                2\cos(a)\cos(b) &= \cos(a+b) + \cos(a-b) \\
                2\sin(a)\sin(b) &= \cos(a-b) - \cos(a+b) \\
                2\sin(a)\cos(b) &= \sin(a+b) + \sin(a-b)
            \end{aligned}$
            &
            Exprimer les produits avec la formule d’Euler et développer puis rassembler. Sinon, additionner les formules d’addition et soustraction. 
            \\
            $\begin{aligned}[t]
                2\cos^2(a) &= 1 + \cos(2a) \\
                2\sin^2(a) &= 1 - \cos(2a)
            \end{aligned}$
            & 
            Poser $b = a$ dans les formules ci-dessus. 
            \\
            $\begin{aligned}[t]
                \cos(a) + \cos (b) &= 2\cos\left(\frac{a+b}{2}\right)\cos\left(\frac{a-b}{2}\right) \\
                \cos(a) - \cos (b) &= -2\sin\left(\frac{a+b}{2}\right)\sin\left(\frac{a-b}{2}\right) \\
                \sin(a) + \sin (b) &= 2\sin\left(\frac{a+b}{2}\right)\cos\left(\frac{a-b}{2}\right) \\
                \sin(a) - \sin (b) &= 2\cos\left(\frac{a+b}{2}\right)\sin\left(\frac{a-b}{2}\right)
            \end{aligned}$
            & 
            Utiliser l’angle moitié, ou remplaçer $a$ et $b$ par $(a+b)/2$ et $(a-b)/2$ dans les formules de multiplication. 
            \\
            $\begin{aligned}[t]
                \tan(a+b) &= \frac{\tan(a)+\tan(b)}{1 -\tan(a)\tan(b)} \\
                \tan(a-b) &= \frac{\tan(a)-\tan(b)}{1 + \tan(a)\tan(b)}
            \end{aligned}$
            &  
            Utiliser les formules d’addition puis diviser en haut et en bas par $\cos(a)\cos(b)$. 
            \\
            $\begin{aligned}[t]
                \cos(a) &= \frac{1 - t^2}{1 + t^2} \\
                \sin(a) &= \frac{2t}{1 + t^2} \qquad \text{avec } t = \tan\left(\frac{a}{2}\right)\\
                \tan(a) &= \frac{2t}{1 - t^2}
            \end{aligned}$
            & Remarquer que $\begin{aligned}[b]
                \cos(a) &= \frac{\cos\left(2 \frac{a}{2}\right)}{\cos^2(a) + \sin^2(a)} \\
                &= \frac{\cos^2(a) - \sin^2(a)}{\cos^2(a) + \sin^2(a)}
            \end{aligned}$
            puis diviser en haut et en bas par $\cos^2(a)$. 
            
            De même pour $\sin$. 
            
            Prendre $b = a$ dans la formule d’addition de la tangente.
    \end{longtblr}

\newpage

\section{Calcul algébrique}

\subsection{Sommes et produits classiques}

    \begin{prop}{Somme et produit télescopique}{}
        Soient $m$ et $n$ deux entiers tels que $m \leq n$ et $a_m, \ldots,a_{n+1}$ des nombres.

        \begin{alors}
            \item $\sum\limits^n_{k=m} (a_{k+1} - a_k) = a_{n+1} - a_m$
            \item $\prod\limits^n_{k=m} \frac{a_{k+1}}{a_k} = \frac{a_{n+1}}{a_m}$
        \end{alors}
    \end{prop}

    \begin{omed}{Savoir-faire \textcolor{black}{(calculer une somme de puissances d’entiers consécutifs)}}{myolive}
        Soit $n \in \mathbb{N}^*$.
    
        Alors 
        \begin{align*} 
            \sum\limits_{k=1}^n k &= \frac{n(n+1)}{2} \\ 
            \sum\limits_{k=1}^n k^2 &= \frac{n(n+1)(2n+1)}{6} 
        \end{align*}
    \end{omed}

    \begin{demo}{Méthode}{myolive}
        Utiliser une somme télescopique.
        Pour la seconde par exemple, 
        \begin{align*}
            \sum\limits_{i=1}^n \left((k+1)^3 -k^3\right) & \overset{\text{téléscopage}}{=} (n+1)^3 -1 \\
            & \overset{\text{en développant}}{=} 3 \sum\limits_{i=1}^n k^2 + 3 \sum\limits_{i=1}^n k + \sum\limits_{i=1}^n 1
        \end{align*}
        On peut ainsi obtenir la valeur de $\sum\limits_{i=1}^n k^2$, à partir de celle de $\sum\limits_{i=1}^n k$. 
        
        Ce résultat peut se généraliser à $\sum\limits_{i=1}^n k^m$ pour $m \in \mathbb{N}$.
    \end{demo}

    \begin{prop}{Somme des termes d’une suite géométrique}{}
        Soient $(n,m) \in \mathbb{N}^2, \, m < n$ et $q \in \mathbb{C}$.
        \[ \sum\limits_{k = m}^{n} q^k = \sisinon{n-m+1}{q = 1}{q^m \times \frac{1-q^{n-m+1}}{1-q}} \]
    \end{prop}

    \begin{demo}{Preuve}{myolive}
        Utiliser une somme télescopique.
        \begin{align*}
            q \sum\limits_{k=0}^n q^k - \sum\limits_{k=0}^n q^k & \overset{\text{en factorisant}}{=} (q-1) \sum\limits_{k=0}^n q^k \\
            & \overset{\text{téléscopage}}{=} q^{n+1} - 1
        \end{align*}
        On veillera à distinguer le cas $q = 1$.
    
        Pour obtenir le résultat, utiliser que $\sum\limits_{k=m}^n q^k = q^m \sum\limits_{k=0}^{n-m} q^k  $
    \end{demo}

    \begin{prop}{Factorisation de $a^{n}-b^{n}$}{}
        Soient $n \in \mathbb{N}$ et $a$ et $b$ deux nombres.
    
        Alors 
        \[ a^{n}-b^{n} = (a-b) \sum\limits_{k = 0}^{n-1} a^{k}b^{n-1-k} \]
    \end{prop}
    
    \begin{demo}{Idée de preuve}{myolive}
        Par développement du membre de droite, avec changement d’indice, on retrouve le terme de gauche.
    \end{demo}

    \begin{prop}{Triangle de Pascal}{}
        Soient $n \in \mathbb{N}$ et $k \in \intervalleEntier{0}{n-1}$.
    
        Alors \[ \binom{n}{k} + \binom{n}{k+1} = \binom{n+1}{k+1} \]
    \end{prop}
    
    \begin{demo}{Idée}{myolive}
        Faire le calcul.
    \end{demo}

    \begin{theo}{Formule du binôme de Newton}{}
        Soient $a$ et $b$ deux nombres et $n \in \mathbb{N}$.
    
        Alors \[ (a+b)^{n} = \sum\limits_{k = 0}^{n} \binom{n}{k} a^{k} b^{n-k} \]
    \end{theo}
    
    \begin{demo}{Principe}{myred}
        Par récurrence.
    \end{demo}

\subsubsection{Sommes trigonométriques}

    \begin{omed}{Savoir-faire \textcolor{black}{(Linéariser)}}{mypink}
        Linéariser une puissance de fonction trigonométrique en somme de fonctions trigonométriques avec formule d’Euler combinée au binôme de Newton.
    \end{omed}
     
    \begin{omed}{Exemple}{mypink}
        Soit $t \in \mathbb{R}$.
    \begin{align*}
       \cos^5(t) &= \left(\frac{e^{it} + e^{-it}}{2}\right)^5 \\
       & = \frac{1}{32} \left(e^{i5t} + 5e^{i3t} + 10e^{it} + 10 e^{-it} + 5 e^{-i3t} + e^{-i5t}\right) \\
       & = \frac{1}{16} \left(\cos(5t) + 5 \cos(3t)+ 10 \cos(t)\right) 
    \end{align*}
    \end{omed}

    \begin{omed}{Savoir-faire}{mypink}
        Calculer une somme qui implique un cosinus ou sinus en $nx$.
    \end{omed}

    \begin{omed}{Exemple}{mypink}
        Soit $\theta \in \mathbb{R}$ et $n \in \mathbb{N}$.
    
        Calculons $S_n (\theta) = \sum\limits_{k=0}^n \cos(k\theta)$ et $T_n(\theta) = \sum\limits_{k=0}^n \sin(k\theta)$
    
        On remarque que $B_n (\theta) = S_n (\theta) + i T_n(\theta) = \sum\limits_{k=0}^n e^{ik\theta}$
    
        $\circ$ si $\theta \equiv 0 \, [2\pi]$, $S_n(\theta) = n+1$ et $T_n(\theta) = 0$
    
        $\circ$ si $\theta \not\equiv 0 \, [2 \pi]$, $e^{i\theta} \neq 1$ donc,
    \begin{align*}
        B_n(\theta) &= \frac{e^{i(n+1)} -1}{e^{i\theta}-1} \\
        &= \frac{e^{i(\frac{n+1}{2})\theta}}{e^{i\frac{\theta}{2}}} \frac{e^{i(\frac{n+1}{2})\theta} - e^{-i(\frac{n+1}{2})\theta}}{e^{i\frac{\theta}{2}} - e^{-i\frac{\theta}{2}}} \\
        &= \left(\cos \left(n\frac{\theta}{2}\right) + i\sin \left(n\frac{\theta}{2}\right)\right)\frac{\sin(\frac{n+1}{2}\theta)}{\sin(\frac{\theta}{2})}
    \end{align*}
    Il suffit d’utiliser $S_n = \Re(B_n)$ et $T_n = \Im(B_n)$.
    \end{omed}

    \begin{omed}{Savoir-faire}{mypink}
        Écrire une fonction trigonométrique en $nx$ comme polynôme de fonctions trigonométriques en $x$.
    \end{omed}
        
    \begin{omed}{Exemple}{mypink}
        Écrivons $\cos(5x)$ en fonction de $\cos(x)$ :
        \begin{align*}
            \cos(5x) &= \Re\left((e^{ix})^5\right) \\
            &= \Re \left((\cos(x) + i\sin(x))^5\right) \\
            &= \cos^5(x) - 10\cos^3(x)\sin^2(x) + 5 \cos(x) \sin^4(x) \\
            &= \cos^5(x) - 10\cos^3(x) (1-\cos^2(x)) + 5 \cos(x) (1 - \cos^2(x))^2 \\
            &= 16 \cos^5(x) - 20 \cos^3(x) + 5 \cos(x)
        \end{align*}
    \end{omed}

\subsection{Systèmes échelonnés et méthode du pivot}

    \begin{defi}{Opérations élémentaires}{}
    On appelle \textbf{opérations élémentaires} sur les lignes d’un système les trois opérations suivantes :
    \begin{itemize}
        \item \textbf{Échange de lignes}, noté $L_i \leftrightarrow L_j$
        \item \textbf{Dilatation}, notée $L_i \leftarrow \lambda L_j$
        \item \textbf{Transvection}, notée $L_i \leftarrow L_i + \lambda L_j$
    \end{itemize}
    \end{defi}

    \begin{omed}{Méthode \textcolor{black}{(Algorithme du pivot de Gauss)}}{myyellow}
	\begin{enumerate}
	\item Si nécessaire, échanger la première ligne avec une ligne qui contient la première variable.
	\item On divise la première ligne par le coefficient de la première variable.
	\item On élimine la première variable dans toutes les lignes suivantes en leur ajoutant un multiple approprié de $L_{1}$.
	\item On recommence avec la seconde ligne.
	\end{enumerate}
    \end{omed}

\section{Calcul dans C}

    \begin{defi}{Parties réelle et imaginaire}{}
	    Soient $x,y \in \mathbb{R}$ et $z = x + iy \in \mathbb{C}$.

	    Alors
	    \begin{itemize}
		    \item $x$ est la \textbf{partie réelle} de $z$, notée $\Re(z)$
		    \item $y$ est la \textbf{partie imaginaire} de $z$, notée $\Im(z)$
	    \end{itemize}
    \end{defi}

    \begin{prop}{Opérations sur les parties réelle et imaginaire}{}
	    Soient $z,z' \in \mathbb{C}$ et $\lambda \in \mathbb{R}$.

	    \begin{alors}
		    \item $\Re(z + \lambda z') = \Re(z) + \lambda \Re(z')$ \newline $\Im(z + \lambda z') = \Im(z) + \Lambda \Im(z')$
		    \item $\Re(zz') = \Re(z) \Re(z') - \Im(z) \Im(z')$ \newline $\Im(zz') = \Im(z) \Re(z') + \Re(z) \Im(z')$
		    \item $\Re(z^{-1}) = \frac{\Re(z)}{\Re^2(z) + \Im^2(z)}$ \newline  $\Im(z^{-1}) = \frac{-\Im(z)}{\Re^2(z) + \Im^2(z)}$
	    \end{alors}
    \end{prop}

    \begin{defi}{Conjugaison}{}
        Soient $x,y \in \mathbb{R}$ et $z = x + iy \in \mathbb{C}$.
    
        Alors le \textbf{conjugué} de $z$ est $\bar{z} = x - iy$.
    \end{defi}

    \begin{prop}{Propriété de la conjugaison}{}
        Soient $z,z' \in \mathbb{C}$ et $\lambda \in \mathbb{R}$.
    
        \begin{alors}
            \item $\overline{(z + \lambda z')} = \bar{z} + \lambda \bar{z'}$
            \item $\overline{zz'} = \bar{z} \bar{z'}$
            \item $\overline{z^n} = \bar{z}^n$
            \item $\bar{\bar{z}} = z$
            \item $\Re(z) = \frac{1}{2} \Re(z + \bar{z})$
            \item $\Im(z) = \frac{1}{2} \Im(z - \bar{z})$
        \end{alors}
    \end{prop}

    \begin{defi}{Module}{}
        Soient $x,y \in \mathbb{R}$ et $z = x + iy \in \mathbb{C}$.
    
        Alors le \textbf{module} de $z$ est le nombre $|z| = \sqrt{x^2 + y^2}$.
    \end{defi}

    \begin{prop}{Interprétation géométrique du module : cercles et disques}{}
        \begin{soient}
            \item $\vec{w} \in \vect{\mathcal{P}}$ d’affixe $z_{\vec{w}}$,
            \item $A,B \in \mathcal{P}$ d’affixes respectives $z_A,z_B$.
        \end{soient}
        \begin{alors}
            \item $|z_{\vec{w}}| = ||\vec{w}||$
            \item $|z_A| = OA$
            \item $|z_A - z_B| = AB$
            \item \begin{enumerate}[label=(\alph*)]
                \item L’ensemble des points de $\mathcal{P}$ dont l’affixe $z$ vérifie $|z - z_A| = r$ est le cercle de centre A et de rayon $r$.
                \item L’ensemble des points de $\mathcal{P}$ dont l’affixe $z$ vérifie $|z - z_A| \leq r$ est le disque fermé de centre A et de rayon $r$.
                \item L’ensemble des points de $\mathcal{P}$ dont l’affixe $z $vérifie $|z - z_A| < r$ est le disque ouvert de centre A et de rayon $r$.
            \end{enumerate}
        \end{alors}
    \end{prop}

    \begin{prop}{Propriétés du module}{}
        Soient $z,z' \in \mathbb{C}$.
    
        \begin{alors}
            \item $|z| = 0 \Leftrightarrow z = 0$
            \item $\Re(z) = |z| \Leftrightarrow z \in \mathbb{R}*$ et $\Im(z) = |z| \Leftrightarrow z \in i \mathbb{R}*$
            \item $|\Re(z)| \leq |z|$ et $|\Im(z)| \leq |z|$
            \item $|\bar{z}| = |z|$ et $||z|| = |z|$
            \item $z \bar{z} = |z^2|$ et $z^{-1} = \frac{1}{|z^2|} \bar{z}$
            \item $|zz'| = |z||z'|$ et $|z^n|=|z|^n $
            \item $\abs{\frac{z}{z'}} = \frac{|z|}{|z'|}$ et $|z^{-n}| = |z|^{-n}$
            \item $\big||z|-|z'|\big| \leq |z+z'| \leq |z|+|z'|$ avec égalité \textit{ssi}
            \[ \exists \, \lambda \in \mathbb{R}_+, \, z = \lambda z' \text{ ou } z' = 0 \]
        \end{alors}
    \end{prop}

    \begin{defi}{Argument}{}
        Soit $z \in \mathbb{C}^*$.
    
        Alors
            \begin{itemize}
                \item On appelle \textbf{argument} de $z$ et on note $\arg(z)$ tout réel $\theta$ tel que $\frac{z}{|z|} = e^{i \theta}$. 
          
                L’argument principal de $z$ est $\theta$ si $\theta \in \intervalleOF{-\pi}{\pi}$.
                \item L’écriture $z = |z|e^{i \theta}$ est appelée écriture exponentielle. 
                
                L’écriture $z = |z|(\cos\theta + i\sin\theta)$ est appelée écriture trigonométrique.
            \end{itemize}
    \end{defi}

    \begin{prop}{Propriétés de l’argument}{}
        Soient $z,z' \in \mathbb{C}$ et $\lambda \in \mathbb{R}^*$.
    
        \begin{alors}
            \item $\arg(\lambda z) \equiv \arg(z) [2 \pi] \text{ si } \lambda > 0$  et $\arg(\lambda z) \equiv \arg(z) + \pi \text{ si } \lambda < 0$
            \item $\arg(zz') \equiv \arg(z) + \arg(z') [2 \pi]$
            \item $\arg(-z) \equiv \arg(z) + \pi [2 \pi]$
            \item $\arg(\bar{z}) \equiv -\arg(z) [2 \pi]$
            \item $\arg\left(\frac{1}{z}\right) \equiv -\arg(z) [2 \pi]$
            \item $\arg\left(\frac{z}{z'}\right) \equiv \arg(z) - \arg(z') [2 \pi]$
            \item $\arg(z^n) \equiv n\arg(z) [2 \pi]$
            \item $z \in \mathbb{R}^* \iff \arg(z) \equiv 0 \, [\pi]$
		    \item $z \in \mathbb{R}^*_+ \iff \arg(z) \equiv 0 \, [2 \pi]$
		    \item $z \in i \mathbb{R}^* \iff \arg(z) \equiv \frac{\pi}{2} \, [\pi]$
        \end{alors}
    \end{prop}

    \begin{defi}{Exponentielle complexe}{}
        Soient $x,y \in \mathbb{R}$ et $z = x + iy \in \mathbb{C}$.
    
        On pose \[ e^z = e^{x +iy} = e^x e^{iy} = e^x (\cos(y) + i\sin(y)) \]
    \end{defi}

    \begin{prop}{Propriétés de l’exponentielle complexe}{}
        Soient $x,y \in \mathbb{R}$ et $z = x + iy, z' \in \mathbb{C}$.
    
        \begin{alors}
            \item $|e^z| = e^x$. En particulier, $e^z \neq 0$ et $\arg(e^z) \equiv y \, [2 \pi]$
            \item $e^{\bar{z}} = \overline{e^z}$
            \item $\exp(z + z') = \exp(z) \exp(z')$
            \item $(\exp(z))^{-1} = \exp(-z)$
            \item $\left\{ \exp(z), \, z \in \mathbb{C} \right\} = \mathbb{C}^*$
            \item $\exp(z) = \exp(z') \iff \exists \, k \in \mathbb{Z}, \, z = z' + 2ik\pi$
        \end{alors}
    \end{prop}
    
\subsection{Résolution d’équation de degré 2}

    \begin{omed}{Méthode}{myolive}
	    Pour trouver une racine carrée d’un nombre complexe $z = a +ib$ :
	    \begin{enumerate}
		    \item On pose $\omega = x +iy$,
		    \item on résout le système en $(x,y)$ obtenu en identifiant les parties réelle et imaginaire et les modules de $\omega^2$ et $z$.
	    \end{enumerate}
    \end{omed}

    \begin{prop}{}{}
        Soient $(a,b,c) \in \mathbb{C}^3$ tel que $a \neq 0$.
        
        On pose $\Delta = b^2 - 4ac$.
    
        \begin{alors}
            \item Si $\Delta = 0$, l’équation $az^2 + bz +c = 0$ admet une unique solution.
            \item Si $\Delta \neq 0$, l’équation $az^2 + bz + c = 0$ admet exactement deux solutions complexes.
        \end{alors}

        En notant $z_1$ et $z_2$ ces deux solutions (éventuellement confondues), 
        alors 
        \[ z_1 + z_2 = -\frac{b}{a} \quad \text{et} \quad z_1 z_2 = \frac{c}{a} \]

        En particulier, il existe un unique ensemble $\left\{z_1,z_2\right\}$ de complexes tels que \[ \left\{ \begin{array}{l}
			z_1 + z_2 = S \\
			z_1z_2 = P
		\end{array} \right. \]
		Ces nombres sont les solutions de $z^2 -Sz + P = 0$.
    \end{prop}

\subsection{Racines n-èmes}

    \begin{defi}{Racine $n$-ème de l’unité}{}
	    Soit $n \in \mathbb{N}$. 
	
	    Une racine $n$-ième de l’unité est $z \in \mathbb{C}$ tel que $z^{n}=1$. On note $\mathbb{U}_n$ l’ensemble des racines $n$-ièmes de l’unité, et $\mathbb{U}_n \subset \mathbb{U}$.
    \end{defi}

    \begin{prop}{L’ensemble $\mathbb{U}$}{}
        On note $\mathbb{U} = \left\{z \in \mathbb{C}, |z| = 1\right\}$ l’ensemble des racines $n$-èmes de l’unité.

        $(\mathbb{U},\times)$ est un groupe commutatif.
    \end{prop}

    \begin{demo}{Preuve}{myolive}
		Immédiat en utilisant le fait que $\abs{z} = 1$.
	\end{demo}

    \begin{theo}{Description des racines n-ièmes de l’unité}{}
        Soit $n \in \mathbb{N}^*$.
    
        Alors \[ \mathbb{U}_n = \left\{ e^{i \times \frac{2k \pi}{n}} \,|\, k \in \intervalleEntier{0}{n-1} \right\} \]
        En particulier, \[ \card(\mathbb{U}_n) = n \]
    \end{theo}

    \begin{demo}{Démonstration}{myred}
		Par double inclusion, dont l’une est évidente.

	Pour l’autre, on pose $z \in \mathbb{U}_n$. Comme $\abs{z} = 1, \, \exists \theta \in \mathbb{R}, \, z = e^{i\theta}$. De plus, $z^n = e^{i n \theta} =1 = e^0$. Donc $n \theta \equiv 0$, et donc $\exists k \in \mathbb{Z}$ tel que $z = e^{i \frac{2k \pi}{n}}$

	Pour montrer que $k \in \intervalleEntier{0}{n-1}$, utiliser la division euclidienne de $k$ par $n$. On doit aussi montrer que les éléments décrits sont disjoints, en utilisant à nouveau les congruences :
	\[ e^{i\frac{2k\pi}{n}} = e^{i \frac{2 k' \pi}{n}} \iff k \equiv k' \, [n] \]
	\end{demo}

    \begin{coro}{somme et produit des racines $n$-èmes de l’unité}{}
        Soit $n \in \mathbb{N}^* \backslash \left\{ 1\right\}$.
    
        \begin{alors}
            \item La somme des racines $n$-èmes de l’unité est 0.
            \item Le produit des racines $n$-èmes de l’unité est $(-1)^{n-1}$.
        \end{alors}
    \end{coro}

    \begin{demo}{Preuve}{myorange}
		$ \zeta_n = e^{i \frac{2\pi}{n}} \neq 1$ car $n \geq 2$.
	\begin{align*}
		\sum\limits_{k=0}^{n-1} \zeta_n^k &= \frac{\zeta_n^n -1}{\zeta_n -1} =0 \\
		\prod\limits_{k=0}^{n-1} \zeta_n^k &= \zeta_n^{\sum_{k=0}^{n-1} k} \\
		&=  \zeta_n^{\frac{n(n-1)}{2}} \\
		&= e^{i\pi (n-1)} \\
		&= (-1)^{n-1}
	\end{align*}
	\end{demo}

    \begin{theo}{Description des racines n-ièmes d’un complexe}{}
        Soient $n \in \mathbb{N}^* \backslash \left\{ 1\right\}$, $a \in \mathbb{C}^*$ et $z \in \mathbb{C}$.
    
        Alors \[ z^{n}=a \Leftrightarrow z \in \left\{\sqrt[n]{|a|} \times e^{i\frac{\arg(a)}{n}} \times e^{i\frac{2k\pi}{n}} \,|\, k \in \intervalleEntier{0}{n-1} \right\} \]
        En particulier, l’équation $z^n = a$ admet $n$ racines complexes deux à deux distinctes.
    \end{theo}
    
    \begin{demo}{Démonstration}{myred}
        Raisonner par double inclusion, dont l’une est évidente.
        
        Pour l’autre, si $z$ est une racine $n$-ème de $a$, $\abs{z}^n = \abs{a}$ donc $\abs{z} = \sqrt[n]{a}$. 
        
        De plus, \begin{align*}
            \arg(z^n) \equiv \arg(a) \, [2\pi] & \iff n\arg(z) \equiv \arg(a) \, [2 \pi] \\
            & \iff \exists k \in \mathbb{Z}, \, \arg(z) = \frac{1}{n} (\arg(a) + 2 k \pi)
        \end{align*}
    \end{demo}


\newpage

\section{Calcul polynomial}

\subsection{Définition et opérations}

    \begin{defi}{Polynôme à une indéterminée}{}
        \begin{itemize}
            \item Un \textbf{polynôme $P$ à une indéterminée} à coefficients dans $\mathbb{K}$ est une suite $(a_k)_{k \in \mathbb{N}} \in \mathbb{K}^{\mathbb{N}}$, nulle à partir d’un certain rang.
            \item Les nombres $a_0,\ldots,a_n$ sont appelés \textbf{coefficients} de $P$.
            \item Le polynôme dont tous les coefficients sont nuls est appelé \textbf{polynôme nul}, et noté 0.
            \item Si $P \neq 0$, le degré de $P$ est le plus grand entier $n$ tel que $a_n \neq 0$. On le note $\deg(P)$. 
            
            Par convention, on définit le degré du polynôme nul comme étant $\deg(0) = -\infty$. Les polynômes de degré 0 ou $-\infty$ sont appelés polynômes constants.
        \end{itemize}
    \end{defi}

    \begin{defi}{Indéterminée}{}
        On pose $X = (0,1,0,\ldots) = (\delta_{k,1})_{k \in \mathbb{N}}$. On remarque que $ \forall \ell \in \mathbb{N}^*, \, X^{\ell} = (\delta_{k,l})_{k \in \mathbb{N}}$. On convient que $X^0 = (1,0,\ldots)$.

        Soit $P = (a_0, a_1, \ldots, a_n, 0, \ldots)$ un polynôme. Alors $P = \sum\limits_{i=0}^n a_i X^i$.

        \begin{itemize}
            \item $X$ est appelée une indéterminée.
            \item Si $P \neq 0$, et $\deg(P) = n$, $a_n X^n$ est appelé terme de plus haut degré. Le nombre $a_n$ est appelé coefficient dominant. \newline
            Si $a_n = 1$, le polynôme est dit unitaire.
            \item On note alors $\mathbb{K}[X]$ l’ensemble des polynômes à une indéterminée à coefficients dans $\mathbb{K}$. \newline
            Si $n \in \mathbb{N}$, on note $\mathbb{K}_n[X]$ l’ensemble des polynômes de degré inférieur ou égal à $n$.
        \end{itemize}
    \end{defi}

    \begin{theo}{Lois sur les degrés}{}
        Soient $P = \sum\limits_{k =0}^n a_k X^k$, $Q=\sum\limits_{k =0}^m b_k X^k$ dans $\mathbb{K}[X]$, et $\lambda \in \mathbb{K}$. \newline
        On pose $N = \max(n,m)$, et $\et{\forall k > n, \, a_k = 0}{\forall k > m, \, b_k = 0}$
    
        \begin{alors}
            \item \begin{enumerate}[label=\alph*.]
                \item $P + Q = \sum\limits_{k =0}^N (a_k + b_k) X^k$
                \item $P + Q \in \mathbb{K}[X]$
                \item $\deg(P+Q) \leq \max(\deg(P),\deg(Q))$, avec égalité si $ \left\{ \begin{array}{l}
                    \deg(P) \neq \deg(Q) \\
                    \text{ou } P = Q = 0 \\
                    \text{ou } \et{\deg(P)=\deg(Q) \in \mathbb{N}}{a_{\deg(P)} + b_{\deg(Q)} \neq 0}
                \end{array} \right. $
            \end{enumerate}
            \item \begin{enumerate}[label=\alph*.]
                \item $\lambda P = \sum\limits_{k =0}^n (\lambda a_k) X^k$
                \item $\lambda P \in \mathbb{K}[X]$
                \item Si $\lambda \neq 0$, $\deg(\lambda P) = \deg(P)$
            \end{enumerate}
            \item \begin{enumerate}[label=\alph*.]
                \item $PQ = \sum\limits_{k=0}^{n+m} \left( \sum\limits_{i = 0}^k a_i b_{k-i} \right) X^k$
                \item $PQ \in \mathbb{K}[X]$
                \item $\deg(PQ) = \deg(P) + \deg(Q)$
            \end{enumerate}
        \end{alors}
    \end{theo}
    
    \begin{theo}{}{}
        Soit $n \in \mathbb{N}$.
    
        Alors $(1, X, \ldots, X^n)$ est une base de $\mathbb{K}_n[X]$, appelée base canonique de $\mathbb{K}_n[X]$.
    \end{theo}
    
    \begin{demo}{Preuve}{myred}
        La famille est génératrice par évidence. Pour étudier la liberté de cette famille, remarquer qu’un polynôme est le polynôme si et seulement si tous ses coefficients sont nuls.
    \end{demo}

    \begin{omed}{Remarque}{myred}
        Les coordonnées d’un polynôme $\sum\limits_{k=0}^n a_k X^k$ dans la base canonique sont donc $(a_0,\ldots,l_n)$. En particulier, deux polynômes sont égaux si et seulement si ils ont les mêmes coefficients.
    \end{omed}

    \begin{prop}{Toute famille de polynômes de degrés échelonnés ne contenant pas 0 est libre}{}
        Soit $\mathcal{F} = (P_0, \ldots, P_s)$ une famille d’éléments de $\mathbb{K}[X]$. \newline
        \begin{suppose}
            \item $0 \notin \mathcal{F}$
            \item $ \deg(P_0) < \ldots < \deg(P_s) $
        \end{suppose}
        Alors la famille est libre.
    \end{prop}
    
    \begin{demo}{Démonstration}{myolive}
        Raisonner par l’absurde, en posant  $i_0 = \max(\{i \in \intervalleEntier{0}{s}, \, \lambda_i \neq 0\})$ et obtenant l’égalité \[ \lambda_{i_0}P_{i_0} = -\sum\limits_{i=0}^{i_0 -1} \lambda_i P_i\] 
        Ainsi il y a non concordance des degrés car la famille est de degrés échelonnés.
    \end{demo}

\subsection{Composition}

    \begin{defi}{Composition}{}
        Soient $P = \sum\limits_{k = 0}^n a_k X^k$ et $Q$ deux éléments de $\mathbb{K}[X]$.
    
        Alors on définit \[ P \circ Q = \sum\limits_{k = 0}^n a_k Q^k \] avec la convention que $Q^0 = 1$.
        
        On le note aussi $P(Q)$.
    \end{defi}

    \begin{prop}{Propriétés de la composition}{}
        Soient $P,Q \in \mathbb{K}[X]$. 
    
        \begin{alors}
            \item $P \circ Q \in \mathbb{K}[X]$
            \item Si $\deg(Q) \geq 1$, $\deg(P \circ Q) = \deg(P) \times \deg(Q)$
        \end{alors}
    \end{prop}

\subsection{Dérivation}

    \begin{defi}{Dérivation}{}
        Soit $P = \sum\limits_{k = 0}^n a_k X^k \in \mathbb{K}[X]$.
    
        Alors le \textbf{polynôme dérivé} de $P$ est le polynôme 
        \[ \sum\limits_{k = 1}^n k a_k X^{k-1} = \sum\limits_{k = 0}^{n-1} (k+1) a_{k+1} X^k \] 
        On le note $D(P)$ ou $P'$.
    \end{defi}
    
    \begin{prop}{Propriétés du polynôme dérivé}{}
        Soient $P,Q \in \mathbb{K}[X]$, et $\lambda \in \mathbb{K}$.
    
        \begin{alors}
            \item $D(P) \in \mathbb{K}[X]$
            \item Si $\deg(P) \geq 1$, $\deg(D(P)) = \deg(P)-1$
            \item $D(P) = 0 \iff P \text{ est constant}$
            \item $D(\lambda P + Q) = \lambda D(P) + D(Q)$
            \item \begin{multline*}
                \forall s \in \mathbb{N}^*, \, \forall (\lambda_1, \ldots, \lambda_s) \in \mathbb{K}^s, \, \forall (P_1, \ldots, P_s) \in \mathbb{K}[X]^s, \\
                D\left( \sum\limits_{i=1}^s \lambda_i P_i \right) = \sum\limits_{i=1}^s \lambda_i D(P_i)
            \end{multline*}
            \item $D(PQ) = D(P)Q + PD(Q)$
            \item $D(P \circ Q) = D(Q) \times (D(P) \circ Q)$
        \end{alors}
    \end{prop}
    
    \begin{demo}{Démonstration}{myolive}
        Pour \textbf{(i)} à \textbf{(v)}, pas de difficulté particulière.
    
        \textbf{(vi)} s’obtient en développant les deux termes de l’égalité, en utilisant des réindexations de sommes. (Calcul lourd)
    
        \textbf{(vii)} Par récurrence finie sur $\intervalleEntier{1}{n}$, en utilisant (vi) dans l’hypothèse de récurrence, on obtient que $ \forall k \in \mathbb{N}^*, \, D(Q^k) = k Q^{k-1} D(Q) $. Ainsi, 
        \begin{align*}
            D(P \circ Q) &= D \left(\sum\limits_{k=0}^n a_k Q^k\right) \\
            &= \sum\limits_{k=0}^n a_k D(Q^k) \\
            &= \sum\limits_{k=1}^{n}k a_k Q^{k-1} D(Q) \\
            &= (D(P) \circ Q)D(Q)
        \end{align*}
    \end{demo}

    \begin{defi}{Dérivée $k$-ème}{}
        Soit $P \in \mathbb{K}[X]$, et $k \in \mathbb{N}^*$.
    
        Alors la \textbf{dérivée $k$-ème} du polynôme $P$ est
        \[ D^k(P)= \sisinon{P}{k = 0}{D(D^{k-1}(P))} \] 
        On la note aussi $P^{(k)}$.
    \end{defi}
    
    \begin{prop}{}{}
        Soient $a \in \mathbb{K}$ et $(k,n) \in \mathbb{N}^2$.
    
        Alors 
        \[ D^k((X-a)^n) = \sisinon{0}{k > n}{\frac{n!}{(n-k)!}(X-a)^{n-k}} \]
        En particulier, $D^n((X-a)^n) = n!$.
    \end{prop}
    
    \begin{demo}{Preuve}{myolive}
        Par récurrence finie sur $\intervalleEntier{0}{n}$
    \end{demo}
    
    \begin{coro}{}{}
        Soient $P \in \mathbb{K}[X]$ et $k \in \mathbb{N}$.
    
        Alors \[ D^k(P) = 0 \iff \deg(P) \leq k-1 \] 
    \end{coro}
    
    \begin{demo}{Démonstration}{myorange}
        Étudier à part le cas $\deg(P) = -\infty$.
    
        Sinon, discuter en fonction $k$ (si supérieur ou non à $\deg(P)$). Si ce n’est pas le cas, remarque qu’on peut appliquer la proposition précédente à $X^k$.
    \end{demo}
    
    \begin{theo}{Formule de Leibniz}{}
        \begin{soient}
            \item $P,Q \in \mathbb{K}[X]$
            \item $k \in \mathbb{N}$
        \end{soient}
    
        Alors \[ D^k(PQ) = \sum\limits_{l=0}^k \binom{k}{l} D^l(P) D^{k-l}(Q) \]
    \end{theo}
    
    \begin{demo}{Idée de démonstration}{myred}
        De la même façon que le binôme de Newton, par récurrence, en utilisant dans l’hérédité une réindexation de somme et la formule du triangle de Pascal.
    \end{demo}

    \begin{theo}{Formule de Taylor}{}
        \begin{soient}
            \item $P \in \mathbb{K}[X]$
            \item $n \geq \deg(P)$
            \item $a \in \mathbb{K}$
        \end{soient}
    
        Alors \[ P = \sum\limits_{k=0}^n \frac{\widetilde{D^k(P)}(\alpha)}{k!}(X-\alpha)^k \]
    \end{theo}
    
    \begin{demo}{Démonstration}{myred}
        Pour $n \in \mathbb{N}$, on pose \begin{multline*}
            \mathcal{H}_n \, : \, \forall P \in \mathbb{K}[X], \, \deg(P) \leq n \\
            \implies P = \sum\limits_{k=0}^n \frac{\widetilde{D^k(P)}(\alpha)}{k!}(X-\alpha)^k
        \end{multline*}
        $\mathcal{H}_0$ se démondre aisément.
        
        Soit $n \in \mathbb{N}$ tel que $\mathcal{H}_n$ est vraie. Soit $P \in \mathbb{K}[X]$ tel que $\deg(P) \leq n+1$. 
        Appliquer $\mathcal{H}_n$ à $D(P)$. Ensuite, remarquer, en posant 
        \[ R = \sum\limits_{k=0}^{n+1} \frac{\widetilde{D^k(P)}(\alpha)}{k!}(X-\alpha)^k \]
        que $D(R) = D(P)$, i.e. $P-R$ est un polynôme constant. Or, l’évaluation en $\alpha$ donne $P = R$. Donc $\mathcal{H}_{n+1}$ est vraie.
    \end{demo}
    
    \begin{omed}{Remarque}{myred}
        La famille $(1,(X-\alpha), \ldots, (X-\alpha)^n)$ est de degrés échelonnés et ne contient pas 0 donc est libre. Le théorème donne que cette famille est génératrice de $\mathbb{K}_n[X]$ et donne donc les coordonnées d’un polynôme dans cette base.
    \end{omed}

\subsection{Polynômes Caractéristiques}

    \subsubsection{Polynômes interpolateurs de Lagrange}

    \begin{defi}{Polynômes interpolateurs de Lagrange}{}
        Soient $n \in \mathbb{N}^*$ et $(a_0,\ldots,a_n) \in \mathbb{K}^{n+1}$ des scalaires distincts. 
        
        Pour $i \in \intervalleEntier{0}{n}$, on pose le $i$-ème \textbf{polynôme interpolateur de Lagrange}
        \[ L_i(X) = \prod_{\substack{j = 0 \\ j \neq i}}^{n} \frac{X - a_j}{a_i - a_j} \]
    \end{defi}

    \begin{omed}{Remarque}{myyellow}
        Dans certaines situations, on pose seulement 
        \[ L_i(X) = \prod_{\substack{j = 0 \\ j \neq i}}^{n} (X - a_j) \]
        qui est du moins unitaire.
    \end{omed}

    \begin{prop}{Base de $\mathbb{K}_n[X]$}{}
        $(L_0,\ldots, L_n)$ est une base de $\mathbb{K}_n[X]$.
    \end{prop}

    \begin{demo}{Preuve}{myolive}
        On sait que $\dim(\mathbb{K}_n[X]) = n+1$. Or $\card(L_0,\ldots,L_n) = n+1$. Donc $(L_0,\ldots,L_n)$ est une base \textit{ssi} elle est libre.

        Soient $(\lambda_0,\ldots,\lambda_n) \in \mathbb{K}^{n+1}$ tels que 
        \[ \sum_{k=0}^{n} \lambda_k L_k = 0 \]  
        En en déduit que pour tout $i \in \intervalleEntier{0}{n}$, 
        \[ \sum_{k=0}^{n} \lambda_k L_k(a_i) = 0 \]   
        où $L_k(a_i) = \sisi{0}{i \neq k}{1}{i = k}$. Donc $\lambda_i = 0$ pour $i \in \intervalleEntier{0}{n}$, d’où la famille est libre.
    \end{demo}

    \begin{omed}{Première application \textbf{(DEES)}}{myolive}
        Soit 
        \[ F(X) = \frac{P(X)}{\prod_{i=0}^n (X-a_i)} \] 
        où $\deg(P) \leq n$ et $a_0,\ldots,a_n$ sont distincts.

        Alors il existe des uniques $\lambda_0,\ldots,\lambda_n$ tels que 
        \[ F = \sum_{i=0}^{n} \frac{\lambda_i}{X - a_i} \]
    \end{omed}

    \begin{demo}{Preuve}{myolive}
        \begin{align*}
            \frac{P(X)}{\prod_{i=0}^{n} X - a_i} 
            &= \sum_{i=0}^{n} \frac{\lambda_i}{X - a_i} \\
            &= \frac{\sum_{i=0}^{n} \lambda_i L_i(X)}{\prod_{i=0}^{n} X-a_i}
        \end{align*}
        Ce qui équivaut à 
        \[ P(X) = \sum_{i=0}^{n} \lambda_i L_i(X) \]   
        D’où l’existence et l’unicité de $(\lambda_0,\ldots,\lambda_n)$ car $(L_0,\ldots,L_n)$ est une base de $\mathbb{K}_n[X]$.
    \end{demo}

    \begin{exo}{DESS}{}
        Donner la DEES du polynôme 
        \[ F = \frac{1}{X^n - 1} \]   
        dans $\mathbb{C}[X]$
    \end{exo}

    \begin{demo}{Solution}{nfpgreen}
        On sait que 
        \[ X^n - 1 = \sum_{k=1}^{n} \left(X - e^{\frac{2ik\pi}{n}}\right) \]
        D’après le résultat précédent, 
        \[ \frac{1}{X^n - 1} = \sum_{k=1}^{n} \frac{\lambda_k}{X - e^{\frac{2ik\pi}{n}}} \]
        Pour tout $k \in \intervalleEntier{1}{n}$, 
        \begin{align*}
            \lambda_k 
            &= \left. \frac{X - e^{\frac{2ik\pi}{n}}}{X^n - 1} \right|_{X = e^{\frac{2ik \pi}{n}}} \\
            \text{Or } X^n - 1 
            &= \left( X - e^{\frac{2ik\pi}{n}} \right) Q(X) \\
            \text{Donc } nX^{n-1} 
            &= Q(X) + (X- e^{\frac{2ik\pi}{n}}) Q'(X) \\
            \text{d’où } n \left( \left(e^{\frac{2ik\pi}{n}}\right)^{n-1} \right)
            &= Q\left( e^{\frac{2ik\pi}{n}} \right) \\
            \text{Ainsi, } \lambda_k 
            &= \frac{1}{Q\left(e^{\frac{2ik\pi}{n}}\right)} \\
            &= \frac{1}{n \left( \left(e^{\frac{2ik\pi}{n}}\right)^{n-1} \right)} \\
            &= \frac{e^{\frac{2ik\pi}{n}}}{n}
        \end{align*}
        Donc 
        \[ \frac{1}{X^n -1} = \frac{1}{n} \sum_{k=1}^{n} \frac{e^{\frac{2ik\pi}{n}}}{(X - e^{\frac{2ik\pi}{n}})} \]
    \end{demo}

    \begin{omed}{Application \textcolor{black}{(Isomorphisme)}}{myolive}
        Soit $E = \mathbb{K}_n[X]$ et $(a_0,\ldots,a_n) \in \mathbb{K}^{n+1}$ distincts.

        Alors 
        \[ \fonction{\Phi}{E}{\mathbb{K}^{n+1}}{P}{\left(P(a_0),\ldots,P(a_n)\right)} \]  
        est un isomorphisme.
    \end{omed}

    \begin{demo}{Preuve}{myolive}
        \begin{itemize}
            \item \textbf{Linéarité} \quad Par évidence.
            \item \textbf{Bijectivité} \quad Comme $\Phi$ est linéaire et $\dim(E) = n+1 = \dim(\mathbb{K}^{n+1})$, il suffit de montrer que $\Phi$ est injective, \textit{i.e.} que $\ker(\Phi) = \{0\}$.
            Soit $P \in \ker(\Phi)$. Alors pour tout $i \in \intervalleEntier{0}{n}$, $P(a_i) = 0$ donc $P$ admet $n+1$ racines distinctes. Or $\deg(P) \leq n$, donc $P = 0$.
        \end{itemize}
    \end{demo}

    \begin{omed}{Remarque}{myolive}
        Comme $\Phi$ est bijective, pour tout $(\alpha_0,\ldots,\alpha_n)$, il existe un unique polynôme $P$ tel que 
        \[ \forall i \in \intervalleEntier{0}{n} \quad P(a_i) = \alpha_i \]  
        Ainsi, 
        \[ P = \sum_{i=0}^{n} \alpha_i L_i \]  
    \end{omed}

    \subsubsection{Polynômes de Tchebychev}

    \begin{defi}{Polynômes de Tchebychev}{}
        Soit $n \in \mathbb{N}$. 

        On appelle \textbf{polynôme de Tchebychev} de degré $n$ l’application définie par 
        \[ \fonction{T_n}{\intervalleFF{-1}{1}}{\mathbb{R}}{x}{\cos(n \arccos(x))} \] 
    \end{defi}

    On peut calculer explicitement la valeur de ces polynômes : 
    \begin{align*}
        T_n(x)&=\Re\left((\cos n\alpha +i\sin n\alpha)\right)\\
        &=\Re\left((\cos \alpha +i\sin \alpha)^n\right)\\
        &=\Re\left((\cos \arccos x +i\sin \arccos x)^n\right)\\
        &=\Re\left(( x +i\sqrt{1-x^2})^n\right)\\
        &=\Re\left(\sum_{k=0}^n \binom{n}{k} x^{n-k} i^{k}(\sqrt{1-x^2})^k\right)\\
        &=\sum_{k=0}^{\ent{n/2}} \binom{n}{2k} x^{n-2k} (i)^{2k}(1-x^2)^k\\
        &=\sum_{k=0}^{\ent{n/2}} \binom{n}{2k} x^{n-2k}(x^2-1)^k
        \end{align*}
    On remarque que \[ \forall n \in \mathbb{N}, \lilbox{myolive}{$T_n(-1) = (-1)^n$} \text{ et } \lilbox{myolive}{$T_n(1) = 1$} \] 
    On peut aussi trouver aisément \[ \forall n \in \mathbb{N}, T_n(0) = \sisi{0}{n \text{ est impair}}{(-1)^{\frac{n}{2}}}{n \text{ est pair}} \]

    \begin{prop}{Relation de récurrence}{}
        $T_0(x) = 1$, $T_1(x) = x$ et pour tout $n \in \mathbb{N}$,
        \[ T_{n+2}(x) = 2x T_{n+1}(x) - T_n(x) \]
    \end{prop}

    \begin{omed}{Remarque}{myolive}
        Les polynômes de Tchebychev sont parfois définis par cette relation de récurrence. Pour retrouver la formule de la définition, on peut montrer par récurrence double que \[ \forall n \in \mathbb{N}, T_n(\cos(\alpha)) = \cos(n \alpha) \] en utilisant les mêmes formules de trigonométries que dans la preuve qui suit.
    \end{omed}

    \begin{demo}{Preuve}{myolive}
        Soit $x \in \intervalleFF{-1}{1}$ et $\alpha = \arccos(x)$. 

        En utilisant les formules d’addition et de soustraction du cosinus, on obtient 
        \begin{align*}
            \cos((n+2)\alpha) &= \cos((n+1)\alpha)\cos(\alpha) - \sin((n+1)\alpha)\sin(\alpha) \\
            \cos(n\alpha) &= \cos((n+1)\alpha)\cos(\alpha) + \sin((n+1)\alpha)\sin(\alpha)
        \end{align*}
        En sommant les deux égalités, on obtient : 
        \[ \cos((n+2)\alpha) + \cos(n\alpha) = 2 \cos((n+1)\alpha) \cos(\alpha) \] 
        On a alors, pour tout $x \in \intervalleFF{-1}{1}$, 
        \[ T_{n+2}(x) + T_n(x) = 2x T_{n+1}(x) \] 
    \end{demo}

    \begin{prop}{}{}
        Soit $n \in \mathbb{N}^*$.

        Le coefficient dominant de $T_n$ est $2^{n-1}$ et $\deg(T_n) = n$.
    \end{prop}

    \begin{demo}{Preuve}{myolive}
        Par récurrence double.
    \end{demo}

    Soit $n \in \mathbb{N}^*$. On remarque que pour $k$ parcourant $\intervalleEntier{0}{n-1}$, 
    \[ T_n \left(\cos\left(\frac{(2k+1)\pi}{2n}\right)\right) = \cos\left(\frac{\pi}{2} + k\pi\right) = 0 \] 
    Or pour $k \in \intervalleEntier{0}{n-1}, \frac{(2k+1)\pi}{2n} \in \intervalleFF{0}{\pi}$ et cos est strictement décroissante sur cet intervalle, donc injective. Donc les $\cos \left(\frac{(2k+1)\pi}{2n}\right)$ sont deux à deux distincts. 
    
    On a trouvé $n$ racines deux à deux distinctes d’un polynôme de degré $n$, donc l’ensemble des racines de $T_n$ est \[ \lilbox{myolive}{$\cos\left(\frac{(2k+1)\pi}{2n}\right), k \in \intervalleEntier{0}{n-1}$} \] et ces racines sont toutes simples.

    \begin{itemize}
        \item Le produit des racines de $T_n$, comptées autant de fois que leur multiplicité, est $(-1)^n \frac{a_0}{a_n}$, où $a_0$ est le coefficient constant et $a_n$ le coef. dominant de $T_n$. 
        
        Ainsi,
        \[ \lilbox{myolive}{$ \prod\limits_{k=0}^{n-1} \cos \left( \frac{(2k+1)\pi}{2n}\right) $} = \sisi{0}{n \text{ est impair}}{\frac{(-1)^{n/2}}{2^{n-1}}}{n \text{ est pair}} \] 
        \item La somme des racines de $T_n$ est $-\frac{a_{n-1}}{a_n}$, où $a_{n-1}$ est le coefficient de $X^{n-1}$ dans $T_n$. Or on peut démontrer que ce coefficient est nul, donc 
        \[ \lilbox{myolive}{$ \sum\limits_{k=0}^{n-1} \cos \left( \frac{(2k+1)\pi}{2n}\right) $} = 0 \] 
    \end{itemize}

    \begin{prop}{}{}
        Les polynômes de Tchebychev sont solutions de l’équation différentielle 
        \[ (1-X^2) T_n'' - XT_n' + n^2 T_n = 0 \]
    \end{prop}

    \begin{prop}{}{}
        Soit $n, m \in \mathbb{N}$. 

        Alors $T_m \circ T_n (X) = T_{mn}(X)$
    \end{prop}

    \begin{prop}{}{}
        Les polynômes de Tchebychev forment une famille orthogonale pour le produit scalaire 
        \[ \spr{f}{g} = \int_{-1}^{1} f(t)g(t) \frac{1}{\sqrt{1-t^2}} dt \]
        Plus précisément, on a, pour $m,n \in \mathbb{N}$,
       \[ \int_{-1}^1 T_n(x)T_m(x)\frac 1{\sqrt{1-x^2}}dx=
    \left\{
	\begin{array}{cl}
	0 &\text{si } n \neq m \\
	\pi &\text{si }n = m = 0 \\
	\frac{\pi}2 &\text{si } n = m \neq 0.
    \end{array}
    \right. \]
    \end{prop}

\newpage

\section{Calcul matriciel}

\subsection{L’ensemble MnK}

    \begin{theo}{}{}
        $(\mk{n,p},+,.)$ est un $\mathbb{K}$-espace vectoriel.
    \end{theo}

    \begin{theo}{}{}
        La famille $\{ E_{i,j}, (i,j) \in \intervalleEntier{1}{n} \times \intervalleEntier{1}{p} \}$ est une base, dite base canonique de $\mk{n,p}$.
    \end{theo}

    \begin{demo}{Idée}{myred}
        La famille est bien libre, car une matrice est nulle si et seulement si tous ses coefficients sont nuls. Elle est aussi génératrice car toute matrice peut être décomposée comme $\sum\limits_{i,j \in \intervalleEntier{1}{n} \times \intervalleEntier{1}{p}} a_{i,j}E_{i,j} \in \Vect((E_{i,j})_{i,j \in \intervalleEntier{1}{n} \times \intervalleEntier{1}{p}})$
    \end{demo}

\subsection{Produit de matrices}

    \begin{defi}{Coefficients du produit de matrices}{}
        Soient $A \in \mk{n,p}$ et $B \in \mk{p,q}$ deux matrices telles que le nombre de lignes de $B$ soit égal au nombre de colonnes de $A$.

        Alors le produit des matrices $A$ et $B$ est la matrice $C \in \mk{n,q}$ dont le coefficient $c_{i,j}$ est donné par 
        \[ c_{i,j} = \sum\limits_{k=1}^p a_{i,k} b_{k,j} \]
        On le note $A \times B$, $A.B$ ou $AB$.
    \end{defi}

    \begin{prop}{}{}
        \begin{soient}
            \item $A \in \mk{n,p}$, $B \in \mk{p,q}$ et $X \in \mk{p,1}$
            \item $(i,j) \in \intervalleEntier{1}{n} \times \intervalleEntier{1}{p}$
        \end{soient}

        \begin{alors}
            \item Le coefficient d’indice $(i,j)$ de $AB$ est le coefficient de la matrice produit de la $i$-ème ligne de $A$ par la $j$-ème colonne de $B$.
            \item La $j$-ème colonne de $AB$ est le produit de $A$ par la $j$-ème colonne de $B$.
            \item La $i$-ème ligne de $AB$ est le produit de la $i$-ème ligne de $A$ par $B$.
            \item $AX$ est une combinaison linéaire des colonnes de $A$.
        \end{alors}
    \end{prop}

    \begin{demo}{Preuve}{myolive}
        L’écrire et s’en convaincre.
    \end{demo}

    \begin{prop}{}{}
        Soit $A \in \mk{n,p}$.

        \begin{alors}
            \item Soit $(i,j) \in \intervalleEntier{1}{q} \times \intervalleEntier{1}{n}$ et $E_{i,j} \in \mk{q,n}$. 
            
            $E_{i,j} A$ est la matrice de taille $q \times p$ dont toutes les lignes sont nulles sauf la $i$-ème qui vaut la $j$-ème ligne de $A$.
            \item Soit $(i,j) \in \intervalleEntier{1}{p} \times \intervalleEntier{1}{q}$ et $E_{i,j \in \mk{p,q}}$. 
            
            $A E_{i,j}$ est la matrice de taille $n \times q$ dont touts les colonnes sont nulles sauf la $j$-ème qui vaut la $i$-ème colonne de $A$.
        \end{alors}
    \end{prop}

    \begin{demo}{Preuve}{myolive}
        L’écrire.
    \end{demo}

    \begin{coro}{}{}
        Soient $(i,j) \in \intervalleEntier{1}{n}^2$ et $(k,l) \in \intervalleEntier{1}{n}^2$.
        
        On note $E_{i,j}$ et $E_{k,l}$ les matrices élémentaires de taille $n \times n$ correspondantes. 

        Alors \[ E_{i,j} E_{k,l} = \sisinon{0_n}{k \neq j}{E_{i,l}} \]
    \end{coro}

    \begin{theo}{Formules du binôme et de factorisation pour les matrices}{}
        Soient $A,B \in \mk{n}$.
        
        On suppose que $A$ et $B$ commutent.

        Alors pour tout $p \in \mathbb{N}$, 
        \begin{enumerate}
            \item $(A+B)^p = \sum\limits_{k=0}^p \binom{p}{k} A^k B^{p-k}$
            \item $A^p - B^p = (A-B) \sum\limits_{k=0}^{p-1} A^k B^{p-1-k}$
        \end{enumerate}
    \end{theo}

    \begin{prop}{Effet de la multiplication par une matrice diagonale}{}
        \begin{soient}
            \item $D$ une matrice diagonale de taille $n$
            \item $A \in \mk{n}$
        \end{soient}

        Alors \[ DA = \begin{pmatrix}
            d_1 L_1 \\
            \vdots \\
            d_n L_n
        \end{pmatrix} \quad AD = \begin{pmatrix}
            d_1 C_1 & \ldots & d_n C_n 
        \end{pmatrix} \]
        En particulier, si $A$ est diagonale, \[ AD = \begin{pmatrix}
            a_1 d_1 & 0 & \ldots & 0 \\
            0 & \ddots & \ddots & \vdots \\
            \vdots & \ddots & \ddots & 0 \\
            0 & \ldots & 0 & a_n d_n
            \end{pmatrix} \] 
    \end{prop}

    \begin{prop}{Effet de la multiplication par une matrice triangulaire supérieure}{}
        \begin{soient}
            \item $T$ une matrice triangulaire supérieure de taille $n$
            \item $A \in \mk{n}$
        \end{soient}

        Alors \begin{align*}
            TA & = \begin{pmatrix}
                t_{1,1} L_1 + \ldots + t_{1,n} L_n \\
                \vdots \\
                t_{n,n} L_n
            \end{pmatrix} \\
            AT & = \begin{pmatrix}
                t_{1,1} C_1 & \ldots & t_{1,n} C_1 + \ldots + t_{n,n} C_n  
            \end{pmatrix} 
        \end{align*}
    \end{prop}

\subsection{Matrices d’opérations élémentaires}

    \begin{defi}{Opérations élémentaires}{}
        Soient $(i,j) \in \intervalleEntier{1}{n}^2$ tel que $i \neq j$ et $\lambda \in \mathbb{K}$.

        On appelle \textbf{opérations élémentaires} sur les lignes d’une matrice les trois opérations suivantes :
        \begin{enumerate}
            \item échange de lignes ($L_i \leftrightarrow L_j$)
            \item dilatation ($L_i \leftarrow \lambda L_i$)
            \item transvection ($L_i \leftarrow L_i + \lambda L_j$)
        \end{enumerate}
        On définit de même les opérations élémentaires sur les colonnes.
    \end{defi}

    \begin{defi}{Matrice de dilatation}{}
        Soient $i \in \intervalleEntier{1}{n}$ et $\lambda \in \mathbb{K}^*$.

        On appelle \textbf{matrice de dilatation} de taille $n$ tout matrice de taille $n \times n$ de la forme 
        \[ D_i(\lambda) = \begin{pmatrix}
            1 & 0 & \ldots & \ldots & \ldots & \ldots & 0 \\
            0 & \ddots & \ddots & & & & \vdots \\
            \vdots & \ddots & 1 & \ddots & & & \vdots \\
            \vdots & & \ddots & \lambda & \ddots & & \vdots \\
            \vdots & & & \ddots & 1 & \ddots & \vdots \\
            \vdots & & & & \ddots & \ddots & 0 \\
            0 & \ldots & \ldots & \ldots & \ldots & 0 & 1
        \end{pmatrix} \] où $\lambda$ est le coefficient d’indice $(i,i)$.
    \end{defi}

    \begin{prop}{Produit par une matrice de dilatation}{}
        Soit $A \in \mk{n,p}$.

        \begin{alors}
            \item Pour tout $i \in \intervalleEntier{1}{n}$, pour tout $\lambda \in \mathbb{K}^*$, $D_i(\lambda)A$ est la matrice obtenue à partir de $A$ par la dilatation $L_i \leftarrow \lambda L_i$.
            \item Pour tout $j \in \intervalleEntier{1}{j}$, pour tout $\lambda \in \mathbb{K}^*$, $AD_j(\lambda)$ est la matrice obtenue à partir de $A$ par la dilatation $C_j \leftarrow \lambda C_j$.
        \end{alors}
    \end{prop}

    \begin{defi}{Matrice de transvection}{}
        Soient $(i,j) \in \intervalleEntier{1}{n}^2$ tel que $i \neq j$ et $\lambda \in \mathbb{K}^*$.
    
        On appelle \textbf{matrice de transvection} de taille $n$ toute matrice de taille $n \times n$ de la forme
        \[ T_{i,j}(\lambda) = \begin{pmatrix}
            1 & 0 & \ldots & \ldots & \ldots & \ldots & 0 \\
            0 & \ddots & \ddots & & \lambda & & \vdots \\
            \vdots & \ddots & \ddots & \ddots & & & \vdots \\
            \vdots & & \ddots & \ddots & \ddots & & \vdots \\
            \vdots & & & \ddots & \ddots & \ddots & \vdots \\
            \vdots & & & & \ddots & \ddots & 0 \\
            0 & \ldots & \ldots & \ldots & \ldots & 0 & 1
        \end{pmatrix} \] où $\lambda$ est le coefficient d’indice $(i,j)$.
    \end{defi}

    \begin{prop}{Produit par une matrice de transvection}{}
        Soit $A \in \mk{n,p}$.

        \begin{alors}
            \item Pour tout $(i,j) \in \intervalleEntier{1}{n}^2$, pour tout $\lambda \in \mathbb{K}^*$, $T_{i,j}(\lambda)A$ est la matrice obtenue à partir de $A$ par la transvection $L_i \leftarrow L_i + \lambda L_j$.
            \item Pour tout $(i,j) \in \intervalleEntier{1}{j}^2$, pour tout $\lambda \in \mathbb{K}^*$, $AT_{i,j}(\lambda)$ est la matrice obtenue à partir de $A$ par la transvection $C_j \leftarrow C_j + \lambda C_i$.
        \end{alors}
    \end{prop}

    \begin{defi}{Matrice de transposition}{}
        Soit $(i,j) \in \intervalleEntier{i}{j}^2$.

        On appelle \textbf{matrice de transposition} de taille $n$ toute matrice $P_{i,j}$ de taille $n \times n$ de la forme 
        \[ \begin{pmatrix}
            1 & 0 & \ldots & \ldots & \ldots & \ldots & \ldots & \ldots & \ldots & \ldots & 0 \\
            0 & \ddots &  &  &  &  &  &  &  &  & \vdots \\
            \vdots &  & 1 &  &  &  &  &  &  &  & \vdots \\
            \vdots &  &  & 0 & \ldots & \ldots & \ldots & 1 &  &  & \vdots \\
            \vdots &  &  & \vdots & 1 &  &  & \vdots &  &  & \vdots \\
            \vdots &  &  & \vdots &  & \ddots &  & \vdots &  &  & \vdots \\
            \vdots &  &  & \vdots &  &  & 1 & \vdots &  &  & \vdots \\
            \vdots &  &  & 1 & \ldots & \ldots & \ldots & 0 &  &  & \vdots \\
            \vdots &  &  &  &  &  &  &  & 1 &  & \vdots \\
            \vdots &  &  &  &  &  &  &  &  & \ddots & 0 \\
            0 & \ldots & \ldots & \ldots & \ldots & \ldots & \ldots & \ldots & \ldots & 0 & 1 
            \end{pmatrix} \] 
            où les coefficients d’indices $(i,i)$ et $(j,j)$ valent 0 et ceux d’indices $(i,j)$ et $(j,i)$ 1.
    \end{defi}

    \begin{prop}{Produit par une matrice de transposition}{}
        Soit $A \in \mk{n,p}$.

        \begin{alors}
            \item Pour tout $(i,j) \in \intervalleEntier{1}{n}^2$, $P_{i,j}A$ est la matrice obtenue à partir de $A$ par l’échange de ligne $L_i \leftrightarrow L_j$.
            \item Pour tout $(i,j) \in \intervalleEntier{1}{j}^2$, $AP_{i,j}$ est la matrice obtenue à partir de $A$ par l’échange de colonnes $C_j \leftrightarrow C_i$.
        \end{alors}
    \end{prop}

\subsection{Matrices inversibles}

    \subsubsection{Propriétés}

    \begin{prop}{}{}
        Les matrices d’opérations élémentaires sont inversibles.
    \end{prop}

    \begin{prop}{}{}
        Soit $A \in \mk{n}$.

        \begin{alors}
            \item Si $A \in \mathcal{G}\ell_n(\mathbb{K})$, alors $A$ admet un unique inverse.
            \item \begin{align*}
                A \in \mathcal{G}\ell_n(\mathbb{K}) & \iff \exists B \in \mk{n}, BA = I_n \\
                & \iff \exists B \in \mk{n}, AB = I_n 
            \end{align*}
        \end{alors}
    \end{prop}

    \begin{prop}{}{}
        Soit $A \in \mk{n}$.

        Alors $A$ est inversible si et seulement si pour tout $Y \in \mk{n,1}$, le système $AX = Y$ admet une unique solution $X \in \mk{n,1}$.
    \end{prop}

    \begin{coro}{}{}
        Soit $T$ une matrice triangulaire. 

        Alors $T$ est inversible si et seulement si ses coefficients diagonaux sont tous non-nuls.
    \end{coro}

    \begin{coro}{}{}
        Soit $D$ une matrice diagonale de coefficients $d_1,\ldots,d_n$.

        Alors $D$ est inversible si $\forall k \in \intervalleEntier{1}{n}, \, d_k \neq 0$, et dans ce cas, $D^{-1}$ est la matrice diagonale de coefficients $\frac{1}{d_1},\ldots,\frac{1}{d_n}$.
    \end{coro}

    \subsubsection{Calcul de l’inverse d’une matrice carrée par le pivot de Gauss}

    \begin{omed}{Méthode \textcolor{black}{(Pivot de Gauss)}}{mybrown}
        On écrit $A$ et $I_n$ côte à côte, puis on applique le pivot de Gauss à $A$ dans une colonne jusqu’à arriver à $I_n$, et on fait les mêmes opérations sur les lignes de $I_n$.

        Si cela est possible, $A^{-1}$ est alors la matrice obtenue à partir de $I_n$.
    \end{omed}

    \begin{demo}{Pourquoi ça marche ?}{mybrown}
        \begin{itemize}
            \item Chaque opération revient à multiplier la matrice résultant de $A$ par une matrice d’opération élémentaire $E_{\ell}$.
            \item La matrice résultant de $I_n$ stocke dans l’ordre les matrices d’opérations élémentaires utilisées : à chaque étape, si on note $A_{\ell}$ la matrice résultant de $A$ et $B_{\ell}$ celle de $I_n$, $A_{ell} = B_{\ell} A$. À la fin de l’algorithme, on a bien $B_{\ell,\text{fin}} A = I_n$.
        \end{itemize}
    \end{demo}

\newpage

\section{Calcul de primitives}

\subsection{Primitives de fonctions usuelles}

    \begin{longtblr}[
        caption={Primitives usuelles}
        ]{
            colspec={|X[3,c]||X[2,c] |}, width = \linewidth,
            rowhead = 1, 
            hlines={0.4pt, black},
            row{odd} = {myolive!30}, row{1} = {myolive, fg=white, font=\bfseries},
            rows = {1cm}
        }
        $f(x)$ &  $F(x)$ \\
        $x^a$ & $ \frac{x^{a+1}}{a+1}$ \\
        $\ln(x)$ & $x\ln(x) - x$ \\
        $\tan(x)$ & $-\ln|\cos(x)|$ \\
        $\frac{1}{1 + x^2}$ & $\arctan(x)$ \\
        $\frac{1}{\sqrt{1 - x^2}}$ & $\arcsin(x) \esp{ou} -\arccos(x)$ \\
        $1 + \tan^2(x) = \frac{1}{\cos^2(x)}$ & $\tan(x)$ \\
    \end{longtblr}

\subsection{Primitives d’une composée}

    \begin{longtblr}[
        caption={Primitives d’une composée}
        ]{
            colspec={|X[3,c]||X[2,c] |}, width = \linewidth,
            rowhead = 1, 
            hlines={0.4pt, black},
            row{odd} = {myolive!30}, row{1} = {myolive, fg=white, font=\bfseries},
            rows = {1cm}
        }
        $f$ & $F$ \\
        $\frac{u'}{u}$ & $\left( \ln|u| \right)'$ \\
        $u'u$ & $\left(\frac{1}{2}u^2\right)'$ \\
        $u'u^k$ & $\left( \frac{1}{k + 1} u^{k+1} \right)'$ \\
        $u'e^u$ & $\left(e^u\right)'$ \\
        $\frac{u'}{1 + u^2}$ & $\left(\arctan(u)\right)'$
    \end{longtblr}

\subsection{Primitives de fonctions trigonométriques}

    \begin{omed}{Méthode \textcolor{black}{(Calcul de primitives trigonométriques)}}{mybrown}
        Soient $p,q \in \mathbb{N}^*$.
        \begin{itemize}
            \item Primitives de la forme $\cos^p \sin^q$.
            \begin{itemize}
                \item Si $p$ ou $q$ impair : conserver un cos ou sin et remplacer les autres avec $\cos^2 + \sin^2 = 1$ puis primitiver.
                \item  Si $p$ et $q$ pairs, linéariser le tout avec Moivre et primitiver.
            \end{itemize}
            \item Primitives de la forme $x \mapsto e^{ax}\cos(bx)$ ou $x \mapsto e^{ax}\sin(bx)$.
            
            Ce sont les parties réelles et imaginaires de $x \mapsto e^{(a+ib)x}$, que l’on sait primitiver en 
            \[ \left(\frac{e^{(a+ib)x}}{a + ib}\right)\]
        \end{itemize}
    \end{omed}

\subsection{Primitives de fonctions rationnelles}

    On considère $a$ et $b$ deux fonctions polynomiales, et 
    \[ f : x \longmapsto \frac{a(x)}{b(x)} \]

    \begin{theo}{Décomposition en éléments simples}{}
        Soient $a$ et $b$ deux fonctions polynomiales telles que $\deg(a) < \deg(b)$.

        On suppose qu’il existe des fonctions polynomiales $b_1, b_2, \ldots, b_l$ irréductibles n’ayant pas de racines communes telles que $b = b_1 b_2 \ldots b_l$.

        Alors il existe des fonctions polynomiales $p_1, p_2,\ldots,p_l$ telles que :
        \[ \frac{a}{b} = \frac{p_1}{b_1} + \frac{p_2}{b_2} + \ldots + \frac{p_l}{b_l} \]
        avec $\deg(b_i) = 1 \text{ ou } 2$ et $\deg(p_i) = \left\{ \begin{array}{cl}
            0 &\text{si } \deg(b_i) = 1 \\
            1 &\text{si } \deg(b_i) = 2
        \end{array} \right.$
    \end{theo}

    \begin{omed}{Méthode universelle}{myred}
        \begin{enumerate}
            \item Division euclidienne du numérateur par le dénominateur (si $\deg(a) \geq \deg(b)$). 
            
            On trouve deux polynômes $q$ et $r$ tels que $\deg(r) < \deg(b)$ et $a = bq + r$. Ainsi, 
            \[ f : x \mapsto q(x) + \frac{r(x)}{b(x)} \]
            \item Décomposition en éléments simples.
            \item On est alors ramené à des cas plus simples, que l’on sait primitiver.
        \end{enumerate}
    \end{omed}

    \subsubsection{Cas où le numérateur est constant}

    \begin{align}
        \frac{1}{x-a} &= (\ln|x-a|)' \\
        \frac{1}{(x-a)^n} &= \left( - \frac{1}{n-1} \frac{1}{(x-a)^{n-1}}  \right)' \\
        \frac{1}{x^2 + a^2} &= \left( \frac{1}{a} \arctan \left(\frac{x}{a}\right) \right)'
    \end{align}

    \begin{equation}
        \frac{1}{ax^2 + bx + c} \qquad \text{avec } \lilbox{myred}{$b^2 -4ac < 0$}
    \end{equation}
        
    \begin{omed}{Méthode}{myred}
        \begin{enumerate}[label=(\alph*)]
            \item Mettre le polynôme sous forme bicarrée : 
            \[ a \bigg( x+\overbrace{\frac{b}{2a}}^{\beta} \bigg)^2 + \overbrace{\frac{- \Delta}{4a}}^{\gamma} \]
            \item Factoriser par $\frac{1}{a}$ pour obtenir 
            \[ \frac{1}{a} \left( \frac{1}{(x+ \beta)^2 + \gamma / a} \right) \]
            \item Primitiver avec la formule (3), ce qui donne 
            \[ \frac{1}{a} \frac{1}{\sqrt{\gamma / a}} \arctan \left( \frac{x + \beta}{\sqrt{\gamma / a}} \right) \]
        \end{enumerate}
    \end{omed}

    \begin{equation}
        \frac{1}{(x-a)(x-b)} \qquad \text{ i.e. cas précédent avec } \lilbox{mypurple}{$\Delta > 0$}
    \end{equation}

    \begin{omed}{Méthode}{myred}
        \begin{enumerate}[label=(\alph*)]
            \item Décomposer en éléments simples afin d’obtenir \[ \frac{\alpha}{x-a} + \frac{\beta}{x-b} \]
            \item Primitiver en \[ \alpha \ln \left(|x-a|\right) + \beta \ln \left(|x-b|\right)\]
        \end{enumerate}
    \end{omed}

    \subsubsection{Cas où le numérateur n’est pas constant}

    \begin{align}
        \frac{\alpha x + \beta}{ x^2 + a^2} &= \left( \frac{\alpha}{2} \ln (x^2 + a^2) + \frac{\beta}{a} \arctan \left( \frac{x}{a} \right) \right)' \\
        \frac{\alpha x + \beta}{ax^2 + bx + c} &= \frac{\alpha}{2a} \frac{2ax + b}{ax^2 + bx + c} + \frac{\beta - \alpha b / 2a}{ax^2 + bx + c} 
    \end{align}

    Puis utiliser les méthodes précédentes pour primitiver.

\subsection{Intégration par parties}

    \begin{theo}{IPP}{}
        \begin{soit}
            \item $\intervalleFF{a}{b}$ un segment de $\mathbb{R}$,
            \item $u,v \in \mathcal{C}^1(\intervalleFF{a}{b},\mathbb{K})$.
        \end{soit}
        Alors 
        \[ \int_{a}^{b} u(t)v'(t)dt = [u(t)v(t)]_a^b - \int_{a}^{b} u'(t)v(t)dt \]
    \end{theo}

\subsection{Changement de variable}

    \begin{omed}{L’idée}{myred}
        Soient $u \in \mathcal{C}^1(\intervalleFF{a}{b},\mathbb{R})$ et $F \in \mathbb{C}^1(I, \mathbb{R})$ deux fonctions telles que $\forall x \in \intervalleFF{a}{b}, \, u(x) \in I$.

        Alors $F \circ u \in \mathcal{C}^1(\intervalleFF{a}{b},\mathbb{R})$ et $\forall t \in \intervalleFF{a}{b}, \, (F \circ u)'(t) = u'(t)f(u(t))$.

        D’où \[ \int_{a}^{b} u'(t)f(u(t))dt = [F \circ u(t)]^b_a \]
    \end{omed}

    \begin{theo}{Changement de variable}{}
        \begin{soient}
            \item $\intervalleFF{a}{b}$ un segment de $\mathbb{R}$
            \item $u \in \mathcal{F}(\intervalleFF{a}{b},\mathbb{R})$
            \item $f \in \mathcal{F}(I,\mathbb{R})$
        \end{soient}
        \begin{suppose}
            \item $u \in \mathcal{C}_1(\intervalleFF{a}{b},\mathbb{R})$
            \item $f$ est continue sur $I$
            \item $\forall t \in \intervalleFF{a}{b}, \, u(t) \in I$
        \end{suppose}
        Alors \[ \int_{a}^{b} u'(t)f(u(t))dt = \int_{u(a)}^{u(b)} f(x)dx \]
    \end{theo}

    \begin{omed}{Choix du changement de variable}{myred}
        \begin{enumerate}
            \item $\sqrt{1-x^2} \rightarrow$ utiliser $x = \cos(t)$ ou $x = \sin(t)$
            \item $\sqrt{1+x^2} \rightarrow$ utiliser $x = \sinh(t)$
            \item si $\sqrt{t}$ ou $e^t \rightarrow$ utiliser respectivement $x = \sqrt{t}$ ou $x = e^t$
            \item \textbf{Règles de Bioche} \quad Si $f$ est une fonction rationnelle en sin et cos, 
            \begin{itemize}
                \item si $f(-x) = f(x)$, on pose $x = \cos(t)$.
                \item si $f(\pi-x) = f(x)$, on pose $x = \sin(t)$
                \item si $f(\pi+x) = f(x)$, on pose $x = \tan(t)$
                \item sinon, on pose $x = \tan \left( \frac{t}{2} \right)$
            \end{itemize}
        \end{enumerate}
    \end{omed}


