\chapter{Calcul différentiel}
\chaptertoc

\section{Calcul différentiel}

\subsection{Premiers pas informels avec les fonctions de deux variables}

    On se restreint dans un premier temps aux fonctions numériques définies sur un ouvert $\mathcal{U}$ de $\mathbb{R}^2$. On considèrera donc 
    \[ \fonction{f}{\mathcal{U} \subset \mathbb{R}^2}{\mathbb{R}}{(x,y)}{f(x,y)} \]   

    \begin{defi}{Ligne de niveau}{}
        On appelle ligne de niveau de hauteur $k$ la courbe d’équation $f(x,y) = k$. 
    \end{defi}

    Les lignes de niveau sont les intersections de $\mathcal{S}$ (surface de la fonction) avec les plans $z = k$. Ceci permet de donner des représentations graphiques claires de ces fonctions.

    \subsubsection{Limites et continuité}

    \begin{defi}{Fonction continue}{}
        \begin{soient}
            \item $\mathcal{U}$ un ouvert de $\mathbb{R}^2$
            \item $f \in \mathcal{F}(\mathcal{U},\mathbb{R})$
            \item $(x_0,y_0) \in \mathcal{U}$
        \end{soient}
        On dit que $f$ est \textbf{continue} en $(x_0,y_0)$ lorsque 
        \[ \forall \varepsilon > 0,  \exists \eta > 0,  \forall(x,y) \in \mathcal{B}\left((x_0,y_0),\eta\right),  \abs{f(x,y)-f(x_0,y_0)} \leq \varepsilon \]
        On dit que $f$ est continue sur $\mathcal{U}$ si elle est continue en tout $(x_0,y_0) \in \mathcal{U}$.
    \end{defi}

    \begin{prop}{Théorèmes généraux pour les fonctions de deux variables}{}
        \begin{soient}
            \item $\mathcal{U}$ un ouvert de $\mathbb{R}^2$
            \item $(f,g) \in \mathcal{F}(\mathcal{U},\mathbb{R})^2$ et $\lambda,\mu \in \mathbb{R}^2$
            \item $(x_0,y_0) \in \mathcal{U}$
        \end{soient}
        On suppose que $f$ et $g$ sont continues en $(x_0,y_0)$

        \begin{alors}
            \item $\lambda f + \mu g$ est continue en $(x_0,y_0)$
            \item $fg$ est continue en $(x_0,y_0)$
            \item Si $g(x_0,y_0) \neq 0$, il existe $\alpha >0$  tel que $g$ ne s’annule pas sur $\mathcal{B}\left((x_0,y_0), \alpha\right)$ et le quotient $ f / g$ des restrictions de $f$ et $g$ à $\mathcal{B}\left((x_0,y_0), \alpha\right)$ est continu en $(x_0,y_0)$
        \end{alors}
        On peut étendre ces résultats à tout $\mathcal{U}$.
    \end{prop}

    \begin{defi}{Fonction polynomiale}{}
        \begin{itemize}
            \item Une fonction polynomiale en $(x,y)$ est une combinaison linéaire de fonctions du type $(x,y) \longmapsto x^p y^q$, où $(p,q) \in \mathbb{N}^2$
            \item Une fonction rationnelle en $(x,y)$ est un quotient de fonctions polynomiales en $(x,y)$
        \end{itemize}
    \end{defi}

    \begin{coro}{Continuité des fonctions polynomiales et rationnelles}{}
        \begin{enumerate}
            \item Les fonctions polynomiales sont continues sur $\mathbb{R}^2$
            \item Les fonctions rationnelles sont continues sur leur ensemble de définition.
        \end{enumerate}
    \end{coro}

    \begin{demo}{Preuve}{myorange}
        Les fonctions $(x,y) \mapsto x$, $(x,y) \mapsto y$ et $(x,y) \mapsto 1$ sont continues sur $\mathbb{R}^2$.
    \end{demo}

    \begin{prop}{Composée à gauche par une fonction de la variable réelle}{}
        \begin{soient}
            \item $\mathcal{U}$ un ouvert de $\mathbb{R}^2$ et $D$ une union finie d’intervalles non réduits à un point de $\mathbb{R}$
            \item $f \in \mathcal{F}(\mathcal{U},\mathbb{R})$ et $g \in \mathcal{F}(D,\mathbb{R})$
            \item $(x_0,y_0) \in \mathcal{U}$
        \end{soient}
        \begin{suppose}
            \item $f(\mathcal{U}) \subset D$
            \item $f$ est continue en $(x_0,y_0)$ (resp. sur $\mathcal{U}$)
            \item $g$ est continue en $f(x_0,y_0)$ (resp. sur $\mathcal{D}$)
        \end{suppose}
        Alors $g \circ f$ est continue en $(x_0,y_0)$ (resp. sur $\mathcal{U}$)
    \end{prop}

    \subsubsection{Dérivées partielles premières}

    \begin{defi}{Dérivée directionnelle}{}
        \begin{soient}
            \item $\mathcal{U}$ un ouvert de $\mathbb{R}^2$
            \item $f \in \mathcal{F}(\mathcal{U},\mathbb{R})$
            \item $(x_0,y_0) \in \mathcal{U}$
            \item $v = (a,b) \in \mathbb{R}^2 \backslash \{ (0,0) \}$
        \end{soient}
        On dit que $f$ est dérivable en $(x_0,y_0)$ dans la direction $v$ lorsque $g_{(x_0,y_0)} : t \mapsto f((x_0,y_0)+tv)$ est dérivable en 0. 
        
        Dans ce cas, la dérivée de $f$ selon $v$ en $(x_0,y_0)$ est 
        \[D_v f(x_0,y_0) = g_{(x_0,y_0)}(0) = \lim\limits_{t \rightarrow 0} \frac{f((x_0,y_0)+tv) - f((x_0,y_0))}{t} \]
    \end{defi}

    \begin{defi}{Dérivée partielle}{}
        \begin{soient}
            \item $\mathcal{U}$ un ouvert de $\mathbb{R}^2$
            \item $f \in \mathcal{F}(\mathcal{U},\mathbb{R})$
            \item $(x_0,y_0) \in \mathcal{U}$
        \end{soient}
        Lorsqu’elle existe, la \textbf{dérivée partielle} de $f$ par rapport à $x$ (resp. $y$) en $(x_0,y_0)$ est $D_{(1,0)}f(x_0,y_0)$ (resp. $D_{(0,1)}f(x_0,y_0)$), notée $\frac{\partial f}{\partial x}(x_0,y_0)$ (resp. $\frac{\partial f}{\partial y}(x_0,y_0)$)
    \end{defi}

    \begin{omed}{Remarques}{myyellow}
        \begin{itemize}
            \item $\frac{\partial f}{\partial x}$ n’est qu’une notation pour la dérivée par rapport à la première variable.
            \item L’existence des dérivées partielles n’entraîne pas la continuité de la fonction.
        \end{itemize}
    \end{omed}

    \begin{prop}{Théorèmes généraux pour les dérivées partielles}{}
        \begin{soient}
            \item $\mathcal{U}$ un ouvert de $\mathbb{R}^2$
            \item $f,g \in \mathcal{F}(\mathcal{U},\mathbb{R})$ et $\lambda,\mu \in \mathbb{R}$
            \item $(x_0,y_0) \in \mathcal{U}$
        \end{soient}
        On suppose que $f$ et $g$ admettent des dérivées partielles en $(x_0,y_0)$

        \begin{alors}
            \item La fonction $\lambda f + \mu g$ admet des dérivées partielles en $(x_0,y_0)$ et 
            \[ \frac{\partial \lambda f + \mu g}{\partial x}(x_0,y_0) = \lambda \frac{\partial f}{\partial x}(x_0,y_0) + \mu \frac{\partial g}{\partial x}(x_0,y_0) \] 
            \item La fonction $fg$ admet des dérivées partielles en $(x_0,y_0)$ et 
            \[ \frac{\partial fg}{\partial x}(x_0,y_0) = \frac{\partial f}{\partial x}(x_0,y_0)g(x_0,y_0) + \frac{\partial g}{\partial x}(x_0,y_0) f(x_0,y_0) \] 
            \item Si $g$ ne s’annule pas sur $\mathcal{U}$ (ou au moins sur une boule ouverte centrée sur $(x_0,y_0)$), alors $f / g$ admet des dérivées partielles en $(x_0,y_0)$ et 
            \[ \frac{\partial f/g}{\partial x}(x_0,y_0) = \frac{\frac{\partial f}{\partial x}(x_0,y_0)g(x_0,y_0) - \frac{\partial g}{\partial x}(x_0,y_0)f(x_0,y_0)}{g(x_0,y_0)^2} \]
        \end{alors}
        Les définitions sont similaires pour les dérivées partielles par rapport à la seconde variable.
    \end{prop}

    \begin{prop}{Composition à gauche par une fonction de la variable réelle}{}
        \begin{soient}
            \item $\mathcal{U}$ un ouvert de $\mathbb{R}^2$ et $D$ une union finie d’intervalles non réduits à un point de $\mathbb{R}$
            \item $f\in \mathcal{F}(\mathcal{U},\mathbb{R})$ et $ g \in \mathcal{F}(D,\mathbb{R})$
            \item $(x_0,y_0) \in \mathcal{U}$
        \end{soient}
        \begin{suppose}
            \item $f(\mathcal{U}) \subset D$
            \item $f$ admet des dérivées partielles en $(x_0,y_0)$
            \item $g$ est dérivable en $f(x_0,y_0)$
        \end{suppose}
        Alors $g \circ f$ admet des dérivées partielles en $(x_0,y_0)$, et 
        \[ \frac{\partial (g \circ f)}{\partial x}(x_0,y_0) = \frac{\partial f}{\partial x}(x_0,y_0) g'(f(x_0,y_0)) \] 
        La définition est similaire pour la dérivée partielle par rapport à la seconde variable.
    \end{prop}

    \begin{defi}{Fonctions de classe $\mathcal{C}^1$}{}
        \begin{soient}
            \item $\mathcal{U}$ un ouvert de $\mathbb{R}^2$
            \item $f\in \mathcal{F}(\mathcal{U},\mathbb{R})$
            \item $(x_0,y_0) \in \mathcal{U}$
        \end{soient}
        On dit que \textbf{$f$ est de classe $\mathcal{C}^1$} sur $\mathcal{U}$ lorsque 
        \begin{enumerate}
            \item $f$ admets des dérivées partielles en tout $(x_0,y_0) \in \mathcal{U}$
            \item $\frac{\partial f}{\partial x}$ et $\frac{\partial f}{\partial y}$ sont continues.
        \end{enumerate}
        On note $\mathcal{C}^1(\mathcal{U},\mathbb{R})$ l’ensemble des fonctions de classe $\mathcal{C}^1$ sur $\mathcal{U}$ à valeurs dans $\mathbb{R}$.
    \end{defi}

    \begin{defi}{Gradient}{}
        \begin{soient}
            \item $\mathcal{U}$ un ouvert de $\mathbb{R}^2$
            \item $f\in \mathcal{C}^1(\mathcal{U},\mathbb{R})$
            \item $(x_0,y_0) \in \mathcal{U}$
        \end{soient}
        Le \textbf{gradient} de $f$ en $(x_0,y_0)$ est 
        \[ \nabla f(x_0,y_0) = \left(\frac{\partial f}{\partial x}(x_0,y_0), \frac{\partial f}{\partial y}(x_0,y_0)\right) \]
    \end{defi}

    \begin{prop}{Théorèmes généraux pour les fonctions de deux variables}{}
        \begin{soient}
            \item $\mathcal{U}$ un ouvert de $\mathbb{R}^2$
            \item $f,g \in \mathcal{C}^1(\mathcal{U},\mathbb{R})$ et $\lambda,\mu \in \mathbb{R}^2$
        \end{soient}
        \begin{alors}
            \item $\lambda f + \mu g$ est de classe $\mathcal{C}^1$ sur $\mathcal{U}$ et
            \[ \forall (x_0,y_0) \in \mathcal{U}, \nabla (\lambda f + \mu g)(x_0,y_0) = \lambda \nabla f(x_0,y_0) + \mu \nabla g(x_0,y_0) \]
            \item $fg$ est de classe $\mathcal{C}^1$ sur $\mathcal{U}$ et
            \[ \forall (x_0,y_0) \in \mathcal{U}, \nabla (fg)(x_0,y_0) = g(x_0,y_0) \nabla f(x_0,y_0) + f(x_0,y_0) \nabla g(x_0,y_0) \]
            \item Si $\forall (x,y) \in \mathcal{U}, g(x,y) \neq 0$, alors $f / g$ est de classe $\mathcal{C}^1$ sur $\mathcal{U}$ et
            \[ \forall (x_0,y_0) \in \mathcal{U}, \nabla \left(\frac{f}{g}\right)(x_0,y_0) = \frac{g(x_0,y_0) \nabla f(x_0,y_0) - f(x_0,y_0) \nabla g(x_0,y_0)}{g(x_0,y_0)^2} \]
        \end{alors}
    \end{prop}

    \begin{coro}{Classe $\mathcal{C}^1$ des fonctions polynomiales et rationnelles}{}
        \begin{enumerate}
            \item Les fonctions polynomiales sont de classe $\mathcal{C}^1$ sur $\mathbb{R}^2$
            \item Les fonctions rationnelles sont de classe $\mathcal{C}^1$ sur leur ensemble de définition.
        \end{enumerate}
    \end{coro}

    \begin{prop}{Composition}{}
        \begin{soient}
            \item $ \mathcal{U}$ un ouvert de $\mathbb{R}^2$ et $D$ une union finie d’intervalles non réduits à un point de $\mathbb{R}$
            \item $f \in \mathcal{F}(\mathcal{U},\mathbb{R})$ et $g \in \mathcal{F}(D,\mathbb{R})$
        \end{soient}
        \begin{suppose}
            \item $f(\mathcal{U}) \subset D$
            \item $f \in \mathcal{C}^1(\mathcal{U},\mathbb{R})$
            \item $g \in \mathcal{C}^1(D,\mathbb{R})$
        \end{suppose}
        Alors $g \circ f \in \mathcal{C}^1(\mathcal{U},\mathbb{R})$ et 
        \[ \nabla (g \circ f)(x_0,y_0) = g'(f(x_0,y_0)) \nabla f(x_0,y_0) \] 
    \end{prop}

    \begin{theo}{Formule de TY à l’ordre 1}{}
        \begin{soient}
            \item $\mathcal{U}$ un ouvert de $\mathbb{R}^2$
            \item $f\in \mathcal{C}^1(\mathcal{U},\mathbb{R})$
            \item $(x_0,y_0) \in \mathcal{U}$
        \end{soient}
        Alors \[  f(x,y) \underset{\norm{(x,y)-(x_0,y_0)} \rightarrow 0}{=} f(x_0,y_0) + \spr{\nabla f(x_0,y_0)}{(x-x_0,y-y_0)} + o\left(\norm{(x-x_0,y-y_0)}\right) \]
        De manière équivalente, on peut écrire ce DL
        \[ f(x_0 + h,y_0 + k) \underset{\norm{(h,k)} \rightarrow 0}{=} f(x_0,y_0) + \frac{\partial f}{\partial x}(x_0,y_0) h + \frac{\partial f}{\partial y}(x_0,y_0)k + o\left(\norm{(h,k)}\right) \]
    \end{theo}

    La surface d’équation $z = f(x,y)$ admet alors un plan tangent en tout point $(x_0,y_0,z_0)$ tel que $z_0 = f(x_0, y_0)$. Celui-ci a pour équation 
    \[ z = f(x_0, y_0) + \dpart{f}{x}(x_0,y_0)(x-x_0) + \dpart{f}{y}(x_0,y_0)(y - y_0) \]   
    C’est une équation de la forme $a(x-x_0) + b(x - x_0) + c(x-x_0)$ avec $a,b$ les dérivées partielles et $c = -1$. Ainsi, le vecteur $\left(\dpart{f}{x}(x_0,y_0), \dpart{f}{y}(x_0,y_0) , -1\right)$ est normal au plan tangent. 

    Si $f$ est de classe $\mathcal{C}^1$, l’application 
    \[ (h,k) \mapsto \dpart{f}{x}(x_0,y_0) h + \dpart{f}{y}(x_0,y_0)k \]   est linéaire, on l’appelle différentielle de $f$ en $(x_0, y_0)$ et on la note $df_{(x_0,y_0)}$. Ici, $df_{(x_0,y_0)} \in \mathcal{L}(\mathbb{R}^2, \mathbb{R})$. 

    Le gradient de $f$ en $(x_0, y_0)$ est ainsi l’unique vecteur vérifiant, pour tout $u = (h,k)$, 
    \[ df_{(x_0,y_0)}(u) = \spr{\nabla f(x_0,y_0)}{u} \]   

    \begin{coro}{}{}
        \begin{soient}
            \item $\mathcal{U}$ un ouvert de $\mathbb{R}^2$
            \item $f\in \mathcal{F}(\mathcal{U},\mathbb{R})$
        \end{soient}
        On suppose que $f$ est de classe $\mathcal{C}^1$ sur $\mathcal{U}$.

        Alors $f$ est continue sur $\mathcal{U}$.
    \end{coro}

    \begin{coro}{}{}
        \begin{soient}
            \item $\mathcal{U}$ un ouvert de $\mathbb{R}^2$
            \item $f\in \mathcal{C}^1(\mathcal{U},\mathbb{R})$
            \item $(a,b) \in \mathbb{R}^2 \backslash \{ (0,0) \}$
        \end{soient}

        Alors $f$ admet une dérivée suivant $v = (a,b)$ en tout $(x_0,y_0) \in \mathcal{U}$ et 
        \[ D_v f(x_0,y_0) = \spr{\nabla f(x_0,y_0)}{v} \]
    \end{coro}

    \subsubsection{Règles de la chaîne}

    \begin{prop}{Règle de la chaîne}{}
        \begin{soient}
            \item $D$ une union finie d’intervalles de $\mathbb{R}$ non réduits à un point
            \item $x$ et $y$ deux fonctions définies sur $D$ à valeurs dans $\mathbb{R}$
            \item $\mathcal{U}$ un ouvert de $\mathbb{R}^2$
            \item $f \in \mathcal{F}(\mathcal{U},\mathbb{R})$
        \end{soient}
        \begin{suppose}
            \item $\forall t \in D, (x(t),y(t)) \in \mathcal{U}$
            \item $x,y \in \mathcal{C}^1(D,\mathbb{R})$
            \item $f \in \mathcal{C}^1(\mathcal{U},\mathbb{R})$
        \end{suppose}
        Alors $g : t \mapsto f(x(t),y(t)) \in \mathcal{C}^1(D,\mathbb{R})$ et
        \begin{align*}
            \forall t \in D, g'(t) &= \spr{\nabla f(x(t),y(t))}{(x'(t),y'(t))} \\
            &= \frac{\partial f}{\partial x}(x(t),y(t)) x'(t) + \frac{\partial f}{\partial y}(x(t),y(t)) y'(t)
        \end{align*} 
    \end{prop}

    \begin{omed}{Remarque \textcolor{black}{(Le gradient est orthogonal aux lignes de niveau)}}{myolive}
        Si on définit, pour $\lambda \in \mathbb{R}$, 
        \[ C_{\lambda} = \left\{ (x,y) \in \mathbb{R}^2, f(x,y) = \lambda  \right\}\] 
        une courbe de niveau de $f$, qui est paramétrée par une courbe paramétrée $\gamma = (x,y)$.

        Alors $f \circ \gamma = \lambda$ pour tous $t$, donc par la règle de la chaîne, 
        \[ \spr{\nabla f(\gamma(t))}{\gamma'(t)} = 0 \] 
        et donc le gradient est orthogonal au vecteur tangent.
    \end{omed}

    \begin{prop}{Règle de la chaîne, deuxième version}{}
        \begin{soient}
            \item $\mathcal{U}$ et $\mathcal{U}'$ deux ouverts de $\mathbb{R}^2$
            \item $x$ et $y$ deux fonctions définies sur $\mathcal{U}$, à valeurs dans $\mathbb{R}$
            \item $f \in \mathcal{F}(\mathcal{U}',\mathbb{R})$
        \end{soient}
        \begin{suppose}
            \item $\forall (u,v) \in \mathcal{U}, \left(x(u,v),y(u,v)\right) \in \mathcal{U}'$
            \item $x,y \in \mathcal{C}^1 (\mathcal{U},\mathbb{R})$
            \item $ f \in \mathcal{C}^1 (\mathcal{U}',\mathbb{R})$
        \end{suppose}
        Alors $\fonction{h}{\mathcal{U}}{\mathbb{R}}{(u,v)}{f(x(u,v),y(u,v))} \in \mathcal{C}^1(\mathcal{U},\mathbb{R})$ et 
        \begin{align*}
            \forall (u,v) \in \mathcal{U}, \frac{\partial h}{\partial u}(u,v) &= \frac{\partial f}{\partial x}(x(u,v),y(u,v))\frac{\partial x}{\partial u}(u,v) \\ &+ \frac{\partial f}{\partial y}(x(u,v),y(u,v))\frac{\partial y}{\partial u}(u,v)
        \end{align*}
    \end{prop}

    \begin{omed}{Exemple \textcolor{black}{(Coordonnées polaires)}}{myolive}
        Il est courant de passer aux coordonnées polaires, i.e de considérer la fonction 
        \[ F : (r,\theta) \mapsto f(\underbrace{r \cos \theta}_x,\underbrace{ r \sin \theta}_y) \]
        On a alors 
        \begin{align*}
            r \frac{\partial F}{\partial r}(r,\theta) &= \phantom{-}x \frac{\partial f}{\partial x}(x,y) + y \frac{\partial f}{\partial y}(x,y) \\
            \frac{\partial F}{\partial \theta}(r,\theta) &= -y \frac{\partial f}{\partial x}(x,y) + x \frac{\partial f}{\partial y}(x,y) \\
        \end{align*}
    \end{omed}

    \subsubsection{Dérivées partielles secondes}

    On étudie désormais les dérivées partielles secondes, qui sont au nombre de 4 : $\frac{\partial^2 f}{\partial x^2}$, $\frac{\partial^2 f}{\partial y^2}$, $\frac{\partial^2 f}{\partial x \partial y}$ et $\frac{\partial^2 f}{\partial y \partial x}$.

    \begin{defi}{}{}
        On dit que $f$ est de classe $\mathcal{C}^2$ sur un ouvert $\mathcal{U}$ de $\mathbb{R}^2$ si les dérivées partielles secondes existent et sont continues en tout point de $\mathcal{U}$. 
    \end{defi}

    \begin{theo}{Théoème de Schwarz}{}
        Soit $f : \mathcal{U} \to \mathbb{R}^2$ une application de classe $\mathcal{C}^2$ sur un ouvert $\mathcal{U}$ de $\mathbb{R}^2$. Alors 
        \[ \forall (x_0,y_0) \in \mathcal{U}, \quad \frac{\partial^2 f}{\partial y \partial x}(x_0,y_0) = \frac{\partial^2 f}{\partial x \partial y}(x_0,y_0) \]   
    \end{theo}

    \begin{theo}{Formule de Taylor-Young à l’ordre 2}{}
        Si $f$ est de classe $\mathcal{C}^2$ sur un ouvert $\mathcal{U}$ de $\mathbb{R}^2$, alors pour tout $(x_0,y_0) \in \mathcal{U}$, 
        \begin{multline*}
            f(x_0 + h, y_0 + k) = f(x_0, y_0) + \dpart{f}{x}(x_0,y_0) \cdot h + \dpart{f}{y}(x_0,y_0) \cdot k \\
            + \frac{1}{2} \left[\frac{\partial^2 f}{\partial x^2}(x_0,y_0) \cdots h^2 + 2 \frac{\partial^2 f}{\partial x \partial y}(x_0,y_0) h k + \frac{\partial^2 f}{\partial y^2}(x_0,y_0) k^2 \right] + o(h^2 + k^2)
        \end{multline*}
    \end{theo}

    Cette dernière formule rendra de précieux services dans l’étude des extrema d’une fonction numérique.

    \subsubsection{Exemple de résolution d’équations aux dérivées partielles} 

    Nous n’aborderons la résolution d’équations aux dérivées partielles (EDP) qu’à travers des exemples relativement simples et, à ce titre, très restreints. $\mathcal{U}$ désignera ici l’ouvert $I \times J$ où $I$ et $J$ sont deux intervalles ouverts et $f$ solution des EDP proposées. Nous justifierons dans la partie suivante le choix d’un tel domaine.
    \begin{itemize}
        \item Si $\forall (x,y) \in \mathcal{U}$, $\dpart{f}{x}(x,y) = 0$, alors $\forall (x,y) \in \mathcal{U}$,  $f(x,y) = \varphi(y)$ avec $\varphi : J \to \mathbb{R}$. 
        \item Si $\forall (x,y) \in \mathcal{U}$, $\dpart{f}{x}(x,y) = g(x,y)$, alors $\forall (x,y) \in \mathcal{U}$,  $f(x,y) = \int g(x,y) dx +  \varphi(y)$ avec $\varphi : J \to \mathbb{R}$.
        \item Si $\forall (x,y) \in \mathcal{U}$, $\frac{\partial^2 f}{\partial x^2}(x,y) = 0$, alors $\forall (x,y) \in \mathcal{U}$,  $f(x,y) = \varphi(y)x + \psi(y)$ avec $\varphi, \psi : J \to \mathbb{R}$.
        \item Si on est face à une équation de d’Alembert, de la forme 
        \[ \forall (x,t) \in \mathbb{R}^2, \quad \frac{\partial^2 f}{\partial x^2} (x,t) = \frac{1}{c^2} \frac{\partial^2 f}{\partial t^2}(x,t) \qquad (c \neq 0) \]   
        On passe par le changement de variables de classe $\mathcal{C}^2$ et bijectif (car $c \neq 0$) :  $\et{u = x-ct}{v = x + ct}$. En partant de $f(x,t) = g(u,v)$, on cherche une EDP vérifiée par $g$.
        \begin{align*}
            \dpart{f}{x} &= \dpart{g}{u} \dpart{u}{x} + \dpart{g}{v} \dpart{v}{x} = \dpart{g}{u} + \dpart{g}{v} \\
            \frac{\partial^2 f}{\partial x^2} &= \left(\frac{\partial^2 g}{\partial^2 u} \dpart{u}{x} + \frac{\partial^2 g}{\partial v \partial u} \dpart{v}{x}\right) + \left(\frac{\partial^2 g}{\partial^2 v} \dpart{u}{x} + \frac{\partial^2 g}{\partial u \partial v} \dpart{v}{x}\right) \\
            &= \frac{\partial^2 g}{\partial^2 u} + 2 \frac{\partial^2 g}{\partial u \partial v} + \frac{\partial^2 g}{\partial^2 v}
        \end{align*}
        De même, on trouve 
        \[ \frac{\partial^2 f}{\partial t^2} = c^2 \left(\frac{\partial^2 g}{\partial^2 u} - 2 \frac{\partial^2 g}{\partial u \partial v} + \frac{\partial^2 g}{\partial^2 v}\right) \]   
        L’équation devient donc, $\forall (u,v) \in \mathbb{R}^2$, 
        $\frac{\partial^2 g}{\partial u \partial v} = 0$, et donc 
        \[ f(x,t) = \varphi(x-ct) + \psi(x + ct) \]   
        
    \end{itemize}

\subsection{Applications différentiables et applications de classe C1}

    On considère désormais $f : \mathcal{U} \subset E \to F$ où $E$ et $F$ sont 2 EVN sur $\mathbb{R}$ de dimensions respectives $p$ et $n$ et $\mathcal{U}$ un ouvert de $E$. Toutes les normes étant équivalentes en dimension finie, le choix d’une norme spécifique pour $E$ ou $F$ importera peu. 

    On parle de champ de vecteurs lorsque $f : E \to E$ et de champ scalaire lorsque $f : E \to \mathbb{R}$.

    On travaillera le plus souvent dans les espaces $E = \mathbb{R}^p$ et $F = \mathbb{R}^n$, munis de leur structure euclidienne canonique. Ainsi, on pourra noter 
    \[ f(x) = f(x_1,\ldots,x_p) = (f_1(x), \ldots, f_n(x)) \]   
    OA alors, en identifiant les grandeurs dans des bases de $E$ et $F$, de façon à ce que 
    \[ x = \sum_{j=1}^{p} x_j e_j \esp{et} f(x) = \sum_{i=1}^{n} f_i(x) e'_i \]   
     \begin{itemize}
        \item applications partielles de $f$ les applications $x_j \mapsto f(x)$ pour $j \in \intervalleEntier{1}{p}$.
        \item On appelle applications composantes de $f$ les applications $f_i : x \mapsto f_i(x)$ pour $i \in \intervalleEntier{1}{n}$. 
    \end{itemize}
    
    \subsubsection{Différentielle}

    \begin{defitheo}{Différentielle en un point}{}
        $f$ est dite différentiable en $a \in \mathcal{U}$ s’il existe $\varphi \in \mathcal{L}(E,F)$ telle que :
        \[ f(a+h) = f(a) + \varphi(h) + \comp{o}{h}{0}{\norm{h}} \]   
        Si l’application $\varphi$ existe, elle est unique et appelée \textbf{différentielle} de $f$ au point $a$, ou \textbf{application linéaire tangente} à $f$ en $a$. On la note $\d f_a$ ou $\d f(a)$. 
    \end{defitheo}

    L’existence d’un DL à l’ordre 1 relève ainsi de la définition même de la différentiabilité : 
    \[ f(a+ h) = f(a) + \left[\d f(a)\right](h) + \comp{o}{h}{0}{\norm{h}} \]   
    ON $\d f_a(h) = \left[\d f(a)\right](h) = \d f(a) \times h$. Cette dernière notation fais sens lorsque l’on raisonne en termes matriciels : c’est le produit d’une matrice $n \times p$ par un vecteur $p$. 

    \begin{demo}{Preuve}{mypurple}
        Sous réserve d’existence, dans le cas où $\varphi$ et $\psi$ vérifient l’égalité, alors $\Psi = \varphi - \psi$ est linéaire et vérifie $\Psi(h) = \comp{o}{h}{0}{\norm{h}}$, d’où la nullité de $\Psi$. On a, en effet, pour $x \in E$ non nul et $t \in \mathbb{K}^*$, $\norm{\Psi(x)} = \norm{x} \cdot \frac{\norm{\Psi(tx)}}{\norm{tx}} \limi{t}{0} 0$.
    \end{demo}

    \begin{prop}{}{}
        Si $f$ est différentiable en $a \in \mathcal{U}$, alors $f$ est continue en $a$. 
    \end{prop}

    \begin{demo}{Preuve}{myolive}
        $f(a+h) - f(a) = \d f_a(h) + \comp{o}{h}{0}{\norm{h}} \limi{h}{0_E} \d f_a(0_E) = 0_F$ par continuité de $\d f_a$ en tant qu’AL sur un EVN de DF. 
    \end{demo}

    Pour justifier quelques règles de calcul sur les différentielles en un point, nous auront, par exemple, besoin d’observer que $\d f_a(\comp{o}{h}{0}{\norm{h}}) = \comp{o}{h}{0}{\norm{h}}$ ou bien que $\comp{o}{h}{0}{\norm{\d f_a(h)}} = \comp{o}{h}{0}{\norm{h}}$. Montrons plus généralement que ceci est vrai pour toute application linéaire $\varphi$. En dimension finie, 
    \begin{enumerate}
        \item $\varphi$ est continue en 0 donc $\varphi(\comp{o}{h}{0}{\norm{h}}) = h \varphi(\varepsilon(h)) = \comp{o}{h}{0}{\norm{h}}$ car $\varphi(\varepsilon(h)) \limi{h}{0} \varphi(0) = 0$. 
        \item $\varphi$ est linéaire donc lipchitzienne. Il existe $C > 0$ tel que pour tout $h \in E$, $\norm{\varphi(h)} \leq C \norm{h}$ d’où la seconde relation. 
    \end{enumerate}

    \begin{prop}{Opérations algébriques sur les différentielles}{}
        \begin{enumerate}
            \item Si $f,g : \mathcal{U} \subset E \to F$ sont différentiables en $a$, alors pour $\lambda \in \mathbb{R}$, $\lambda f + g$ est différentiable en $a$ et 
            \[ \d (\lambda f + g)_a = \lambda \d f_a + \d g_a \]   
            \item Si $g : \mathcal{V} \subset F \to G$ avec $f(\mathcal{U}) \subset \mathcal{V}$, alors si $f$ est différentiable en $a$ et $g$ en $f(a)$, alors $g \circ f$ est différentiable en $a$ et 
            \[ \d (g \circ f)_a = \d g_{f(a)} \circ \d f_a \]  
        \end{enumerate}
    \end{prop}

    \begin{demo}{Preuve}{myolive}
        Tout repose sur les relations écrites précédemment. Prenons le deuxième cas pour exemple : 
        \begin{align*}
            (g \circ f)(a + h) 
            &= g(f(a) + \underbrace{\d f_a(h) + \comp{o}{h}{0}{\norm{h}}}_{=k}) \\
            &= g(f(a)) + \d g_{f(a)}(k) + \comp{o}{h}{0}{k} \\
            &= g(f(a)) + \left[\d g_{f(a)} \circ \d f_a\right](h) + \comp{o}{h}{0}{\norm{h}}
        \end{align*}
        L’unicité de la différentielle garanti le résultat.
    \end{demo}

    On peut énoncer une propriété plus générale, bien que largement moins intuitive.

    \begin{prop}{}{}
        Si $M : F^n \to \mathbb{R}$ est une application multilinéaire et si $f_1,\ldots,f_n : E \to F$ sont différentiables en $a \in E$, alors $M(f_1,\ldots,f_n)$ est différentiable en $a$ et 
        \[ \d (M(f_1,\ldots,f_n))_a = \sum_{k=1}^{n} M(f_1(a), \ldots, f_{k-1}(a), \d(f_k)_a, f_{k+1}(a), \ldots, f_n(a)) \]   
    \end{prop}

    La définition de différentielle d’une application découle de la notion de différentielle en un point : 

    \begin{defi}{Différentielle et application de classe $\mathcal{C}^1$}{}
        \begin{itemize}
            \item L’application $f : \mathcal{U} \subset E \to F$ est dite différentiable sur $\mathcal{U}$ si $f$ est différentiable en tout point de $\mathcal{U}$. On appelle alors \textbf{différentielle} de $f$ l’application 
            \[ \fonction{\d f}{\mathcal{U}}{\mathcal{L}(E,F)}{a}{\d f_a} \]
            \item L’application $f$ est dite de classe $\mathcal{C}^1$ sur $\mathcal{U}$, ou continûment différentiable sur $\mathcal{U}$ si elle est différentiable sur $\mathcal{U}$ et $\d f$ est continue sur $\mathcal{U}$. 
        \end{itemize}
    \end{defi}

    \subsubsection{Dérivées selon un vecteur et dérivées partielles}

    On considère toujours $f$ une fonction de $\mathcal{U} \subset E \to F$.

    \begin{defi}{Dérivée selon un vecteur}{}
        Si la fonction de la variable réelle $t \mapsto f(a + tu)$ avec $u \in E$ est dérivable en 0, alors on dit que $f$ est dérivable en $a$ selon le vecteur $u$ et on pose 
        \[ D_u(f)(a) = \lim_{t \to 0} \frac{f(a + tu) - f(a)}{t} \]   
    \end{defi}

    Quand une fonction est différentiable, elle est dérivable dans toutes les directions :

    \begin{prop}{}{}
        Si $f$ est différentiable en $a$, alors $f$ est dérivable en $a$ selon $u$ pour tout vecteur $u \in E$  et $D_u(f)(a) = \d f_a(u)$. 
    \end{prop}

    Si $E$ est muni d’une base $(e_1,\ldots,e_p)$, on peut définir les dérivées partielles de $f$ qui sont les dérivées selon des directions privilégiées.

    \begin{defi}{Dérivées partielles}{}
        Pour $j \in \intervalleEntier{1}{p}$, on appelle, sous réserve d’existence, dérivée partielle en $a$ d’indice $j$ la dérivée de $f$ en $a$ suivant $e_j$, \textit{i.e.} : 
        \[ \dpart{f}{x_j}(a) = \lim_{t \to 0} \frac{f(a + t e_j) - f(a)}{t} \]    
    \end{defi}

    Si $f$ est différentiable en $a$, alors ces dérivées partielles existent et 
    \[ \forall j \in \intervalleEntier{1}{p}, \qquad \dpart{f}{x_j}(a) = \d f_a(e_j) \]    

    Remarquons alors que par linéarité de $\d f_a$, 
    \[ \d f_a(h) = \d f_a\left(\sum_{j=1}^{p} h_j e_j\right) = \sum_{i=1}^{p} h_j \d f_a(e_j) = \sum_{j=1}^{p} h_j \dpart{f}{x_j}(a) \]    

    En notant $\d x_j$ les applications $h \mapsto h_j$, on retrouve une forme bien connue des physiciens : 
    \[ \d f_a = \sum_{j=1}^{p} \dpart{f}{x_j}(a) \d x_j \]   
    
    Ainsi, lorsque $f$ est différentiable sur $\mathcal{U}$ et $a \in \mathcal{U}$,
    \[ f(a + h) = f(a) + \d f_a(h) + \comp{o}{h}{0}{\norm{h}} = f(a) + \sum_{j=1}^{p} h_j \dpart{f}{x_j}(a) + \comp{o}{h}{0}{\norm{h}} \]    

    \begin{theo}{Caractérisation des applications de classe $\mathcal{C}^1$}{}
        $f$ est de classe $\mathcal{C}^1$ sur $\mathcal{U}$ \textit{ssi} les dérivées partielles de $f$ existent et sont continues en tout point de $\mathcal{U}$.
    \end{theo}

    Ce résultat est essentiel : il permet de justifier facilement la différentiabilité d’une fonction, à la seule condition d’existence et de continuité des dérivées partielles (ce qui confère en plus la continiuté de la différentielle). 

    \begin{demo}{Démonstration}{myred}
        On note $\norm{.}_F$ une norme sur $F$. L’équivalence des normes en dimension finie nous permet de choisir une norme particulière sur $E$ : celle définie par $\norm{h} = \sum_{j=1}^{p} \norm{h_j}$ où $j = \sum h_j e_j$. On munit $\mathcal{L}(E,F)$ de la norme subordonnée $\norm{.}_{\text{op}}$ relative à $\norm{.}$ et $\norm{.}_F$.
        \begin{itemize}
            \item[\textcolor{myred}{$\implies$}] L’existence des dérivées partielles a déjà été établie, et il reste à prouver leur continuité. Si $a \in \mathcal{U}$, QQS $x \in \mathcal{U}$ et $j \in \intervalleEntier{1}{p}$, 
            \[ \norm{\dpart{f}{x_j}(x) - \dpart{f}{x_j}(a)}_F = \norm{\d f_x(e_j) - \d f_a(e_j)}_F \leq \norm{\d f_x - \d f_a}_{op} \cdots \norm{e_j} \limi{x}{a} 0 \]   
            \item[\textcolor{red}{$\impliedby$}] Il faut prouver que $f$ est différentiable en tout point $a \in \mathcal{U}$ et que cette différentielle est continue. 
            
            Soit $a \in \mathcal{U}$ et $h = \sum h_j e_j$ tel que $a + h \in \mathcal{U}$. Par dérivabilité de $f$ suivant $e_p$, 
            \[ f(a + h) = f \left(a + \sum_{j = 1}^p h_j e_j\right) = f \left(a + \sum_{j = 1}^{p-1} h_j e_j\right) + h_p \dpart{f}{x_p}\left(a + \sum_{j = 1}^p h_j e_j\right) + \comp{o}{h}{0}{\norm{h}} \]   
            Par continuité de $\dpart{f}{x_p}$ en $a$, on obtient 
            \[ f(a+h) = f\left(a + \sum_{j = 1}^p h_j e_j\right) + h_p \dpart{f}{x_p}(a) + \comp{o}{h}{0}{\norm{h}} \]   
            De proche en proche, on obtient que 
            \[ f(a + h) = f(a) + \sum_{j=1}^{p} h_j \dpart{f}{x_j} (a) + \comp{o}{h}{0}{\norm{h}} \]   
            Ainsi, $f$ est bien différentiable en $a$, et $\d f_a(h) = \sum_{j=1}^{p} h_j \dpart{f}{x_j}(a)$. Pour la continuité, si $x \in \mathcal{U}$, QQS $h \in E$, 
            \[ \norm{\d f_x(h) - \d f_a(h)}_F \leq \sum_{j=1}^{p}\norm{h_j}\cdot \norm{\dpart{f}{x_j}(x) - \dpart{f}{x_j}(a)}_F \leq \norm{h} \cdot \max_{1\leq j \leq p} \norm{\dpart{f}{x_j}(x) - \dpart{f}{x_j}(a)}_F \]   
            Ainsi, $\norm{\d f_x - \d f_a}_{\text{op}} \leq \max_{1\leq j \leq p} \norm{\dpart{f}{x_j}(x) - \dpart{f}{x_j}(a)}_F \limi{x}{a} 0$ par continuité des $\dpart{f}{x_j}$.
        \end{itemize}
    \end{demo}

    \subsubsection{Représentation matricielle de la différentielle en un point}

    On munit à nouveau nos espaces $E$ et $F$ des bases $(e_j)_{1 \leq j \leq p}$ et $(e_i')_{1 \leq i \leq n}$. Pour toute fonction $f : E \to F$, on peut alors écrire
    \[ f(x) = f(x_1,\ldots,x_p) = (f_1(x), \ldots, f_n(x)) \] 
    
    On suppose que $f$ est de classe $\mathcal{C}^1$. Les fonctions $\dpart{f_i}{x_j}$ sont alors définies et continues pour tout $(i,j) \in \intervalleEntier{1}{n} \times \intervalleEntier{1}{p}$. Ce sont des fonctions définies sur $\mathbb{R}^n$ à valeurs dans $\mathbb{R}$. 

    \begin{defitheo}{Jacobienne}{}
        La matrice représentative de $\d f_x$ dans les bases $(e_j)_j$ et $(e_i)_i$ est $\left(\dpart{f_i}{x_j}(x)\right)$. Elle est appelée \textbf{matrice jacobienne} de $f$ au point $x = (x_1,\ldots,x_p)$. ON $J_f(x)$. 
    \end{defitheo}

    Ainsi, pour toute fonction $f \in \mathcal{C}^1(E,F)$, 
    \[ J_f(x) = \begin{bmatrix}
        \dpart{f_1}{x_1}(x) & \cdots & \dpart{f_1}{x_p}(x) \\
        \vdots & & \vdots \\
        \dpart{f_n}{x_1}(x) & \cdots & \dpart{f_n}{x_p}(x)
    \end{bmatrix} \qquad \in \mk{n,p} \]   

    Soient $f : E \to F$ et $g : F \to G$ de classe $\mathcal{C}^1$ sur leurs ensembles respectifs. Alors $g \circ f$ est de classe $\mathcal{C}^1$ sur $E$ et 
    \[ \forall x \in E, \quad J_{g \circ f}(f(x)) \times J_f(x) \esp{puisque} \d (g \circ f)_x = \d g_{f(x)} \circ \d f_x \]   

    Ce produit matriciel nous permet de retrouver la règle de la chaîne, qui s’écrit plus clairement :

    \begin{theo}{Règle de la chaîne}{}
        Soit $f : \mathcal{U} \subset \mathbb{R}^p \to \mathbb{R}$, de classe $\mathcal{C}^1$ et $x : I \subset \mathbb{R} \to \mathbb{R}^p$ de classe $\mathcal{C}^1$ telle que $x(I) \subset \mathcal{U}$. QQS $t_0 \in I$, 
        \[ \frac{\d}{\text{dt}}\left(t \mapsto f(x(t))\right) = \spr{\nabla f (x(t_0))}{x'(t_0)} = \sum_{i=1}^{p} \dpart{f}{x_i}(x(t_0)) x_i'(t_0) \]   
        Où $x(t) = (x_1(t), \ldots, x_p(t))$.
    \end{theo}

    \begin{demo}{Preuve}{myred}
        OP $a = x(t_0) \in \mathcal{U}$. $f$ possède un $DL_1$ en $a$, et 
        \[ f(a+h) = f(a) + \spr{\nabla f (a)}{h} + \norm{h} \varepsilon(h) \]   
        Par ailleurs, les fonctions $x_i$ sont de classe $\mathcal{C}^1$ car $x$ l’est, et un DL en $t_0$ donne 
        \[ x_i(t_0 + \delta) = x_i(t_0) + \delta x_i'(t_0) + \comp{o}{\delta}{0}{\delta} \]   
        D’où
        \[ x(t_0 + h) = a + \delta x'(t_0) + \comp{o}{\delta}{0}{\delta} \]   
        On en déduit que 
        \begin{align*}
            f(x(t_0 + h)) 
            &= f(a + \underbrace{\delta x'(t_0) + o(\delta)}_{= h}) \\
            &= f(a) + \spr{\nabla f(a)}{\delta x'(t_0) + o(\delta)} + \norm{\delta x'(t_0) + o(\delta)} \varepsilon(\delta x'(t_0) + o(\delta)) \\
            &= f(a) + \delta \spr{\nabla f(a)}{x'(t_0)} + o(\delta)
        \end{align*}
    \end{demo}

    \subsubsection{Gradient d’une fonction numérique}

    On considère dans ce paragraphe uniquement des fonctions à valeurs dans $\mathbb{R}$ et on suppose $E$, espace vectoriel de dimension $p$, muni d’une structure euclidienne. Soit donc $f : \mathcal{U} \subset E \to \mathbb{R}$ avec bien souvent $E = \mathbb{R}^p$. On suppose $f$ de classe $\mathcal{C}^1$ sur l’ouvert $\mathcal{U}$. D’après ce qui précède, en tout point $a \in \mathcal{U}$, $\d f_a$ est une forme linéaire et, dans la base canonique $\mathcal{B}$ de $\mathbb{R}^p$, 
    \[ \d f_a \in \mathcal{L}(\mathbb{R}^p,\mathbb{R}) \esp{et} J_f(a) = \mat{\mathcal{B}}{\d f_a} = \begin{bmatrix}
        \dpart{f}{x_1}(a) & \cdots & \dpart{f}{x_p}(a) 
    \end{bmatrix} \]   

    On peut définir le gradient de $f$ au point $a$ au moyen de ses coordonnées dans la base $\mathcal{B}$, en posant 
    \[ \nabla f(a) = \begin{bmatrix}
        \dpart{f}{x_1}(a) \\ \vdots \\ \dpart{f}{x_p}(a) 
    \end{bmatrix} \]   

    En notant $\spr{.}{.}$ le produit scalaire usuel de $\mathbb{R}^p$, pour tout $h \in \mathbb{R}^p$, 
    \[ \spr{\nabla f(a)}{h} = \sum_{j=1}^{p} h_j \dpart{f}{x_j}(a) = \d f_a(h) \] 

    La différentiabilité de $f$ au point $a$ se traduit alors par l’égalité 
    \[ f(a+h) = f(a) + \spr{\nabla f(a)}{h} + \comp{o}{h}{0}{\norm{h}} \]   

    Il est également possible de définir le gradient de manière intrinsèque, sans l’usage donc d’une base de $E$, à l’aide du théorème de représentation de Riesz.

    \begin{defi}{}{}
        Soit $f : \mathcal{U} \subset E \to \mathbb{R}$ différentiable en $a$. L’application $\d f_a$ étant une forme linéaire, il peut exister un vecteur appelé gradient de $f$ en $a$ -- et noté $\nabla f(a)$ -- tel que QQS $h \in E$, 
        \[ \d f_a(h) = \spr{\nabla f (a)}{h} \]   
    \end{defi}

    Pour $h$ proche de $0$, l’inégalité de Cauchy-Schwarz donne :
    \[ \abs{f(a+h) - f(a)} \approx \abs{\spr{\nabla f(a)}{h}} \leq \norm{\nabla f(a)} \times \norm{h} \]   
    
    L’écart absolu est maximal dans le cas d’égalité, \textit{i.e.} lorsque la direction $h$ est colinéaire au gradient. Le gradient indique donc bien la ligne de plus grande pente, comme vu dans le cas de deux variables.

    \subsubsection{Arcs paramétrés et dérivées le long d’un arc}

    On appelle arc paramétré de classe $\mathcal{C}^k$ toute fonction vectorielle de $\mathcal{C}^k$, définie sur un intervalle $I$ de $\mathbb{R}$ à valeurs dans $E$. Soit désormais un arc paramétré $\gamma : I \subset \mathbb{R} \to E$ supposé de classe $\mathcal{C}^1$.

    Soit $t_0 \in I$. On suppose que $\gamma'(t_0) \neq 0$. DCC, la courbe représentative de $\gamma$ admet une tangente au point $\gamma(t_0)$ dirigée par le vecteur $\gamma'(t_0)$. Pour prouver ce résultat, considérons la droite $\Delta_t$ issue de $M(t_0)$ dirigée par le vecteur $\vct{M(t_0)M(t)}$, \textit{i.e.} $\gamma(t) - \gamma(t_0)$, et faisons tendre $t$ vers $t_0$. Plus formellement, on définit la tangente à la courbe en $M(t_0)$ comme la droite passant par $M(t_0)$ et dirigée par le vecteur :
    \[ \frac{\gamma(t) - \gamma(t_0)}{\norm{\gamma(t) - \gamma(t_0)}} = \frac{\gamma(t) - \gamma(t_0)}{t - t_0} \cdot \frac{t - t_0}{\norm{\gamma(t) - \gamma(t_0)}} \limi{t}{t_0} \frac{\pm \gamma'(t_0)}{\norm{\gamma'(t_0)}} \]  

    Soit désormais $f : \mathcal{U} : E \to F$ et $\gamma : I \subset \mathbb{R} \to E$ deux applications supposées de classe $\mathcal{C}^1$ respectivement sur l’espace normé $E$ et l’intervalle $I$. L’application composée $f \circ \gamma$ est elle-même définie sur $I$ et à valeurs dans $F$. Géométriquement, c’est l’image de l’arc par la tranformation $f$. Si $M(t_0)$ est régulier, \textit{i.e.} si $\gamma'(t_0) \neq 0$, alors $\gamma'(t_0)$ dirige la tangeante à la courbe $\gamma$ en $M(t_0)$. Qu’en est-il pour la courbe $f \circ g$ ?

    \begin{prop}{Dérivation le long d’un arc}{}
        Si $f : \mathcal{U} \subset E \to F$ et $\gamma : I \subset \mathbb{R} \to \mathcal{U}$ sont de classe $\mathcal{C}^1$, alors $f \circ \gamma$ est de classe $\mathcal{C}^1$ sur $I$ et 
        \[ \forall t \in I, \quad (f \circ \gamma)'(t) = \d f_{\gamma(t)}(\gamma'(t)) \]   
    \end{prop}

    \begin{demo}{Preuve}{myolive}
        Cela provient du résultat sur la différentiation d’une composée, avec $\gamma'(t) = \d g_t(1)$ pour $t \in I$.
    \end{demo}

    La tengente à la courbe $f \circ \gamma$ en $t_0$ est dirigée par $(f \circ \gamma)'(t_0) = \d f_{\gamma(t_0)}(\gamma'(t_0))$ si celui-ci est non nul. Ce dernier vecteur n’est rien d’autre que l’image du vecteur qui dirige la tangente à $\gamma$ en $t_0$ par l’AL $\d f_{gamma(t_0)}$. Ce résultat appelle plusieurs remarques :
    \begin{itemize}
        \item Si $\gamma(t) = x + tu$ avec $x,u \in E$, alors $\gamma$ est un paramétrage de la droite affine passant par $x$ et dirigée par $u$. $\gamma'(t) = u$ est comme attendu constante et alors $(f \circ \gamma)'(t) = \d f_{\gamma(t)}(u)$. 
        \item Si $\gamma(t) = (x_1(t), \ldots, x_p(t))$ est de classe $\mathcal{C}^1$, on généralise la règle de la chaîne :
        \[ (f \circ \gamma)'(t) = \d f_{\gamma(t)}(\gamma'(t)) = \sum_{j=1}^{p} x_j'(t) \dpart{f}{x_j}(x_1(t), \ldots, x_p(t)) \]   
        \item Enfin, si $f$ est une application numérique, $(f \circ \gamma)'(t) = \d f_{\gamma(t)}(\gamma'(t)) = \spr{\nabla f(y(t))}{\gamma'(t)}$. Ajoutons que si l’on suppose toute ligne de niveau décrite par un paramétrage de classe $\mathcal{C}^1$, $f \circ \gamma$ est constante. En dérivant, l’on obtient $\spr{\nabla f(\gamma(t))}{\gamma'(t)} = 0$, ce qui assure l’orthogonalité du gradient aux vecteurs qui dirigent les tangentes aux lignes de niveau.
    \end{itemize}

    \begin{prop}{Intégration le long d’un arc}{}
        Si $f : \mathcal{U} \subset E \to F$ et $\gamma : \intervalleFF{0}{1} \to \mathbb{U}$ sont de classe $\mathcal{C}^1$ et si $\gamma(0) = a$ et $\gamma(1) = b$, alors 
        \[ f(b) - f(a) = \int_{0}^{1} \d f_{\gamma(t)}(\gamma'(t))dt \]   
    \end{prop}

    \begin{demo}{Preuve}{myolive}
        \[ \int_{0}^{1} \d f_{\gamma(t)} (\gamma'(t))dt = \int_{°}^{1} (f \circ \gamma)'(t)dt = f(b) - f(a) \]   
    \end{demo}

    On notera que le résultat ne dépend pas du chemin choisi.

    \begin{coro}{Caractérisation des fonctions constantes}{}
        Soient $\mathcal{U}$ un ouvert connexe par arcs et $f : \mathcal{U} \subset E \to F$. Alors 
        \[ f \text{ est constante sur } \mathcal{U} \iff \forall a \in \mathcal{U}, \d f_a = 0_{\mathcal{L}(E,F)} \]   
    \end{coro}

    \begin{demo}{Preuve}{myorange}
        L’implication est simple, et la réciproque se montre en considérant la fonction $\gamma$ définie pour $a,b \in \mathcal{U}$ par $\gamma(t) = (1-t)a + tb$ qui est bien un chemin de classe $\mathcal{C}^1$ de $\mathcal{U}$. Par nullité de la différentielle, 
        \[ f(b) - f(a) = \int_{0}^{1} \d f_{\gamma(t)}(\gamma'(t)) = 0 \]   
    \end{demo}

    On généralise ainsi un résultat bien connu : une fonction $f \in \mathcal{C}^1(I,\mathbb{R})$ est constante sur l’intervalle $I$ SSI $f'$ est nulle sur l’intervalle $I$. Rappelons que les parties de $\mathbb{R}$ connexes par arcs sont les intervalles. 

    \subsubsection{Vecteur tangent à une partie}

    On étend maintenant la notion de vecteur tangent aux parties d’un espace vectoriel de dimension finie. 

    \begin{defi}{}{}
        Si $X$ est une partie de $E$ et $x \in X$, un vecteur $v$ de $E$ est dit tangent à $X$ en $x$ s’il existe $\varepsilon > 0$ et un arc paramétré $\gamma$ défini sur $\intervalleOO{-\varepsilon}{\varepsilon}$. dérivable en $0$ et à valeurs dans $X$, tels que $\gamma(0) = x$ et $\gamma'(0) = v$. ON $T_x X$ l’ensemble des vecteurs tangents à $X$ en $x$.
    \end{defi}

    Un tel ensemble $T_x X$ n’est pas, en général, un EV. Pour déterminer les vecteurs tangents à une partie, on procède souvent par double inclusion.

    Étudions désormais les plans tangents à une surface d’équation de $\mathbb{R}^3$ d’équation $z = f(x,y)$ en un point donné. Soient $\mathcal{S}$ la surface d’équation $z = f(x,y)$ où $f: \mathbb{R}^2 \to \mathbb{R}$ est supposée de classe $\mathcal{C}^1$ et $M_0(x_0,y_0,z_0)$ un point de $\mathcal{S}$. Le plan tangent à $M_0$ peut être défini comme la réunion des tangentes aux courbes tracées le long de $\mathcal{S}$ en passant par le point $M_0$, \textit{i.e.} le plan affine passant par $M_0$ et de direction le plan vectoriel $T_{M_0}\mathcal{S}$. Il reste à vérifier que $T_{M_0}\mathcal{S}$ est, comme attendu, un plan vectoriel.

    \begin{itemize}
        \item On construit un arc sur $\mathcal{S}$ en considérant deux fonctions $x,y : \intervalleOO{-\varepsilon}{\varepsilon} \to \mathbb{R}$ de classe $\mathcal{C}^1$ vérifiant $x(0) = x_0$ et $y(0) = y_0$ et $\gamma$ définie par 
        \[ \forall \gamma'(0) = \left(x'(0), y'(0), x'(0) \dpart{f}{x}(x_0,y_0) + y'(0) \dpart{f}{y}(x_0,y_0)\right) \]
        
        Ce vecteur tangent à la courbe en $M_0$ est, quel que soit l’arc $\gamma$ considéré, orthogonal au vecteur 
        \[ n = \left(\dpart{f}{x}(x_0,y_0),\dpart{f}{y}(x_0,y_0), -1\right) \]   
        \item Réciproquement, soit $v = (v_1,v_2,v_3)$ orthogonal à $n$. Il suffit de considérer l’arg $\gamma$ défini par 
        \[ \forall t \in \intervalleOO{-\varepsilon}{\varepsilon}, \quad \gamma(t) = (x_0 + tv_1, y_0 + tv_2, f(x_0+tv_1, y_0 + tv2)) \]   
        pour constater que $v \in T_{M_0} \mathcal{S}$. En effet, $\gamma$ est de classe $\mathcal{C}^1$ et à valeurs dans $\mathcal{S}$, $\gamma(0) = M_0$ et $\gamma'(0) = v$. 
    \end{itemize} 
    Nous venons ainsi de prouver que $T_{M_0}\mathcal{S} = \vect(n)^{\top}$.

    Le plan tangent à $\mathcal{S}$ en $M_0$ peut alors être défini comme l’unique plan passant par $M_0$ et de vecteur normal $n$. On retrouve l’équation déjà donnée par troncature à l’ordre $1$ du DL de $f$ en $(x_0,y_0)$ : 
    \[ z = f(x_0,y_0) + \dpart{f}{x}(x_0,y_0)(x - x_0) + \dpart{f}{y}(x_0,y_0)(y - y_0) \]    

    La description d’une surface de $\mathbb{R}^3$ par une équation de la forme $z = f(x,y)$ n’englobe cependant qu’un nombre restreint de surfaces de l’espace. On définit plus généralement une surface de $\mathbb{R}^3$ par la donnée d’une équation implicite $g(x,y,z) = 0$, comme pa exemple $(x - x_0)^2 + (y - y_0)^2 + (z - z_0)^2 = R^2$. Il n’est pas toujours possible d’exprimer $z$ en fonction de $x$ et $y$ pour se ramener au cas particulier précédent. Le théorème suivant permet de contourner cette objection pour déterminer le plan tangent à une telle surface. Sa démonstration fait appel au théorème des fonctions implicites, ce dernier précisant les conditions pour exprimer localement $z$ comme une fonction de $x$ et $y$.

    \begin{theo}{}{}
        Soient $g$ une fonction numérique de classe $\mathcal{C}^1$ sur l’ouvert $\mathcal{U}$, $X$ l’ensemble des zéros de $g$ et $x \in X$. Si $\d g_x$ est non nulle, $T_x X = \ker(\d g_x) = \nabla g(x)^{\perp}$.
    \end{theo}

    \begin{demo}{Démonstration}{myred}
        Soient $g : \mathcal{U} \subset \mathbb{R}^p \to \mathbb{R}$ de classe $\mathcal{C}^1$, $X$ l’hypersurface d’équation $g(x_1,\ldots,x_p) = 0$ et $x \in X$ tel que $\d g_x \neq 0$. Remarquons pour commencer que $\d g_{x} \neq 0$, donc $\ker(\d g_x)$ est un hyperplan de $\mathbb{R}^p$. De plus, 
        \[ h \in \ker(\d g_x) \iff \d g_x(h) = 0 \iff \spr{\nabla g(x)}{h} = 0 \iff h \in \nabla g(x)^{\perp} \] 
        \begin{itemize}
            \item[$\subset$] Soit $v \in T_x X$. Il existe $\gamma : \intervalleOO{-\varepsilon}{\varepsilon} \to X$ de classe $\mathcal{C}^1$ tel que $\gamma(0) = x$ et $\gamma'(0) = v$.
            \[ \forall t \in \intervalleOO{-\varepsilon}{\varepsilon}, \quad g(\gamma(t)) = 0 \esp{donc} \d g_{\gamma(t)}(\gamma'(t)) = 0 \]   
            En évaluant en $0$, il vient $\d g_x(v) = 0$ 
            \item[$\supset$] Admis.
        \end{itemize}
    \end{demo}



    \subsubsection{Brève extension aux applications de classe Ck}

    Pour définir les dérivées partielles d’ordres supérieurs, il suffit tout bonnement de dériver les dérivées. On notera, sous réserve d’existence, $\frac{\partial^2 f}{\partial x_i \partial x_j}$ ou $\partial_{i,j} f$ la dérivée partielle d’indice $i$ de la dérivée partielle $\dpart{f}{x_j}$. On définit plus généralement par récurrence les dérivées partielles d’ordre $k$, QQS $k \in \mathbb{N}^*$ :
    \[ \partial_{j_k, \ldots, j_1} f = \frac{\partial}{\partial x_{j_k}} \left( \partial_{j_{k-1}, \ldots, j_1} f \right) \]   

    \begin{defi}{Application de classe $\mathcal{C}^k$}{}
        Une application est dite de classe $\mathcal{C}^k$ sur un ouvert de $\mathcal{U}$ si toutes ses dérivées partielles d’ordre $k$ existent et sont continues. 
    \end{defi}

    Les sommes et composées de fonctions de classe $\mathcal{C}^k$ sont encore de classe $\mathcal{C}^k$, et EP les fonctions polynomiales sont de classe $\mathcal{C}^{\infty}$. 

    \begin{theo}{Théorème de Schwarz}{}
        Soit $f : \mathcal{U} \subset E \to F$ une application de classe $\mathcal{C}^2$ sur un ouvert $\mathcal{U}$. Alors 
        \[ \forall i,j \in \intervalleEntier{1}{p}, \quad \forall x \in \mathcal{U}, \quad \partial_{i,j} f(x) = \partial_{j,i} f (x) \]   
    \end{theo}

    Ce théorème, ici admis, s’étend aux dérivées d’ordres supérieurs. Il montre que l’ordre de dérivation importe peu, du moment que la fonction est suffisamment régulière. 

\subsection{Optimisation avec et sans contrainte}

    Dans cette section, $f : \mathcal{U} \subset E \to \mathbb{R}$ désigne une application de classe $\mathcal{C}^2$ définie sur un ouvert $\mathcal{U}$ d’un espace vectoriel de dimension $n$, souvent $\mathbb{R}^n$, à valeurs numériques.

    \subsubsection{Formule de TY à l’ordre 2 pour une fonction numérique}

    \begin{theo}{}{}
        QQS $a \in \mathcal{U}$, 
        \[ f(a+h) = f(a) + \sum_{j=1}^{n} h_j \dpart{f}{x_j}(a) + \frac{1}{2} \sum_{1 \leq i,j \leq n} h_i h_j \partial_{i,j}f(a) + \comp{o}{h}{0}{\norm{h}^2} \]    
    \end{theo}

    \begin{demo}{Démonstration}{myred}
        Soit $a \in \mathcal{U}$. Il existe $r > 0$ tel que pour tout $h \in \mathcal{B}(a,r)$ et $t \in \intervalleFF{0}{1}, a + th \in \mathcal{U}$. 
        \begin{itemize}
            \item On introduit la fonction de la variable réelle $\varphi$ définie sur $\intervalleFF{0}{1}$ par $\varphi(t) = f(a + th)$. $\varphi$ est de classe $\mathcal{C}^2$ comme composée de fonctions de classe $\mathcal{C}^2$ et 
            \[ \forall t \in \intervalleFF{0}{1}, \quad \varphi'(t) = \sum_{j=1}^{n} h_j \dpart{f}{x_j}(a + th) \esp{et} \varphi''(t) = \sum_{i=1}^{n} \sum_{j=1}^{n} h_i h_j \partial_{i,j} f(a + th) \]   
            La formule de Taylor avec reste intégral à l’ordre $2$ pour $\varphi$, \textit{i.e.} $\varphi(1) = \varphi(0) + \varphi'(0) + \int_{0}^{1} (1-t) \varphi''(t) dt$, nous donne celle relative à $f$ : 
            \[ f(a + h) = f(a) + \sum_{j=1}^{n} h_j \dpart{f}{x_j}(a) + \int_{0}^{1} (1-t) \left[\sum_{i=1}^{n} \sum_{j=1}^{n} h_i h_j \partial_{i,j} f (a + t h)\right] dt \]
            \item Exploitons maintenant la continuité des dérivées partielles secondes : 
            \[ \forall i,j \in \intervalleEntier{1}{n}, \partial_{i,j}f(a + th) = \partial f(a) + \varepsilon_{i,j}(th) \esp{où} \varepsilon_{i,j}(u) \limi{u}{0_E} 0 \]   

            La formule de Taylor avec reste intégral devient 
            \[ f(a+h) = f(a) + \sum_{j=1}^{n} h_j \dpart{f}{x_j}(a) + \frac{1}{2} \sum_{1 \leq i,j \leq n} h_i h_j \partial_{i,j} f(a) + \sum_{1 \leq i,j \leq n} \int_{0}^{1} (1-t)\varepsilon_{i,j}(th)dt \]    

            Il reste à prouver que $R(h) = \sum_{1 \leq i,j \leq n} h_i h_j \int_{0}^{1} (1-t) \varepsilon_{i,j}(th) dt = \comp{o}{h}{0}{\norm{h}^2}$. 

            Pour cela, fixons $\varepsilon > 0$. Il existe $\delta > 0$ tel que pour tous $i,j \in \intervalleEntier{1}{p}$, si $\norm{u} \leq \delta$ alors $\abs{\varepsilon_{i,j}(u)} < \varepsilon$. Soit $h \in E$ tel que $\norm{h} \leq \delta$. Alors, pour tout $t \in \intervalleFF{0}{1}$, $\norm{th} \leq \delta$ et donc :
            \[ \forall i,j \in \intervalleEntier{1}{p}, \abs{\int_{0}^{1} (1-t)\varepsilon_{i,j}(th)} \leq \varepsilon \int_{0}^{1} (1-t)dt = \frac{\varepsilon}{2} \]   

            Ainsi, chaque terme de la somme est un $o(h_i h_j)$. De plus $\abs{h_i h_j} \leq \norm{h}^2_{\infty}$ donc $R(h) = \comp{o}{h}{0}{\norm{h}^2}$.
        \end{itemize}
    \end{demo}

    La formule de TY à l’ordre 2 se réécrit sous la forme plus concise suivante : 
    \[ f(a+ h) = f(a) + \d f_a(h) + \frac{1}{2} \d^2 f_a(h,h) + \comp{o}{h}{0}{\norm{h}^2} \]    
    où l’application bilinéaire $\d^2 f_a : (h,k) \mapsto \sum_{1 \leq i , j \leq n} h_i k_j \partial_{i,j} f (a)$ est appelé différentielle seconde de $f$ en $a$. 

    Il importe de savoir réécrire la formule pour une fonction de deux variables (cas le plus courant) : 
    \begin{align*}
        f(x_0 + h_1, y_0 + h_2) &= f(x_0) + \dpart{f}{x}(x_0,y_0) h_1 + \dpart{f}{y}(x_0,y_0) h_2 \\
        &+\frac{1}{2} \left[ \partial_{1,1} f (x_0,y_0) h_1^2 + 2 \partial_{1,2} f (x_0,y_0) h_1 h_2 + \partial_{2,2} f (x_0,y_0) h_2^2 \right] + \comp{o}{(h_1,h_2)}{(0,0)}{h_1^2 + h_2^2} 
    \end{align*}

    L’application $f$ étant à valeurs dans $\mathbb{R}$, on dispose d’une écriture matricielle de la formule de Taylor-Young à l’ordre $2$ assez commode. Elle s’appuie sur les relations suivantes : 
    \[ \d f_a(h) = \sum_{j = 1}^n h_j \dpart{f}{x_j}(a) = \spr{\nabla f(a)}{h} \esp{et} \d^2 f_a(h) = \sum_{1 \leq i,j \leq n} \partial_{i,j} f(a) = \spr{H_f(a)h}{h} \]    
    où l’on a noté $h = (h_j)_{1 \leq j \leq n}$ et $H_f(a) = \left(\partial_{i,j}f(a)\right)_{1 \leq i,j \leq a}$.

    \begin{defi}{Hessienne}{}
        La \textbf{hessienne} au point $a \in \mathcal{U}$ de la fonction numérique $f : \mathcal{U} \subset \mathbb{R}^n \to \mathbb{R}$ de classe $\mathcal{C}^2$ est la matrice 
        \[ H_f(a) = \left(\partial_{i,j}f(a)\right)_{1 \leq i,j \leq a} \qquad \in \S_n(\mathbb{R}) \]   
    \end{defi}

    Notons que la hessienne est bien symétrique, dès lors que $f$ est de classe $\mathcal{C}^2$, en vertu du théorème de Schwarz.

    On retiendra la version matricielle suivante de la formule de TY à l’ordre 2 pour une fonction de classe $\mathcal{C}^2$ sur un ouvert $\mathcal{U}$ de $E$ : QQS $a \in \mathcal{U}$, 
    \[ f(a+h) = f(a) + \spr{\nabla f}{h} + \frac{1}{2} \spr{H_f(a)h}{h} + \comp{o}{h}{0}{\norm{h}^2} = f(a) + \nabla f (a)^{\top} h + \frac{1}{2} h^{\top} H_f(a) h + \comp{o}{h}{0}{\norm{h}^2} \]   

    \subsubsection{Étude des extrema libres}

    Par la suite, $f$ désignera une application définie sur une partie $A$ de $E$, espace vectoriel de dimension $n$, et à valeurs dans $\mathbb{R}$. 

    \begin{defi}{Extremum}{}
        On dit que $f$ admet un minimum (RSP. maximum)  local en $a \in A$ s’il existe un voisinage $\mathcal{U}$ de $a$ tel que 
        \[ \forall x \in \mathcal{U}, \quad f(x) \geq f(a) \qquad (\text{RSP. } f(x) \leq f(a)) \]    
        On qualifie de global un extremum lorsque l’inégalité est valable sur $A$.        
    \end{defi}

    Rappelons que si $A$ est une partie compacte de $E$, $f$ admet nécessairement un minimum et un maximum. De plus, pour une fonction de la variable réelle $f$ de classe $\mathcal{C}^1$ sur un intervalle $I$ :
    \begin{itemize}
        \item Si $f$ atteint un extremum local en $a \in \mathring{I}$, $f'(a) = 0$ ;
        \item $f'$ peut s’annuler sans que $f$ atteigne un extremum ;
        \item si $I$ n’est pas ouvert, $f$ peut atteindre un extremum sans que sa dérivée s’annule.
    \end{itemize}
    Ajoutons que lorsque $f$ est de classe $\mathcal{C}^2$ :
    \begin{itemize}
        \item Si $f$ atteint un minimul local en $a \in \mathring{I}$, $f''(a) \geq 0$ ;
        \item si $f'(a) = 0$ et $f''(a) > 0$ pour $a \in \mathring{I}$, $f$ admet un minimum local.
    \end{itemize}

    La recherche des extrema ne diffère guère de celle que les lecteurs connaissent pour les fonctions de la variable réelle. On prendra garde au fait que la plupart des énoncés ci-dessous sont valables uniquement sur des parties ouvertes. Il sera donc parfois nécessaire de dissocier la recherche des extrema de $f$ à l’intérieur de $A$ et le long de sa frontière.

    \begin{defi}{Point critique}
        Soit $f$ une fonction de classe $\mathcal{C}^1$ sur un ouvert $\mathcal{U}$ de $E$. On dit que $a \in \mathcal{U}$ est un point critique de $f$ si 
        \[ \d f_a = 0_{\mathcal{L}(E,F)} \esp{soit} \nabla f(a) = \vect{0} \]   
    \end{defi}

    Cela revient à dire que pour tout $j \in \intervalleEntier{1}{n}$, $\dpart{f}{x_j}(a) = 0$.

    \begin{theo}{Condition nécessaire d’existence d’un extremum}{}
        Si $f : \mathcal{U} \subset E \to \mathbb{R}$ de classe $\mathcal{C}^1$ sur un ouvert $\mathcal{U}$ admet un extremum en $a$, alors $a$ est un point critique.
    \end{theo}

    \begin{demo}{Preuve}{myred}
        Soit $j \in \intervalleEntier{1}{n}$. L’application de la variable réelle $\varphi : t \mapsto f(a + t e_j)$ est de classe $\mathcal{C}^1$ au voisinage de $0$ et admet un extremum en $0$. Ainsi, $\varphi'(0) = \dpart{f}{x_j}(a) = 0$.
    \end{demo}

    Les extrema locaux sont donc à rechercher parmi les points critiques. Cependant, comme en dimension $1$, la réciproque est fausse (voir point selle).

    Il sera donc nécessaire lors de l’étude d’une fonction de distinguer les point cols des points correspondant à des extrema. Pour justifier que l’on dispose d’un extremum, on peut revenir à la définition.

    L’étude globale du signe de $f(x,y) - f(x_0,y_0)$ menée pour justifier la présence d’un minimum n’est pas satisfaisante car difficilement reproductible pour une fonction quelconque. Pour pallier cette difficulté, il suffit de pousser l’étude locale au voisinage d’un point critique au moyen de la formule de TY. En effet, si $f$ est de classe $\mathcal{C}^2$ sur l’ouvert $\mathcal{U}$ et $a$ est un point critique de $f$, alors 
    \[ f(a+h) = f(a) + \frac{1}{2} h^{\top} H_f(a) h + \comp{o}{h}{0}{\norm{h}^2} \]    
    Le signe de $f(a+h) - f(a)$ est donc localement celui de $\spr{H_f(a)h}{h}$, ce qui conduit au résultat suivant : 

    \begin{theo}{Condition suffisante d’existence d’un extremum}{}
        Soient $f : \mathcal{U} \subset E \to \mathbb{R}$ de classe $\mathcal{C}^2$ sur l’ouvert $\mathcal{U}$ et $a \in \mathcal{U}$. 
        \begin{enumerate}
            \item Si $f$ atteint en $a$ un minimul local, $H_f(a) \in \S_n^+(\mathbb{R})$ ;
            \item si $a$ est un point critique de $f$ et si $H_f(a) \in \S_n^{++}(\mathbb{R})$, alors $f$ atteint en $a$ un minimum local.
        \end{enumerate}
    \end{theo}

    Bien entendu, le théorème précédent se réécrit pour $a$ un maximum et $H_f(a) \in \S_n^{-}(\mathbb{R})$.

    \begin{demo}{Démonstration}{myred}
        Soient $f : \mathcal{U} \subset E \to \mathbb{R}$ de classe $\mathcal{C}^2$ sur l’ouvert $\mathcal{U}$ et $a \in \mathcal{U}$. 
        \begin{enumerate}
            \item $a$ est un point critique de $f$. Si $h \in E$, pour $t \in \mathbb{R}$ suffisamment petit, $a + th \in \mathcal{U}$ et $f(a - th) - f(a) \geq 0$. Or 
            \[ f(a + th) - f(a) = \frac{t^2}{2}\spr{H_f(a)h}{h} + o(t^2) \esp{CAD} \spr{H_f(a)h}{h} = \frac{2}{t^2} (f(a + th) - f(a)) + o(1) \]     
            Un passage à la limite permet de conclure. 
            \item On note $\lambda_{min}$ la plus petite des valeurs propres (nécessairement réelle) de $H_f(a)$. D’après le théorème spectral, 
            \[ \forall h \in E \backslash \left\{0_E\right\}, \quad \spr{H_f(a)h}{h} \geq \lambda_{min}\norm{h}^2 > 0 \]   
            $f(a + h) - f(a) = \frac{1}{2} h^{\top} H_f(a) h + \norm{h}^2 \varepsilon(h)$ donc il existe $r > 0$ tel que pour tout $h \in \mathcal{B}(0,r)$, $\norm{h}^2 \varepsilon(h) \leq \lambda_{min} \frac{\norm{h}^2}{4}$. Ainsi, pour tout $h \in \mathcal{B}(0,r)$, 
            \[ f(a+h) - f(a) = \frac{1}{2}h^{\top} H_f(a) h + \norm{h}^2 \varepsilon(h) \geq \lambda_{min}\frac{\norm{h}^2}{2} - \lambda_{min} \frac{\norm{h}^2}{4} \geq 0 \]
        \end{enumerate}
    \end{demo}

    \begin{coro}{}{}
        Si $a \in \mathcal{U}$ est un point critique de $f$, 
        \begin{itemize}
            \item Si $\sp(H_f(a)) \subset \mathbb{R}^*_+$, $f$ atteint en $a$ un minimum.
            \item Si $\sp(H_f(a)) \subset \mathbb{R}^*_-$, $f$ atteint en $a$ un maximum.
            \item Si $H_f(a)$ possède deux valeurs propres de signes distincts, $a$ est un point selle.
        \end{itemize}
    \end{coro}

    On notera que pour $n = 2$, on dispose d’une caractérisation simple au moyen du déterminant et de la trace. En effet, en notant $\lambda$ et $\mu$ les deux valeurs propres de $H_f(a)$, $\det(H_f(a)) = \lambda \mu$ et $\tr(H_f(a)) = \lambda + \mu$.

    \begin{coro}{}{}
        Soient $f : \mathcal{U} \subset \mathbb{R}^2 \to \mathbb{R}$ de classe $\mathcal{C}^2$ sur l’ouvert $\mathcal{U}$ et $a \in \mathcal{U}$ un point critique de $f$.
        \begin{itemize}
            \item Si $\det(H_f(a)) > 0$, $f$ admet un extremum en $a$. Il s’agit d’un minimul lorsque $\tr(H_f(a)) > 0$, d’un maximum sinon.
            \item Si $\det(H_f(a)) < 0$, $f$ admet un point selle.
        \end{itemize}
    \end{coro}

\newpage

\section{Fonctions vectorielles}

    On appelle fonction vectorielle d’une variable réelle toute fonction définie sur un intervalle $I$ de $\mathbb{R}$, à valeurs dans un espace vectoriel normé $E$ de dimension finie. En pratique, on considèrera souvent des fonctions à valeurs dans $E = \mathbb{R}^p$, \textit{i.e.} des fonctions de la forme 
    \[ \fonction{f}{I}{\mathbb{R}^p}{t}{\left( f_1(t),\ldots,f_p(t) \right)} \]
    Les fonctions numériques $f_i : I \to \mathbb{R}$ pour $i \in \intervalleEntier{1}{p}$ dons appelées \textbf{fonctions composantes} ou \textbf{fonctions coordonnées} de $f$. Plus généralement, si $E$ est un espace vectoriel normé de dimension finie muni d’une base $\mathcal{B} = (e_1,\ldots,e_p)$, alors $f = f_1e_1 + \ldots + f_p e_p$. L’équivalence des normes en dimension finie nous assurera que les propriétés de régularité (comme la continuité et la dérivabilité) ne dépendent pas de la base choisie.

    Dans tout cette section, $f$ désignera une fonction définie sur un intervalle $I$ et à valeurs dans un espace vectoriel normé $\left(E, \norm{.}\right)$ de dimension finie $p$.

\subsection{Limite et continuité d’une fonction vectorielle}

    \begin{defi}{Limite}{}
        On dit que $f$ admet $\ell \in E$ pour \textbf{limite} en $t_0 \in I$ si 
        \[ \forall \varepsilon > 0, \quad \exists \eta > 0, \quad \forall t \in I, \quad \abs{t - t_0} < \eta \implies \norm{f(t) - \ell} < \varepsilon \]
        \textit{i.e.} si 
        \[ \lim_{t \to t_0} \norm{f(t) - \ell} = 0 \]
    \end{defi}

    \begin{omed}{Remarque}{myyellow}
        On peut montrer que lorsque la limite existe, elle est unique. 
        
        De plus, si $E = \mathbb{R}^p$, $f$ admet $\ell = (\ell_1,\ldots,\ell_p) \in \mathbb{R}^p$ pour limite en $t_0 \in \mathbb{R}$ \textit{ssi} chaque fonction composante $f_i$ admet $\ell_i$ comme limite en $t_0$.
    \end{omed}

    \begin{defi}{Continuité}{}
        On dit que $f$ est \textbf{continue} en $t_0 \in I$ si $\lim_{t \to t_0} f(t) = f(t_0)$, \textit{i.e.} si 
        \[ \forall \varepsilon > 0, \quad \exists \eta > 0, \quad \forall t \in I, \quad \abs{t - t_0} < \eta \implies \norm{f(t) - f(t_0)} < \varepsilon \]
        On dit que $f$ est continue sur $I$ si $f$ est continue en tout point de $I$.
    \end{defi}

    \begin{prop}{Caractérisation de la continuité par les fonctions composantes}{}
        $f$ est continue en $t_0 \in I$ \textit{ssi} $\forall i \in \intervalleEntier{1}{p}$, $f_i$ est continue en $t_0$.
    \end{prop}

\subsection{Dérivabilité d’une fonction vectorielle}

    \begin{defi}{Dérivabilité}{}
        \begin{itemize}
            \item $f$ est dite dérivable en $t_0 \in I$ si $\lim_{t \to t_0} \frac{f(t) - f(t_0)}{t - t_0}$ existe.
            
            On appelle alors vecteur dérivé en $t_0$ ce nombre, noté $f'(t_0)$.
        \end{itemize}
    \end{defi}

    De manière équivalente, $f$ est dérivable en $t_0 \in I$ \textit{ssi} $\lim_{h \to 0} \frac{f(t_0 + h) - f(t_0)}{h}$ existe, ce qui est équivalent à 
    \[ f(t_0 + h) = f(t_0) + hf'(t_0) + \comp{o}{h}{0}{h} \]

    \begin{prop}{Dérivabilité des fonctions composantes}{}
        Si $E$ est muni d’une base $\mathcal{B} = (e_1,\ldots, e_p)$ et $f = f_1 e_1 + \ldots f_p e_p$, alors $f$ est dérivable \textit{ssi} les fonctions $f_i$ le sont. Dans ce cas, $f' = f_1' e_1  + \ldots f_p' e_p$.
    \end{prop}

    \begin{omed}{Exemple}{myolive}
        L’application $R : t \mapsto \begin{bmatrix}
            \cos(t) & -\sin(t) \\
            \sin(t) & \cos(t)
        \end{bmatrix}$ est dérivable sur $\mathbb{R}$, de dérivée $t \mapsto R(t + \pi / 2)$.
    \end{omed}

    \begin{prop}{Dérivabilité et application linéaire}{}
        Soient $f : I \to E$ une fonction dérivable sur $I$ et $u \in \mathcal{L}(E,F)$.

        Alors $u \circ f$ est dérivable et $\left(u \circ f\right)' = u \circ f'$.
    \end{prop}

    \begin{demo}{Preuve}{myolive}
        Tout repose sur la linéarité (et donc la continuité de $u$). En effet, 
        \[ \frac{u(f(t_0 + h)) - u(f(t_0))}{h} = u \left(\frac{f(t_0 + h) - f(t_0)}{h}\right) \limi{h}{0} u(f'(t_0)) \]

        On peut aussi écrire $u(f(t_0 + h)) = u(f(t_0)) + h u(f'(t_0)) + \comp{o}{h}{0}{h}$ par continuité de $u$.
    \end{demo}

    \begin{prop}{Dérivabilité et application bilinéaire}{}
        Soient $f : I \to F$ et $g : I \to G$ deux fonctions dérivables sur $I$ et $B : F \times G \to E$ bilinéaire.

        Alors $B(f,g)$ est dérivable sur $I$ et 
        \[ B(f,g)' = B(f',g) + B(f,g') \]
    \end{prop}

    \begin{itemize}
        \item En particulier, si $E = \mathbb{R}^3$, $f \wedge g$ est dérivble sur $I$ et 
        \[ \left(f \wedge g\right)' = f' \wedge g + f \wedge g' \]   
        \item Si $\spr{.}{.}$ désigne un produit scalaire sur $E$, $\spr{f}{g}$ est dérivable sur $I$ et 
        \[ \spr{f}{g}' = \spr{f'}{g} + \spr{f}{g'} \]   
    \end{itemize}

    \begin{demo}{Preuve}{myolive}
        Travaillons ici avec les développements limités. $f$ et $g$ sont supposées dérivables sur $I$ donc pour tout $t_0 \in I$, en utilisant la bilinéarité de $\mathcal{B}$,
        \[ B\left(f(t_0 + h), g(t_0 + h)\right) = B\left(f(t_0),g(t_0)\right) + h \left[B\left( f'(t_0), g(t_0) \right) + B\left(f(t_0), g'(t_0)\right)\right] + \comp{o}{h}{0}{h} \]
        C’est la continuité de $B$ qui permet, par exemple, d’affirmer que $B(f(t_0), o(h)) = o(h)$.
    \end{demo}

    La propriété précédente s’étend à toute application mulitilinéaire.

    \begin{defi}{Classe $\mathcal{C}^k$}{}
        Soit $k \in \mathbb{N}$. L’application $f$ est dite de classe $\mathcal{C}^k$ sur $I$ si elle est dérivable $k$ fois sur $I$ et si sa dérivée $k$-ème, notée $f^{(k)}$, est continue sur $I$.
    \end{defi}

    \begin{prop}{Formule de Leibniz}{}
        Si $f : I \to \mathbb{R}^p$ et $\lambda : I \to \mathbb{R}$ sont de classe $\mathcal{C}^n$ sur $I$, 
        \[ \left(\lambda \cdotp f\right)^{(n)} = \sum_{k=0}^{n} \binom{n}{k} \lambda^{(k)} f^{(n-k)} \]
    \end{prop}

\subsection{Intégration d’une fonction vectorielle sur un segment}

    Pour définir l’intégrale d’une fonction continue sur un segment $\intervalleFF{a}{b}$ à valeur dans un e.v.n. de dimension $p$, il suffit d’intégrer les fonctions composantes (qui sont des fonctions numériques) dans une base prédéfinie :
    \[ \int_{a}^{b} f(t)dt = \int_{a}^{b} \left(\sum_{i=1}^{p} f_i(t)e_i \right) dt = \sum_{i=1}^{p} \left(\int_{a}^{b} f_i(t)dt\right) e_i \]
    On admet que le résultat ne dépend pas de la base choisie.

    On retrouve les propriétés classiques de l’intégrale en raisonnant composante par composante, à l’exception de la positivité et de la croissance (propriétés découlant de la relation d’ordre naturelle sur $\mathbb{R}$).

    \begin{prop}{Propriétés de l’intégrale}{}
        Soient $f,g : \intervalleFF{a}{b} \to E$ deux fonctions continues où $(E,\norm{.})$ est un e.v.n. de dimension finie.
        \begin{enumerate}
            \item Linéarité de l’intégrale.
            \item Relation de Chasles.
            \item Convergence des sommes de Riemann.
            \item Inégalité triangulaire :
            \[ \norm{\int_{a}^{b} f(t)dt} \leq \int_{a}^{b} \norm{f(t)}dt \]
        \end{enumerate}
    \end{prop}

    \begin{demo}{Preuve}{myolive}
        Les trois premières propriétés découlent directement d’un travail sur les fonctions composantes dans une base donnée, la dernière est plus subtile. 

        Soit $t_k = a + k \frac{b-a}{n}$ pour $k \in \intervalleEntier{1}{n}$. Par inégalité triangulaire (discrète), 
        \[ \norm{S_n(f)} := \norm{\frac{b-a}{n} \sum_{k=1}^{n} f(t_k)} \leq \frac{b-a}{n} \sum_{k=1}^{n} \norm{f(t_k)} \eqlabel{\textdagger}\]   
        \begin{itemize}
            \item Comme $S_n(f) \limi{n}{+\infty} \int_{a}^{b} f$ et $\norm{.}$ est continue, 
            \[ \norm{S_n(f)} \limi{n}{+\infty} \norm{\int_{a}^{b} f(t)dt} \]   
            \item De même, par continuité de $\norm{f}$, 
            \[ \frac{b-a}{n} \sum_{k=1}^{n} \norm{f(t_k)} \limi{n}{+\infty} \int_{a}^{b} \norm{f} \]
        \end{itemize}
        On peut donc conclure en passant à l’inégalité dans (\textdagger).
    \end{demo}

    On retrouve également les théorèmes classiques de première année qui établissent le lien entre dérivation et intégration.

    \begin{theo}{TFCI}{}
        Soit $f$ une fonction continue sur un intervalle $I$ de $\mathbb{R}$ et à valeurs dans un e.v.n. de dimension finie.

        Alors pour tout $a \in I$, $F : x \mapsto \int_{a}^{x} f(t)dt$ est de classe $\mathcal{C}^1$ sur $I$ et $F' = f$.
    \end{theo}

    \begin{theo}{Taylor avec reste intégral à l’ordre $0$}{}
        Pour toute fonction $f : I \to E$ de classe $\mathcal{C}^1$, $\int_{a}^{b} f'(t)dt = f(b) - f(a)$.
    \end{theo}

    \begin{coro}{Inégalité des accroissements finis}{}
        Soit $f : I \to E$ une fonction de classe $\mathcal{C}^1$, où $I$ est un intervalle de $\mathbb{R}$. 

        S’il existe un réel $M > 0$ tel que pour tout $t \in I$, $\norm{f'(t)} \leq M$, alors 
        \[ \forall a,b \in I, \quad \norm{f(b) - f(a)} \leq M \cdotp \abs{b-a} \]
    \end{coro}

    \begin{demo}{Preuve}{myorange}
        Soient $a,b \in I$ vérifiant $a < b$. Alors 
        \[ \norm{f(b) - f(a)} = \norm{\int_{a}^{b} f'(t)dt} \leq \int_{a}^{b} \norm{f'(t)} dt \leq M \cdotp \abs{b-a} \]
    \end{demo}

    \begin{theo}{Formules de Taylor}{}
        Soit $f$ une fonction de classe $\mathcal{C}^{n+1}$ sur un intervalle $I$ de $\mathbb{R}$ et à valeurs dans un e.v.n. de dimension finie. Soient $a, x \in I$.
        \begin{enumerate}
            \item \textbf{Formule de Taylor avec reste intégral}
            \[ f(x) = \sum_{k=0}^{n} \frac{f^{(k)}(a)}{k!}(x-a)^k + \int_{a}^{x} \frac{f^{(n+1)}(t)}{n!} (x-t)^n dt \]
            \item \textbf{Inégalité de Taylor-Lagrange}
            \[ \norm{f(x) - \sum_{k=0}^{n} \frac{f^{(k)}(a)}{k!}(x-a)^k} \leq M \cdotp \frac{\abs{x-a}^{n+1}}{(n+1)!} \quad \text{avec } M = \sup_{t \in \intervalleFF{a}{b}} \norm{f^{(n)}(t)} \]
            \item \textbf{Formule de Taylor-Young}
            \[ f(x) = \sum_{k=0}^{n} \frac{f^{(k)}(a)}{k!}(x-a)^k + \comp{o}{x}{a}{(x-a)^n} \]
        \end{enumerate}
    \end{theo}

\subsection{Suites et séries de fonctions vectorielles}

    En vue de l’étude prochaine des équations différentielles, on généralise sommairement l’étude des suites et séries de fonctions numériques aux fonctions vectorielles.

    Par la suite, $(f_n)_{n \in \mathbb{N}}$ désignera une suite de fonctions $f_n : I \to E$ où $I$ désigne un intervalle de $\mathbb{R}$ et $(E,\norm{.})$ un espace vectoriel normé supposé de dimension finie.

    \begin{defi}{Convergences simple et uniforme d’une suite de fonctions vectorielles}{}
        \begin{itemize}
            \item On dit que la suite \textbf{converge simplement} vers la fonction $f$ sur $I$ si 
            \[ \forall x \in I, \quad f_n(x) \limi{n}{+\infty} f(x) \]   
            Autrement dit,
            \[ \forall \varepsilon > 0, \quad \forall x \in I, \quad \exists N \in \mathbb{N}, \forall n \geq N, \quad \norm{f_n(x) - f(x)} \leq \varepsilon \]
            \item On dit que la suite \textbf{converge uniformément} vers la fonction $f$ sur $I$ si $f_n - f$ est bornée à partir d’un certain rang sur $I$ et 
            \[ \nnorm{\infty}{f_n - f} \limi{n}{+\infty} 0 \esp{\textit{i.e.}} \sup_{x \in I} \norm{f_n(x) - f_n} \limi{n}{+\infty} 0 \]    
            Autrement dit,
            \[ \forall \varepsilon > 0, \quad \exists N \in \mathbb{N}, \forall n \geq N, \quad \forall x \in I, \quad \norm{f_n(x) - f(x)} \leq \varepsilon \]
        \end{itemize}
    \end{defi}

    Outre le fait que la convergence uniforme entraîne la convergence simple, on retrouve tous les résultats relatifs à la continuité, dérivabilité et intégrabilité de la limite d’une suite de fonctions à valeurs dans K. Par exemple :

    \begin{theo}{Continuité de la limite uniforme}{}
        La limite uniforme d’une suite convergente de fonctions continues sur $I$ est continue sur $I$.
    \end{theo}

    On considère désormais la série de fonctions $\sum f_n$ où $f_n : I \to E$.

    \begin{defi}{Convergences simple, uniforme et normale d’une série de fonctions vectorielles}{}
        \begin{itemize}
            \item On dit que $\sum f_n$ \textbf{converge simplement} sur $I$ si la suite de fonctions $(S_n)_{n \in \mathbb{N}}$ converge simplement sur $I$. 
            
            En cas de convergence, on appelle \textbf{fonction somme} de la série la fonction $S$ définie par 
            \[ \forall x \in I, \quad S(x) = \sum_{k=0}^{+\infty} f_k(x) = \lim_{n \to +\infty} \sum_{k=0}^{n} f_k(x) \]  
            \item On dit que $\sum f_n$ converge uniformément sur $I$ si la suite de fonctions $(S_n)_{n \in \mathbb{N}}$ converge uniformément sur $I$.
            \item On dit que $\sum f_n$ converge normalement sur $I$ si les fonctions $f_n$ sont bornées sur $I$ apcr., et si la série numérique $\sum \nnorm{\infty,I}{f_n}$ converge.
        \end{itemize}
    \end{defi}

    Bien entendu, si la série de fonctions converge normalement sur $I$ alors elle converge uniformément sur $I$.

    \begin{theo}{}{}
        Soit $\sum  f_n$ une série de fonctions de $I$ dans $E$ convergeant uniformément vers $f$ sur $I$ et soit $a \in I$.

        Si, pour tout $n \in \mathbb{N}$, $f_n$ est continue en $a$, alors $\sum_{n=0}^{+\infty} f_n$ est continue en $a$.
    \end{theo}

    On admet l’extension suivante du théorème.

    \begin{theo}{Théorème de la double limite}{}
        \begin{soit}
            \item $\sum f_n$ une série de fonctions de $I$ dans $E$
            \item $a$ un point adhérent à $I$
        \end{soit}
        \begin{suppose}
            \item Pour tout $n \in \mathbb{N}$, $f_n$ admet une limite $\ell_n$ en $a$.
            \item La série $\sum f_n$ converge uniformément sur $I$.
        \end{suppose}
        \begin{alors}
            \item La série $\sum \ell_n$ converge.
            \item La somme $\sum_{n=0}^{+\infty} f_n$ admet une limite en $a$
            \item $\lim_{x \to a} \sum_{n=0}^{+\infty} f_n(x) = \sum_{n=0}^{+\infty} \lim_{x \to a} f_n(x)$
        \end{alors}
    \end{theo}

    S’ajoutent les deux théorèmes suivants de dérivation sur un intervalle et d’intégration terme à terme \textbf{\textsc{sur un segment}}.

    \begin{theo}{Dérivation terme à terme}{}
        Soit $\sum f_n$ une série de fonctions définies sur un intervalle $I$ de $\mathbb{R}$, à valeurs dans $E$. \begin{suppose}
            \item Pour tout $n \in \mathbb{N}$, $f_n$ est de classe $\mathcal{C}^1$ sur $I$.
            \item $\sum f_n$ converge simplement sur $I$.
            \item $\sum f_n'$ converge uniformément sur tout segment de $I$.
        \end{suppose}
        \begin{alors}
            \item $\sum_{n=0}^{+\infty} f_n$ converge uniformément sur tout segment de $I$.
            \item $\sum_{n=0}^{+\infty} f_n$ est de classe $\mathcal{C}^1$ sur $I$.
            \item $\left(\sum_{n=0}^{+\infty} f_n\right) = \sum_{n=0}^{+\infty} f_n'$
        \end{alors}
    \end{theo}

    \begin{theo}{Intégration terme à terme sur un segment}{}
        Soit $\sum f_n$ une série de fonctions continues sur un \textbf{\textsc{segment}} $\intervalleFF{a}{b}$ et à valeurs dans $E$, convergeant uniformément sur le segment $\intervalleFF{a}{b}$. 

        Alors $\sum  \int_{a}^{b} f_n(x)dx$ converge et 
        \[ \sum_{n=0}^{+\infty} \left(\int_{a}^{b} f_n(x) dx\right) = \int_{a}^{b} \left(\sum_{n=0}^{+\infty} f_n(x)\right) dx \]
    \end{theo}

    En revanche, le théorème de convergence dominée et son homologue, le théorème d’intégration terme à terme sur un intervalle quelconque, ne s’appliquent pas dans le cadre du programme aux fonctions vectorielles.

\newpage